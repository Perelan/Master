\documentclass[UKenglish]{ifimaster}  %% ... or USenglish or norsk or nynorsk
\usepackage[utf8]{inputenc}           %% ... or latin1
\usepackage[T1]{fontenc,url}
\urlstyle{sf}
\usepackage{babel,textcomp,csquotes,duomasterforside,varioref,graphicx}
\usepackage[backend=biber,style=numeric-comp]{biblatex}

%\usepackage{subcaption}
\usepackage[hidelinks]{hyperref}
\usepackage{cleveref}
\usepackage{listings}
\usepackage{parskip}
\usepackage{alltt}
\usepackage{subfig}
\usepackage{todonotes}
\usepackage{enumitem}
\usepackage{textcomp}
\usepackage{epigraph}

\setlength{\parindent}{2em}
\setcounter{secnumdepth}{4}

\colorlet{punct}{red!60!black}
\definecolor{background}{HTML}{EEEEEE}
\definecolor{delim}{RGB}{20,105,176}
\colorlet{numb}{magenta!60!black}

\lstset{
    language=Java,
    columns=flexible,
    breaklines=true,
    keywordstyle=\color{black}\bfseries,
    showstringspaces=false,
    numbers=left,
    numberstyle=\tiny,
    stepnumber=4,
    numbersep=5pt,
    tabsize=4,
}

\lstdefinelanguage{json}{
    basicstyle=\normalfont\ttfamily,
    numbers=left,
    numberstyle=\scriptsize,
    stepnumber=1,
    numbersep=8pt,
    showstringspaces=false,
    breaklines=true,
    frame=lines,
    aboveskip=20pt,
}


\title{Nidra: An Extensible Android Application for Recording, Sharing and Analyzing \\ Breathing Data}
\subtitle{An Engineering Approach}
\author{Jagat Deep Singh}

\addbibresource{sources.bib}            %% ... or whatever

\includeonly{
    sections/introduction/index,
    sections/background/index,
    sections/relatedwork/index,
    sections/design/index,
    sections/implementation/index,
    sections/experiments/index,
    sections/conclusion/index,
    sections/appendix/index
}

\begin{document}
\duoforside[dept={Department of Informatics},   %% ... or your department
  program={Programming and Networks},  %% ... or your programme
  long]                                        %% ... or long

\frontmatter{}
\chapter*{Acknowledgements}
I would like to start of by thanking my supervisor, Professor Dr. Thomas Peter Plagemann, for his guidance throughout the work in this thesis. I am truely humble for the effort and dedication he put into helping me with this thesis. With my lacking experience of writing, this thesis would not be as complete. Therefore, I would sincere thank him. 

Next, I would like to thank all of my friends that I gained during the study, as well as my childhood friends, for the encouragement. Above all, I show my gratitude towards my parents for their unconditional love and care, and to my Mom for without her encouragements and support I would not be where I am today. To her, this degree means much more than it does to me.

Finally, I would like to thank Sagar Sen at SweetZpot Inc. for giving me valuable insights on how to operate with the Flow sensor kit. 

\vspace*{\fill}

\epigraph{\hfill{\textit{A man is but the product of his thoughts; what he thinks, \\he becomes.}}}{\textbf{Mahatma Gandhi}}

\chapter*{Abstract}
 A vast majority of medical examinations requires the presence of a patient at the hospital or laboratory. Statistics Norway \cite{ssb} presents that between 2017-2018 the cost of diagnosis, treatment, and rehabilitation in Norway increased with 7.3 percent for municipal health service. Likewise, the man-years for physicians in the municipal health service increased with 2.4 percent. As such, the growth of medical attendance results in more work and stress induced on the physicians and a higher demand for medical attention from the patients. To overcome this hurdle, leveraging the technology that partakes in the diagnosis, treatment, and rehabilitation can make it more convenient for both parties. In recent years, the mobile phone has become significantly advanced and powerful devices. With the possibility for creating applications that users can interact with, and the support of built-in sensors and external sensors source for collecting physiological signals---such as breathing data and hearth rate data---can aid in examinations of specific illnesses or disorder from home. Applications that focus on improving healthcare are known as mHealth applications, and various applications in the real-world try to analyze, observe, and diagnose the health implications for human beings with mobile technology \cite{contactless_sleep, sam, mobilesleeplab}. An excellent example of a mHealth application is the CESAR project, which aims to use low-cost sensor kits to monitor physiological signals during sleep in order to improve the diagnosis of obstructive sleep apnea (OSA) \cite{cesar}. The project facilitates a tool for managing and collecting physiological signals from external sensor sources (e.g., Bitalino) and the support for integrating future sensor sources. 
 
 In this thesis, we proceed to extend the project by designing and developing an Android application for patients to record, share, and analyze breathing data collected with the Flow sensor kit over an extended period. The motivation is to aid in collecting breathing data to potentially diagnose sleep apnea; albeit, the application can be used in other fields of study (e.g., physical activites). Also, we facilitate for an extensible application, that allows for future developers to create modules that extend the functionality in the application or enrich the data from the patient's records. The name of the application is Nidra---named after the Hindu goddess of sleep.

\tableofcontents

\addcontentsline{toc}{chapter}{List of Tables}
\listoftables

\addcontentsline{toc}{chapter}{List of Figures}
\listoffigures

\addcontentsline{toc}{chapter}{List of Listings}
\lstlistoflistings

\mainmatter{}

\part{Introduction and Background}
\chapter{Implementation}
\chapter{Implementation}
\chapter{Implementation}

\part{Design and Implementation}
\chapter{Implementation}
\chapter{Implementation}

\part{Evaluation and Conclusion}
\chapter{Implementation}

\chapter{Implementation}


\addcontentsline{toc}{part}{Appendix}
\part*{Appendix}

\appendix{}
\chapter{Implementation}

\backmatter{}
\printbibliography
\end{document}
