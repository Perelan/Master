\chapter{Related Work}
The detection and analysis of sleep-related breathing disorders on a mobile device has been a research topic and concert for some time. Various techniques and methods have been applied to detect sleep-related breathing disorders on mobile phone with the use of built-in or external sensors. In this chapter, we survey some of the research conducted in this field.

Nandakumar, Gollakota, and Watson \cite{contactless_sleep} present a contactless solution for detecting sleep apnea events on smartphones. The goal behind the research is to detect sleep apnea events without any sensor on the human body. They achieved this by transforming the phone into an active sonar system that emits frequency-modulated sound signals and observes for the reflections. Based on the experiments, the system operates efficiently at distances of up to a meter, also while the subject is under a blanket. They performed a clinical study with 37 patients and concluded that the system managed to accurately compute the central, obstructive, and hypopnea events (with a correlation coefficient of 0.99, 0.98, and 0.95). 

Alqassim \textit{et al.} \cite{sam} designed and implemented a mobile application (for Windows and Android) to monitor and detect symptoms of sleep apnea using built-in sensors in the smartphone. The purpose of the application is to make users aware of whether they have sleep apnea before they continue with a more expensive and advanced sleep test. They achieved this by measuring the breathing patterns and movements patterns based on the built-in microphone and accelerometer in the smartphone. The system instructs the user to place the smartphone on its arm, abdomen, or near the bed during recording. The data is collected on the smartphone and sent to a central server in the cloud, where authorized doctors can review the samples. To summarize, the system tries to monitor sleep apnea in the aspect of motion and voice recorder, in order to detect sleep apnea on a smartphone. 

Penzel \textit{et al.} \cite{mobilesleeplab} investigates the challenges and develop a system associated with insufficient conventional sleep laboratories and their expensive and time-consuming polysomnographic diagnostics as early as in 1989. The purpose of the system is to make the diagnosis of sleep-related breathing disorders available at any hospital; the system was placed in a wooden case with wheels to be moved between bedside locations. They developed a circuit board that has support for various sensor to evaluate breathing, ECG, blood gases, and the state of sleep. During record, the samples from the sensors were compressed to a resolution of one value per second per parameter (e.g., mean value of respiratory frequency, ventilated volume, actigraph activity, and EOG activity) and stored on a personal computer. After the recording, manual evaluation can be carried out using print-outs, as well as reviewing the data on a screen. Thus, the software of the system facilitates recording and reviewing of the data, also basic evaluation and analysis. 

\section{Summary \& Conclusion}

To summarize, Nandakumar, Gollakota, and Watson created an application to collect samples through a contactless solution, which quite accurately measures central, obstructive, and hypopnea events. Alqassim \textit{et al.} developed a mobile application to sample breathing patterns and movement patterns with the use of the built-in microphone and accelerometer in smartphones. Finally, Penzel \textit{et al.} built a mobile system to record and analyze sleep-related breathing disorders with technology that was advanced at the time.

To conclude this chapter, the developing of a system that records, analyzes, and detects sleep-related breathing disorders (e.g., obstructive sleep apnea) proposes a solution to aid researchers/doctors in analyzing and detecting sleep apnea in patients. Based on the related work in this chapter,  we observe that there are distinguishable techniques and methods for the detection and analyzing of sleep-related breathing disorders through the use of mobile systems. Considering this, the future might introduce more techniques and methods that improve the analysis and detection of sleep apnea. Thus, developing a system that is extensible and modular to new techniques and methods should be considered. Therefore, we look into the opportunities of creating an application that enables the support for third-party modules, these modules extend the functionality of the application or provide more enrichment to the collected data on the patients' device.