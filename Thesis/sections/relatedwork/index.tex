\chapter{Related Work}
The widespread adoption of detecting and analyzing sleep-related breathing disorders on a mobile device has been a research topic and concert for some time. Various techniques and methods has been applied to detect sleep-related breathing disorders on a mobile phone with the use of built-in or external sensors. We will survey some of the research conducted in this field, and in the end, we will present similiarites based on our system. 

Nandakumar, Gollakota, and Watson \cite{contactless_sleep} presents a contactless solution for detecting sleep apnea events on smartphones. The goal behind the research is to detect sleep apnea events without any sensor on the human body. They achieved this by transforming the phone into an active sonar system, that emits frequency-modulated sound signals and observe for the reflections. Based on the experiments, the system operates efficiently at distances of up to a meter, also while the subject is under a blanket. They performed a clincal study with 37 patients, and concluded that the system managed to accurately compute the central, obstructive, and hyponea events (with a correlation coefficient of 0.99, 0.98, and 0.95). 

Alqassim \textit{et al.} \cite{sam} designed and implemented a mobile application (for Windows and Android) to monitor and detect symptoms of sleep apnea using built-in sensors in the smart phone. The purpose of the application is to make users aware of whether they have sleep apnea, before they continue with a more expensive and advanced sleep tests. They achieved this by measuring the breathing patterns and movements patterns based on the built-in microphone and accelerometer in the smart phone. The system instructs the user to place the smart phone on its arm, abdomen or near the bed during recording. The data is collected on the smart phone, and sent to a centeral server in the cloud, where authroized docots can review the samples. To summarize, the system tries to monitor sleep apnea in the aspect of motion and voice recorder, in order to detect sleep apnea on a smart phone. 

Penzel \textit{et al.} \cite{mobilesleeplab} investigates the challenges and develop a system assoicated with insufficent conventional sleep laboratoris and their expensive and time-consuming polysomnographic diagnosistics as early as in 1989. The purpose of the system is to make diagnose of sleep-related breathing disorders avaiable at any hosptial; the system was placed in a wooden case with wheels to be moved between bedside locations. They developed a circuit board that has support for various sensor to evaluate breathing, ECG, blood gases and the state of sleep. During record, the samples from the sensors were compressed to a resolutaion of one value per second per parameter (e.g., mean value of respiratory frequency, ventilated volumn, actigraph actvity, and EOG actvity) and stored on a personal computer. After the recording, manual evalutation can be carried out using print-outs, as well as reviewing the data on a screen. Thus, the software of the system facilitates recording and reviewing of the data, also basical evalutation and analysis. 

To summarize, Nandakumar, Gollakota, and Watson created an application to collect samples through a contactless solution, which quite accurately measures central, obstructive, and hyponea events. Alqassim \textit{et al.} developed an mobile application to sample breathing patterns and movement patterns with the use of the built-in microphone and accelerometer in the smartphones. And Penzel \textit{et al.} built a mobile system to record and analyze sleep-related breathing disords with technology that was advanced at the time.

To conclude this chapther, we can derive that the demand for a mobile system to detect sleep-related breathing disorders is high. Developing a system that records, analysis and detects sleep-related breathing disorders (e.g., Obstructive Sleep Apnea) enables users to diagnose the sleeping disorder from home, in contrast to a expensive and time-consuming polysomnographic diagnosistics at a labratory. Therefore, developing a system that is extensible and modular to new techiniques and methods, as well as for future sensors devices, is approriate in order to extend and evolve the research behind the detecting sleep-related breathing disorders on a mobile device. We developed an application for Android which enables recording, sharing and analytics, in addition to support for third-party modules to extend the functionality of the application, and to enrich the data collected on a patients device. The application is called \textit{Nidra}, which we present in the next chapter. 
