
\subsection{Recording}\label{soc:recording}
The structure of a recording is restrictive in terms of arranging the components due to the design of the CESAR project. That is, the recording structure is based on the interface (e.g., establishment and receiving data) facilitated for the sensor sources by the project. Besides this, the structure can be extended with components that monitor the data acquisition and the connectivity with the sensor and the device. In Figure \ref{fig:hta_recording}, a proposition of the structure for recording with components and their dependencies is illustrated, and the components are described as:

\begin{figure}
    \centering
    \includegraphics[scale=0.7]{images/Recording_Design.pdf}
    \caption{A design proposal for the structure of the recording concern.}
    \label{fig:hta_recording}
\end{figure}

\begin{itemize}
    \item[1.1] \verb|Sensor Discover:| Find all eligible sensors that can enable a recording.
    \item[1.1.1] \verb|Select Sensors:| Select a preferable sensor source for data acquisition. 
    \item[1.1.2] \verb|All Sensors:| Select all available sensors source for data acquisition.
    \item[1.2] \verb|Sensor Initialization:| Establish and initialize a connection with the selected sensor source. Occasionally a sensor might use some time to connect, or unforeseen occurrence is hindering the initialization of the sensor. Therefore, halting the sensor initalization or actively checking for sensor initalization is important. 
    \item[1.3] \verb|Sensor Connectivity Setup:| Establish a connection between the application and the sensor source using the protocols the sensor sources provides (e.g., BlueTooth). All data exchange will occur over the established interface. 
    \item[1.4] \verb|Connection State:| Is the state that determines whether the recording has started (ongoing) or stopped (finalized).
    \item[1.4.1] \verb|Start:| Notify the sensors to begin collecting data, and the application should display that a recording has started to the user. Also, start a timer to display time elapsed on the current recording. 
    \item[1.4.1.1] \verb|Monitor:| Is a mechanism to handle the connectivty state and the incoming samples. It is actively listening to the interface for new data from the sensors, and forwarding the data to the sample management component.  
    \item[1.4.1.1.1] \verb|Sample Management:| Handles a single sample from a sensor by parsing the content according to the payload of the data tuple from the sensor.
    \item[1.4.1.1.2] \verb|Connectivity State:| Makes sure that the connectivity between the sensor and the device is maintained. Further described in Section \ref{sssec:csc}.
    \item[1.4.2] \verb|Stop:| Stop the recording timer and proceed to display results.
    \item[1.4.2.1] \verb|Sensor Finalization:| Notify the sensor to stop sampling data, and close establishment.
    \item[1.4.2.2] \verb|Close Sensor Connectivity:| Close the interface connections between the application and the connected sensors. 
    \item[1.4.2.3] \verb|Stop Recording:| Once all connections are closed, add additional information to the recording (e.g., title, description, rating). In the end, the recording has concluded and it is stored on the mobile device.
\end{itemize}

\subsubsection{Connectivity State Component} \label{sssec:csc}
Connectivity state is a component that monitors for unexpected sensor disconnections or disruptions. Unexpected behavior can occur due to anomalies in the sensor, or the sensor being out of reach from the device for a brief moment.  A naive solution would be to ignore the connectivity state component, and assuming the sensors are connected to the device indefinitely. However, upon disconnections or disruptions, the recording would be missing samples, resulting the record has less meaningful data. This component enables the application to reconnect with the sensor based on a time interval, resulting in more accurate and meaningful record. The following design questions for this component are (1) should the connectivity state component, which implements a time interval that tries to reconnect with the sensor, be implemented in the sensor wrapper, or should it be in the proposed recording structure?; and (2) should the interval between sample arrival be a fixed time or a dynamical time? 


\begin{enumerate}
    \item To achieve a mechanism of reconnecting with the sensor on unexpected disconnects or disruptions, establishing a time interval that monitors for sample arrivals within a time frame (e.g., every 10 seconds) is required. Incorporating the time interval in the sensor wrapper reduces the complexity of Nidra. However, it introduces extra complexity to the sensor wrappers. A sensor wrapper has to distinguish actual disconnects from unexpected disconnects. Although, by extending the functionality of sensor wrapper by implementing a state that indicates whether a recording is undergoing or stopped solves the problem. All future sensor wrappers would then have to implement the proposed solution, resulting in a complicated and time-consuming sensor wrappers development. While implementing the proposed solution in the sensor wrappers is possible, extending the recording structure with the logic in Nidra would be more meaningful and time-saving. In our design, we implement the connectivity state in the structure for recording.
    \item A time interval triggers an event every specified time frame. A time frame can be in a fixed size (e.g., every 10 seconds) or a dynamical size (e.g., start with 10 seconds, then incrementally increase the frame by X seconds). Implementing a fixed time frame increases the stress on put on the sensor, whereas a dynamical time might miss samples if the time frame is significant. Depending on how critical the recording is, a suitable solution for the time frame should be configurable. Also, limiting the number of attempts made to reconnect should be considered, due to actively reconnecting to a sensor that presumably is dead or completely out of reach is unnecessary. Thus, stopping the recording once a limited number of attempts has been reached.  In our design, we implemented a dynamical time and limited the number of attempts to 10. 
\end{enumerate}

\subsection{Sharing} \label{sec:design_sharing}

Sharing is separated into two concerns: export and import. The scope of exporting in Nidra is to select desired records, format and bundle the records into a transmittable file, and sendting the bundle over a media (e.g., mail). The scope of importing is to locate the file on the device, parse the content based on the format, and store it on the device. In Figure \ref{fig:hta_sharing}, a proposition of the structure for sharing with components and their dependencies is illustrated, and the components are described as:


\begin{figure}
    \centering
    \includegraphics[scale=0.8]{images/Sharing_Design.pdf}
    \caption{A design proposal for the structure of the sharing concern.}
    \label{fig:hta_sharing}
\end{figure}

\begin{itemize}
    \item[2.1] \verb|Import|: Is a mechanism that locates a file, parses the data, and stores it on the device.
    \item[2.1.1] \verb|Locate File|: To enable this, the user has to download the file on the device. Then, locate the file on the device by using an interface to browse downloaded files. An interface can be developed; however, using the interface Android provides to locate downloaded files is a more straightforward solution.
    \item[2.1.2] \verb|Parse Formated Content|: Parse the content of the file accordingly to the data format discussed in \Cref{sec:dataformat}.
    \item[2.1.3] \verb|Insert Storage|: Retrieve the necessary data from the parsed file, to store on the device without overriding existing data.   
    \item[2.2] \verb|Export|: Is a mechanism that selects all or a specific record, transforms the record into a formatted file (see \Cref{sec:dataformat}), and transmits the file across application (e.g., sending records from the patient's installed application to the researcher/doctor's installed application with the use of an e-mail application). 
    \item[2.2.1.1] \verb|All Records|: Export all of the records on the device.
    \item[2.2.1.2] \verb|Select Record|: Pick one specific record to export. 
    \item[2.2.2] \verb|Format File|: When a preferred format for the records is selected, bundling the data into a formatted (see \Cref{sec:dataformat}) file for transmission. It is essential to identify the name of the file uniquely to prevent duplicates and overrides of data. For instance, identifying the name of the file with the device identification appended with the time of exporting. 
    \item[2.2.3] \verb|Distribute|: Send the file over a media, e.g., e-mail (see Section \ref{des:distro}).
\end{itemize}

\subsubsection{Distribute Component}\label{des:distro}
The distribute component uses the formatted file and transfer it to other instances of Nidra using a media (e.g., e-mail). There are two distinctive methods to perform this task which is efficient and practical: (1) establish a web-server with logic to handle users and sharing data with the desired recipient, and implementing an interface to retrieve the file within the application; and (2) using the interface provided by Android to share files across applications (e.g., e-mail). 

While the first option might be favorable in terms of practicality, this solution introduces additional concerns (e.g., the privacy matters of storing user data on a server) which is out of the scope for this project. For this reason, using the interface provided by Android is a reasonable solution. The user of the application can utilize the Android interface for sharing files over installed applications (e.g., the e-mail is a flexible media to transfer the file, and the user can specify the recipients accordingly).  



\subsection{Modules}\label{soc:modules}

Modules are independent applications that provide extended functionality or data enrichment to Nidra. The components for locating and launching a module is given by the Android design; however, the component for data exchange between a module and Nidra can be independently designed.  In Figure \ref{fig:hta_modules}, a proposition of the structure for modules with components and their dependencies is illustrated, and the components are described as:

\begin{itemize}
    \item[3.1.1] \verb|Install Modules|: Is the process of locating the application on the device, and storing the reference of the application package name in the storage.  
    \item[3.1.2] \verb|Locate Module|: Retrieve the list of stored modules, and display the installed modules to the user. 
    \item[3.2] \verb|Launch Module|: Get the application location stored in the module, and launch the application with the use of Android Intent. 
    \item[3.3] \verb|Data Exchange|: Enrich the module with data from the application (see Section \ref{des:des}).
\end{itemize}


\begin{figure}[!h]
    \centering
    \includegraphics[scale=0.8]{images/Modules_Design.pdf}
    \caption{A design proposal for the structure of the modules concern.}
    \label{fig:hta_modules}
\end{figure}

\subsubsection{Data Exchange Component}\label{des:des}

The data exchange component facilitates the transportation of data between Nidra and a module. As of now, the data is records and corresponding samples, which is formatted (see Section \ref{sec:dataformat}) accordingly. The two distinct methods to exchange data between a module and Nidra are (1) formatting all of the data and bundling it into the launch of the module, and (2) establishing a communication link for bi-directional requests between Nidra and the module. 

Android provides an interface to attach extra data on activity launch. The first solution is, therefore, convenient and efficient; all of the data is formatted and bundled into the launch. However, once Nidra has launched the module, there are no ways of transmitting new data besides relaunching the module. For this reason, the second option allows for continuous data flow by establishing a communication link with IPC between the applications. The data exchange between Nidra and modules can then be bidirectional; the module can request desired data any time, and Nidra can collect reports and results generated by the module. 

One could argue that new records are not obtained while managing and using a module. However, there might be future modules that do a real-time analysis of a recording; as such, require an interface for continuous data flow. For the simplicity of our design, we use the first option of bundling all of the data and sending it on launch.

\subsection{Analytics}\label{soc:analytics}

\begin{figure}
    \centering
    \includegraphics[scale=0.8]{images/Analytics_Design.pdf}
    \caption{A design proposal for the structure of the analytics concern.}
    \label{fig:hta_analytics}
\end{figure}

Analytics uses techniques and methods to gain valuable knowledge from data. Nidra provides a simple time-series plot for illustration of the data. However, other techniques can be in incorporated. In essence, the facilitation of modules in the application enables the development opportunities for advanced analytics of the data. In Figure \ref{fig:hta_analytics}, the structure of analytics is presented with components and their dependencies: 

\begin{itemize}
    \item[4.1.1] \verb|Select Record|: Select one of the records on the device.
    \item[4.1.2] \verb|All Records|: Select all of the records on the device.
    \item[4.2] \verb|Analytic Method|: Apply an analytical method on the data provided in Nidra (see Section \ref{des:amc}).
\end{itemize}

\subsubsection{Analytic Method Component} \label{des:amc}
The analytic method component uses the records on the device for representation or analysis. Graphical and non-graphical are two techniques for representing data. Graphical techniques visualize the data in a graph that enables analysis in various ways. A few graphical techniques are diagrams, charts, and time series. Non-graphical techniques, better known as statistical data tables, represent the data in tabular format. This provides a measurement of two or more values at a time \cite{datarepresentation}. More advanced techniques to analyze the data are to use machine learning. Machine learning is concerned with developing data-driven algorithms, which can learn from observations without explicit instructions. For example, using recurrent neural networks (e.g., RNN, LSTM) or regression models (e.g., ARIMA), can be used to predict the sleeping patterns \cite{machinelearning}.

In Nidra, a time series graph is used to represent the data of a record. The time series graph represents the respiration data on the Y-axis and the time on the X-axis. Essentially, the facilitation of modules in the application is designed to enable advanced techniques to predict, analyze, and interpret the data acquisition. Therefore, in Nidra, the analytic methods are limited; however, the modules enable developers to construct any method they desire. 

\subsection{Storage}\label{soc:storage}

\begin{figure}
    \centering
    \includegraphics[scale=0.8]{images/Storage_Design.pdf}
    \caption{A design proposal for the structure of the storage concern.}
    \label{fig:hta_storage}
\end{figure}

Storage is the objective of achieving persistent data; data remain available after application termination. The data is characterized into four data entities (i.e., record, sample, module, and user) that contain individual properties.  The components in the storage structure are constructed to be extensible and scalable in terms of future data, restructure of data, and removal of data.  In Figure \ref{fig:hta_storage}, a proposition of the structure for storage with components and their dependencies is illustrated, and the components are described as:

\begin{itemize}
    \item[5.1] \verb|Data Access Layer|: It exposes specific operations (such as insertion of a record) without revealing the database logic. This component is also known as a data access object (DAO), that provides an abstract interface to a database. The advantage of this interface is to have a single entry point for each database operation and to extend and modify the operation for future data easily. 
    \item[5.2] \verb|Database|: Is the storage of all the data (see Section \ref{des:dc}).
    \item[5.2.1] \verb|Record|: Is a table in the database, which contains fields appropriately to the record structure. A record contains meta-data about a recording (e.g., name, recording time, user). The design decision and an example of a recording record are illustrated in \Cref{ssec:record}.
    \item[5.2.2] \verb|Sample|: Is a table in the database, which contains fields appropriately to a single sample from a sensor. A sample contains data received from a sensor during a recording. The design decision and an example of a sample record are illustrated in \Cref{ssec:sample}.
    \item[5.2.3] \verb|Module|: Is a table in the database, which contains fields appropriately to the module structure. A module contains the name of the module-application and a reference to the application package. The design decision and an example of a module record are illustrated in \Cref{ssec:module}.
    \item[5.2.4] \verb|User|: Is an object stored on the device, which contains fields appropriately to the user structure. A user contains the patient's biometrical information (e.g., name, weight, height). The design decision and an example of a user are illustrated in \Cref{ssec:user}.
\end{itemize}

\subsubsection{Database Component}\label{des:dc}

\noindent Android provides several options to store data on the device; depending on space requirement, type of data that needs to be stored, and whether the data should be private or accessible to other applications. Two suitable options for storage are (1) internal file storage: storing files to the internal storage private to the application; and (2) database: Android provides full support for SQLite databases, and the database access is private to the application \cite{android_storage}. Based on the options, the design question for this component is should the data be stored in a flat file database on the internal file storage, or should it be stored in an SQLite database?

Flat files database encode a database model (e.g., table) as a collection of records with no structured relationship, into a plain text or binary file. For instance, each line of text holds on a record of data, and the fields are separable by delimiters (e.g., comma or tabs). Another possibility is to encode the data in a preferable data format, such as JSON or CSV files (see \Cref{sec:dataformat}). Flat file databases are easy to use and suited for small scale use; however, they provide no type of security, there is redundancy in the data, and integrity problems \cite{flatfilerdbms}. Locating a record is made possible by loading the file, and systematically iterating until the desired record is found. Similarly, updating a record and deleting a record. Consequently, the design of flat file databases is for simple and limited storage.

SQLite is a relational database management system, which is embedded and supported in Android. A relational database management system (RDBMS) provides data storage in fields and record, represented as columns (fields) and rows (records), in a table. The advantage is the ability to index records, relations between data stored in tables, and support querying of complex data with a query language (e.g., SQL). Also, RDBMS provides data integrity between transactions, improved security, and backup and recovery controls \cite{flatfilerdbms}. 

While a flat-file database is applicable to store small and unchangeable data, it is not suitable for scalable and invasive data change. The samples acquisition during recording makes it unreasonable to use a flat-file database. In the design of Nidra, SQLite is a preferable solution for storing sample entities. As a result, the record entity is be stored in an SQLite database, in order to associate it with the sample entity (with a foreign key). As for the module entity, it is more convenient to store and update a list of records inside a database; therefore, we use the SQLite database for the module entity. Finally, the user entity is stored in the flat file, that is because there only is one user in the application a time; hence, it is more convenient to use a flat-file (e.g., JSON) to store this type of information.

\subsection{Presentation}\label{soc:presentation}
\begin{figure}
    \centering
    \includegraphics[scale=0.7]{images/Presentation_Design.pdf}
    \caption{A design proposal for the structure of the presentation concern.}
    \label{fig:hta_presentation}
\end{figure}

Presentation facilitates the user interface of the application, in terms of visualizing the functionality of the application to the user. The user interface design is guided by the functionality (concerns discussed) in the application and iterative work on the topic. In Figure \ref{fig:hta_presentation}, a proposition of the structure for presentation with components and their dependencies is illustrated, and the components are described as:

\begin{itemize}
    \item[6.1] \verb|UI|: The user interface where interaction between users and the application occurs. 
    \item[6.1.1] \verb|Navigation|: The navigation is a menu with options to change the screen (e.g., feed, recording, and module). 
    \item[6.1.2] \verb|Layout|: Is based on the functionality that is exposed to the user. The layout consists of incorporating the color palette and the screens in the application (the subsequent components). 
    \item[6.1.3] \verb|Color Palette|: A color palette is a set of colors that persist throughout the application (see Section \ref{des:cpc}).
    \item[6.1.1.1] \verb|Settings|: Is s screen with user details, permissions, and credits, with options to modify permission and user details.
    \item[6.1.1.2] \verb|Feed|: Is a list of all records for the user, displayed with details specific to the record to make it distinguishable and easily recognizable. 
    \item[6.1.1.3] \verb|Analytics|: A interactive time-series graph for a single record. 
    \item[6.1.1.4] \verb|Recording|: The process of establishing a recording session, in addition to showing the results after a recording session has ended. 
    \item[6.1.1.5] \verb|Onboarding|: The initial screen displayed to the user, where the user can supply the application with their biometrical data. 
    \item[6.1.1.6] \verb|Modules|: A list of all installed modules, also an option to add more modules. 
    \item[6.1.4] \verb|Feedback & Interactions|: Each screen has different feedback and interaction, which should be handled appropriately. 
\end{itemize}

\subsubsection{Color Palette Component}\label{des:cpc}
The color palette is a component that decides the color scheme in the application. In the proposal of a color system in the design guidelines by Google \cite{colorsystem}, it is essential to pick colors that reflect the style of the application accordingly to: (1) \textit{primary colors}: the most frequently displayed color in the application; (2) \textit{secondary colors}: provides an accent and distinguish color in the application; and (3) \textit{surface, background, error, typography and iconography colors}: colors that reflect the primary and secondary color choices. 

Moreover, choosing colors that meet the purpose of the application is critical. Nidra is most likely to be used during the evening and the morning. According to Google \cite{darktheme}, a dark color theme reduces luminance emitted by the device screen, which reduces eye strain, while still meeting the minimum color contrast ratios, and conserving battery power. Therefore, in our design, we chose a dark color theme (more specifically, a nuance of dark-blue). 