\subsection{Stakeholders}
McGrath and Whitty \cite{stakeholderdefined} describe the term stakeholder as those persons or organizations that have, or claim an interest in the project. They distinguish stakeholders into four categories: 1) \textit{contributing (primary) stakeholders} participate in developing and sustaining the project; 2) \textit{observer (secondary) stakeholders} affect or influence the project;  3) \textit{end-users (tertiary stakeholder)} interact and uses the output of the application; and 4) \textit{invested stakeholders} has control of the project. In Nidra, there are three stakeholders who affect the application, and each can be categorized respectfully:
\begin{itemize}
    \item \verb|Patients|: Are identified as an end-user; they interact with the application.  
    \item \verb|Researchers/Doctors|: Are identified as an observer stakeholder; they might not use the application itself; however, they might use the data obtained from the patients' recordings for further analysis. Also, request functionality in the application.
    \item \verb|Developers|: Are identified as a contributor stakeholder; they maintain the application from bugs or extend the functionally of the application. Additionally, they can contribute to developing modules that extend the functionality of the application. 
\end{itemize}

\subsection{Resource Efficiency}
The application is designed for the use on a mobile device; modern mobile devices are empowered with multi-core processors, a sufficient amount of ROM, and a variety of sensors. However, the battery capacity is restrictive and based on usage. The device may only last for one day before a charge, due to the size of the battery capacity \cite{androidbattery}. The average battery capacity of a mobile device is approximately 2000 mAh on budget devices and around 3000 mAh on high-end devices \cite{androidbatteryavg}. The application should be able to run at least 7 hours without any power supply. Also, the device should be capable of handling various sensor connections simultaneously. Therefore, the application should be designed to be resource efficient, by utilizing least amount of battery resources during a recording. Also, ensure sufficent amount of power on the device before starting a recording session.  

\subsection{Security and Privacy}
The proposed use of the application is to monitor the sleeping patterns of a patient. The application manages and stores personal- and health-related data about the patient. As a precaution, the application should incorporate the CIA triad, which stresses data confidentiality, integrity, and availability \cite{cia}. Any unauthorized access to the data, data leaks, and confidentiality should be appropriately managed on the device. Sharing the data across application or with researchers/doctors should be granted with the consent from the patient.  Besides, a mobile device can be connected to the Internet, which makes it vulnerable to attacks. Also, other installed application on the device can manipulate the access to the data. Therefore, revising the security policy defined by Android \cite{androidsecurity} should be incentivized. 