%\subsubsection{System Tasks}

\noindent\textbf{Recording}

\noindent A \textit{recording} is a process of collecting and storing physiological signals from sensors over an extended period (e.g., overnight). To enable a recording, we need to establish connections to available sensors, collect samples from the sensors, and storing the samples on the device. A \textit{sensor} is a device that transforms analog signals from the real world into digital signals. The digital signals are transmittable over Link Layer technologies (e.g., Bluetooth), and the communication between a sensor and device occurs over an application programming interface (API). A \textit{sample} is a single sensor reading containing data and metadata, such as time and the physiological data. During a recording session, ensuring for a consistent and uninterrupted data stream from the sensors is vital to obtaining persistent and meaningful data.  Once a recording session has terminated, a \textit{record} with metadata about the recording session is stored, alongside the samples.    

\noindent \textbf{Sharing}

\noindent Sharing is a mechanism to export and import records across applications. \textit{Exporting} consists of bundling one or more records with correlated samples into a transmittable format, and transferring the bundled records over a media (e.g., mail). \textit{Importing}, on the other hand, consists of locating the bundled records on the device, parsing the content and storing it on the device. [skrive mer?]

\noindent \textbf{Module}

\noindent A \textit{module} is an independent application that can be installed and launched in Nidra (hereafter: application), to provide extended functionality and data enrichment. A module does not necessarily interact with the application; however, it utilizes the data (e.g., records). For example, a module could be using the records to feed a machine learning algorithm to predict obstructive sleep apnea. Installing a module is achieved by locating the module-application on the device, and storing the reference in the application. Due to limitations in Android, the module-application cannot be executed within the application. Therefore, the module-application is a standalone Android application; furthermore, the development of the module-application is independent from the application. 

\noindent \textbf{Analytics}

\noindent Analytics is the visualization and interpretation of patterns in the records. The application facilitates the recording of physiological signals, which enables the detection and analysis of sleep-related illnesses. There are various analytical methods, ranging from graphs to advanced machine learning algorithms. Incorporating a simple time series plot can indirectly aid in the analysis. For instance, plotting a time series graph where the physiological signals are on the Y-axis and the time on X-axis, provides a graphical representation of the data that can be further analyzed within the application.

\noindent \textbf{Storage}

\noindent Storage is the objective of achieving persistent data; data remain available after application termination. To enable storage, we use a database for a collection of related data that is easily accessed, managed, and updated. The database should be able to store records, samples, modules, and biometrical data related to user (i.e., gender, age, height, and weight). Structuring a database that is reliable, efficient, and secure is a crucial part of achieving persistent storage. Android provides several options to enable storage on the device (e.g., internal storage and database).

\noindent \textbf{Presentation}

\noindent Presentation is the concept of exhibiting the functionality of the application to the user. A user interface (UI) is the part of the system that facilitates interaction between the user and the system. In Nidra, determining the layout and view of the application, color palette, interactions, and feedback on actions is part of the development of a user interface. 