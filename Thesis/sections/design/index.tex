
\chapter{Analysis and High-Level Design}

It is the goal of this thesis to enable detection of sleep-related disorders with the aid of an Android device and low-cost sensors, and to further analyze and evaluate sleep- and breath-related patterns. We developed an application, called \textit{Nidra}, which attempts to collect, analyze, and share data collected from external sensors, all on a mobile device. Also, Nidra acts as a platform for modules to enrich the data, thus extending the functionality of the application.

The motivation behind this application is to provide an interface for patients to potentially run a self-diagnostic from home, and to aid researchers and doctors with analysis of sleep- and breathing-related illnesses (e.g., Obstructive Sleep Apnea). An overview of the Nidra application pipeline is found in Figure \ref{fig:parts}, beginning with data acquired from a sensor, and ending with the data in the Nidra application. As for now, Nidra consists of three main functionalities, each related to the requirements defined in the problem statement: 

\begin{enumerate}
    \item The application must provide an interface for the patient to (i) record physiological signals (e.g., breathing data); (ii) present the results; and (iii) share the results.
    \item The application must ensure that upon sensor disconnections, the connectivity is reinstated to minimize the data loss and its effects on the data analysis.
    \item The application must provide an interface for the developers to create modules that integrate with the application.
\end{enumerate}

This chapter will give a detailed look at the design of Nidra, including the tasks which constitute the structure of the application, the separate concerns in relation to the tasks, and the structure of the data in the application.

\section{Requirement Analysis}

\subsection{Stakeholders}
McGrath and Whitty \cite{stakeholderdefined} describe the term stakeholder as those persons or organizations that have, or claim an interest in the project. They distinguish stakeholders into four categories: 1) \textit{contributing (primary) stakeholders} participate in developing and sustaining the project; 2) \textit{observer (secondary) stakeholders} affect or influence the project;  3) \textit{end-users (tertiary stakeholder)} interact and uses the output of the application; and 4) \textit{invested stakeholders} has control of the project. In Nidra, there are three stakeholders who affect the application, and each can be categorized respectfully:
\begin{itemize}
    \item \verb|Patients|: Are identified as an end-user; they interact with the application.  
    \item \verb|Researchers/Doctors|: Are identified as an observer stakeholder; they might not use the application itself; however, they might use the data obtained from the patients' recordings for further analysis. Also, request functionality in the application.
    \item \verb|Developers|: Are identified as a contributor stakeholder; they maintain the application from bugs or extend the functionally of the application. Additionally, they can contribute to developing modules that extend the functionality of the application. 
\end{itemize}

\subsection{Resource Efficiency}
The application is designed for the use on a mobile device; modern mobile devices are empowered with multi-core processors, a sufficient amount of ROM, and a variety of sensors. However, the battery capacity is restrictive and based on usage. The device may only last for one day before a charge, due to the size of the battery capacity \cite{androidbattery}. The average battery capacity of a mobile device is approximately 2000 mAh on budget devices and around 3000 mAh on high-end devices \cite{androidbatteryavg}. The application should be able to run at least 7 hours without any power supply. Also, the device should be capable of handling various sensor connections simultaneously. Therefore, the application should be designed to be resource efficient, by utilizing least amount of battery resources during a recording. Also, ensure sufficent amount of power on the device before starting a recording session.  

\subsection{Security and Privacy}
The proposed use of the application is to monitor the sleeping patterns of a patient. The application manages and stores personal- and health-related data about the patient. As a consequence, the application should incorporate the CIA triad, which stresses data confidentiality, integrity, and availability \cite{cia}. Any unauthorized access to the data, data leaks, and confidentiality should be appropriately managed on the device. Sharing the data across application or with researchers/doctors should be granted with the consent from the patient.  Besides, a mobile device can be connected to the Internet, which makes it vulnerable to attacks. Also, other installed application on the device can manipulate the access to the data. Therefore, revising the security policy defined by Android \cite{androidsecurity} should be incentivized. 

\section{High-Level Design}

\subsection{Task Analysis}
Task analysis is a methodology to facilitate the design of complex systems. Hierarchical task analysis (HTA) is an underlying technique that analyzes and decomposes complex tasks such as planning, diagnosis, and decision making into specific subtasks \cite{ta}. In Figure \ref{fig:hta_overview}, an illustration of the tasks of the application is presented. This section introduces these tasks, which are an integral part of the development of the application.

\begin{figure}
    \centering
    \includegraphics[scale=0.23]{images/TA.png}
    \caption{The individual tasks that outline the application: (1) recording, (2) sharing, (3) modules, (4) analytics, (5) storage, and (6) presentation.}
    \label{fig:hta_overview}
\end{figure}

%\subsubsection{System Tasks}

\noindent\textbf{Recording}

\noindent A \textit{recording} is a process of collecting and storing physiological signals from sensors over an extended period (e.g., overnight). To enable a recording, we need to establish connections to available sensors, collect samples from the sensors, and storing the samples on the device. A \textit{sensor} is a device that transforms analog signals from the real world into digital signals. The digital signals are transmittable over Link Layer technologies (e.g., Bluetooth), and the communication between a sensor and device occurs over an application programming interface (API). A \textit{sample} is a single sensor reading containing data and metadata, such as time and the physiological data. During a recording session, ensuring for a consistent and uninterrupted data stream from the sensors is vital to obtaining persistent and meaningful data.  Once a recording session has terminated, a \textit{record} with metadata about the recording session is stored, alongside the samples.    

\noindent \textbf{Sharing}

\noindent Sharing is a mechanism to export and import records across applications. \textit{Exporting} consists of bundling one or more records with correlated samples into a transmittable format, and transferring the bundled records over a media (e.g., mail). \textit{Importing}, on the other hand, consists of locating the bundled records on the device, parsing the content and storing it on the device. [skrive mer?]

\noindent \textbf{Module}

\noindent A \textit{module} is an independent application that can be installed and launched in Nidra (hereafter: application), to provide extended functionality and data enrichment. A module does not necessarily interact with the application; however, it utilizes the data (e.g., records). For example, a module could be using the records to feed a machine learning algorithm to predict obstructive sleep apnea. Installing a module is achieved by locating the module-application on the device, and storing the reference in the application. Due to limitations in Android, the module-application cannot be executed within the application. Therefore, the module-application is a standalone Android application; furthermore, the development of the module-application is independent from the application. 

\noindent \textbf{Analytics}

\noindent Analytics is the visualization and interpretation of patterns in the records. The application facilitates the recording of physiological signals, which enables the detection and analysis of sleep-related illnesses. There are various analytical methods, ranging from graphs to advanced machine learning algorithms. Incorporating a simple time series plot can indirectly aid in the analysis. For instance, plotting a time series graph where the physiological signals are on the Y-axis and the time on X-axis, provides a graphical representation of the data that can be further analyzed within the application.

\noindent \textbf{Storage}

\noindent Storage is the objective of achieving persistent data; data remain available after application termination. To enable storage, we use a database for a collection of related data that is easily accessed, managed, and updated. The database should be able to store records, samples, modules, and biometrical data related to user (i.e., gender, age, height, and weight). Structuring a database that is reliable, efficient, and secure is a crucial part of achieving persistent storage. Android provides several options to enable storage on the device (e.g., internal storage and database).

\noindent \textbf{Presentation}

\noindent Presentation is the concept of exhibiting the functionality of the application to the user. A user interface (UI) is the part of the system that facilitates interaction between the user and the system. In Nidra, determining the layout and view of the application, color palette, interactions, and feedback on actions is part of the development of a user interface. 

\section{Seperation of Concerns}\label{design_soc}
Separation of concern is a paradigm that classifies an application into concerns at a conceptual and implementational level. It is beneficial for reducing complexity, improving understandability, and increasing reusability \cite{soc}. The concerns in this thesis are the individual tasks defined in the task analysis. Each concern is conceptualized with a graph of \textit{components}, the functionality of each component when combined constitutes a {structure}, that is derived based on research and development. This section proposes a solution for the structure of each concern, by analyzing and decomposing the tasks defined in task analysis into components.


\subsection{Recording}\label{soc:recording}
The structure of a recording is restrictive in terms of arranging the components due to the design of the CESAR project. That is, the recording structure is based on the interface (e.g., establishment and receiving data) facilitated for the sensor sources by the project. Besides this, the structure can be extended with components that monitor the data acquisition and the connectivity with the sensor and the device. In Figure \ref{fig:hta_recording}, a proposition of the structure for recording with components and their dependencies is illustrated, and the components are described as:

\begin{figure}
    \centering
    \includegraphics[scale=0.7]{images/Recording_Design.pdf}
    \caption{A design proposal for the structure of the recording concern.}
    \label{fig:hta_recording}
\end{figure}

\begin{itemize}
    \item[1.1] \verb|Sensor Discover:| Find all eligible sensors that can enable a recording.
    \item[1.1.1] \verb|Select Sensors:| Select a preferable sensor source for data acquisition. 
    \item[1.1.2] \verb|All Sensors:| Select all available sensors source for data acquisition.
    \item[1.2] \verb|Sensor Initialization:| Establish and initialize a connection with the selected sensor source. Occasionally a sensor might use some time to connect, or unforeseen occurrence is hindering the initialization of the sensor. Therefore, halting the sensor initalization or actively checking for sensor initalization is important. 
    \item[1.3] \verb|Sensor Connectivity Setup:| Establish a connection between the application and the sensor source using the protocols the sensor sources provides (e.g., BlueTooth). All data exchange will occur over the established interface. 
    \item[1.4] \verb|Connection State:| Is the state that determines whether the recording has started (ongoing) or stopped (finalized).
    \item[1.4.1] \verb|Start:| Notify the sensors to begin collecting data, and the application should display that a recording has started to the user. Also, start a timer to display time elapsed on the current recording. 
    \item[1.4.1.1] \verb|Monitor:| Is a mechanism to handle the connectivty state and the incoming samples. It is actively listening to the interface for new data from the sensors, and forwarding the data to the sample management component.  
    \item[1.4.1.1.1] \verb|Sample Management:| Handles a single sample from a sensor by parsing the content according to the payload of the data tuple from the sensor.
    \item[1.4.1.1.2] \verb|Connectivity State:| Makes sure that the connectivity between the sensor and the device is maintained. Further described in Section \ref{sssec:csc}.
    \item[1.4.2] \verb|Stop:| Stop the recording timer and proceed to display results.
    \item[1.4.2.1] \verb|Sensor Finalization:| Notify the sensor to stop sampling data, and close establishment.
    \item[1.4.2.2] \verb|Close Sensor Connectivity:| Close the interface connections between the application and the connected sensors. 
    \item[1.4.2.3] \verb|Stop Recording:| Once all connections are closed, add additional information to the recording (e.g., title, description, rating). In the end, the recording has concluded and it is stored on the mobile device.
\end{itemize}

\subsubsection{Connectivity State Component} \label{sssec:csc}
Connectivity state is a component that monitors for unexpected sensor disconnections or disruptions. Unexpected behavior can occur due to anomalies in the sensor, or the sensor being out of reach from the device for a brief moment.  A naive solution would be to ignore the connectivity state component, and assuming the sensors are connected to the device indefinitely. However, upon disconnections or disruptions, the recording would be missing samples, resulting the record has less meaningful data. This component enables the application to reconnect with the sensor based on a time interval, resulting in more accurate and meaningful record. The following design questions for this component are (1) should the connectivity state component, which implements a time interval that tries to reconnect with the sensor, be implemented in the sensor wrapper, or should it be in the proposed recording structure?; and (2) should the interval between sample arrival be a fixed time or a dynamical time? 


\begin{enumerate}
    \item To achieve a mechanism of reconnecting with the sensor on unexpected disconnects or disruptions, establishing a time interval that monitors for sample arrivals within a time frame (e.g., every 10 seconds) is required. Incorporating the time interval in the sensor wrapper reduces the complexity of Nidra. However, it introduces extra complexity to the sensor wrappers. A sensor wrapper has to distinguish actual disconnects from unexpected disconnects. Although, by extending the functionality of sensor wrapper by implementing a state that indicates whether a recording is undergoing or stopped solves the problem. All future sensor wrappers would then have to implement the proposed solution, resulting in a complicated and time-consuming sensor wrappers development. While implementing the proposed solution in the sensor wrappers is possible, extending the recording structure with the logic in Nidra would be more meaningful and time-saving. In our design, we implement the connectivity state in the structure for recording.
    \item A time interval triggers an event every specified time frame. A time frame can be in a fixed size (e.g., every 10 seconds) or a dynamical size (e.g., start with 10 seconds, then incrementally increase the frame by X seconds). Implementing a fixed time frame increases the stress on put on the sensor, whereas a dynamical time might miss samples if the time frame is significant. Depending on how critical the recording is, a suitable solution for the time frame should be configurable. Also, limiting the number of attempts made to reconnect should be considered, due to actively reconnecting to a sensor that presumably is dead or completely out of reach is unnecessary. Thus, stopping the recording once a limited number of attempts has been reached.  In our design, we implemented a dynamical time and limited the number of attempts to 10. 
\end{enumerate}

\subsection{Sharing} \label{sec:design_sharing}

Sharing is separated into two concerns: export and import. The scope of exporting in Nidra is to select desired records, format and bundle the records into a transmittable file, and sendting the bundle over a media (e.g., mail). The scope of importing is to locate the file on the device, parse the content based on the format, and store it on the device. In Figure \ref{fig:hta_sharing}, a proposition of the structure for sharing with components and their dependencies is illustrated, and the components are described as:


\begin{figure}
    \centering
    \includegraphics[scale=0.8]{images/Sharing_Design.pdf}
    \caption{A design proposal for the structure of the sharing concern.}
    \label{fig:hta_sharing}
\end{figure}

\begin{itemize}
    \item[2.1] \verb|Import|: Is a mechanism that locates a file, parses the data, and stores it on the device.
    \item[2.1.1] \verb|Locate File|: To enable this, the user has to download the file on the device. Then, locate the file on the device by using an interface to browse downloaded files. An interface can be developed; however, using the interface Android provides to locate downloaded files is a more straightforward solution.
    \item[2.1.2] \verb|Parse Formated Content|: Parse the content of the file accordingly to the data format discussed in \Cref{sec:dataformat}.
    \item[2.1.3] \verb|Insert Storage|: Retrieve the necessary data from the parsed file, to store on the device without overriding existing data.   
    \item[2.2] \verb|Export|: Is a mechanism that selects all or a specific record, transforms the record into a formatted file (see \Cref{sec:dataformat}), and transmits the file across application (e.g., sending records from the patient's installed application to the researcher/doctor's installed application with the use of an e-mail application). 
    \item[2.2.1.1] \verb|All Records|: Export all of the records on the device.
    \item[2.2.1.2] \verb|Select Record|: Pick one specific record to export. 
    \item[2.2.2] \verb|Format File|: When a preferred format for the records is selected, bundling the data into a formatted (see \Cref{sec:dataformat}) file for transmission. It is essential to identify the name of the file uniquely to prevent duplicates and overrides of data. For instance, identifying the name of the file with the device identification appended with the time of exporting. 
    \item[2.2.3] \verb|Distribute|: Send the file over a media, e.g., e-mail (see Section \ref{des:distro}).
\end{itemize}

\subsubsection{Distribute Component}\label{des:distro}
The distribute component uses the formatted file and transfer it to other instances of Nidra using a media (e.g., e-mail). There are two distinctive methods to perform this task which is efficient and practical: (1) establish a web-server with logic to handle users and sharing data with the desired recipient, and implementing an interface to retrieve the file within the application; and (2) using the interface provided by Android to share files across applications (e.g., e-mail). 

While the first option might be favorable in terms of practicality, this solution introduces additional concerns (e.g., the privacy matters of storing user data on a server) which is out of the scope for this project. For this reason, using the interface provided by Android is a reasonable solution. The user of the application can utilize the Android interface for sharing files over installed applications (e.g., the e-mail is a flexible media to transfer the file, and the user can specify the recipients accordingly).  



\subsection{Modules}\label{soc:modules}

Modules are independent applications that provide extended functionality or data enrichment to Nidra. The components for locating and launching a module is given by the Android design; however, the component for data exchange between a module and Nidra can be independently designed.  In Figure \ref{fig:hta_modules}, a proposition of the structure for modules with components and their dependencies is illustrated, and the components are described as:

\begin{itemize}
    \item[3.1.1] \verb|Install Modules|: Is the process of locating the application on the device, and storing the reference of the application package name in the storage.  
    \item[3.1.2] \verb|Locate Module|: Retrieve the list of stored modules, and display the installed modules to the user. 
    \item[3.2] \verb|Launch Module|: Get the application location stored in the module, and launch the application with the use of Android Intent. 
    \item[3.3] \verb|Data Exchange|: Enrich the module with data from the application (see Section \ref{des:des}).
\end{itemize}


\begin{figure}[!h]
    \centering
    \includegraphics[scale=0.8]{images/Modules_Design.pdf}
    \caption{A design proposal for the structure of the modules concern.}
    \label{fig:hta_modules}
\end{figure}

\subsubsection{Data Exchange Component}\label{des:des}

The data exchange component facilitates the transportation of data between Nidra and a module. As of now, the data is records and corresponding samples, which is formatted (see Section \ref{sec:dataformat}) accordingly. The two distinct methods to exchange data between a module and Nidra are (1) formatting all of the data and bundling it into the launch of the module, and (2) establishing a communication link for bi-directional requests between Nidra and the module. 

Android provides an interface to attach extra data on activity launch. The first solution is, therefore, convenient and efficient; all of the data is formatted and bundled into the launch. However, once Nidra has launched the module, there are no ways of transmitting new data besides relaunching the module. For this reason, the second option allows for continuous data flow by establishing a communication link with IPC between the applications. The data exchange between Nidra and modules can then be bidirectional; the module can request desired data any time, and Nidra can collect reports and results generated by the module. 

One could argue that new records are not obtained while managing and using a module. However, there might be future modules that do a real-time analysis of a recording; as such, require an interface for continuous data flow. For the simplicity of our design, we use the first option of bundling all of the data and sending it on launch.

\subsection{Analytics}\label{soc:analytics}

\begin{figure}
    \centering
    \includegraphics[scale=0.8]{images/Analytics_Design.pdf}
    \caption{A design proposal for the structure of the analytics concern.}
    \label{fig:hta_analytics}
\end{figure}

Analytics uses techniques and methods to gain valuable knowledge from data. Nidra provides a simple time-series plot for illustration of the data. However, other techniques can be in incorporated. In essence, the facilitation of modules in the application enables the development opportunities for advanced analytics of the data. In Figure \ref{fig:hta_analytics}, the structure of analytics is presented with components and their dependencies: 

\begin{itemize}
    \item[4.1.1] \verb|Select Record|: Select one of the records on the device.
    \item[4.1.2] \verb|All Records|: Select all of the records on the device.
    \item[4.2] \verb|Analytic Method|: Apply an analytical method on the data provided in Nidra (see Section \ref{des:amc}).
\end{itemize}

\subsubsection{Analytic Method Component} \label{des:amc}
The analytic method component uses the records on the device for representation or analysis. Graphical and non-graphical are two techniques for representing data. Graphical techniques visualize the data in a graph that enables analysis in various ways. A few graphical techniques are diagrams, charts, and time series. Non-graphical techniques, better known as statistical data tables, represent the data in tabular format. This provides a measurement of two or more values at a time \cite{datarepresentation}. More advanced techniques to analyze the data are to use machine learning. Machine learning is concerned with developing data-driven algorithms, which can learn from observations without explicit instructions. For example, using recurrent neural networks (e.g., RNN, LSTM) or regression models (e.g., ARIMA), can be used to predict the sleeping patterns \cite{machinelearning}.

In Nidra, a time series graph is used to represent the data of a record. The time series graph represents the respiration data on the Y-axis and the time on the X-axis. Essentially, the facilitation of modules in the application is designed to enable advanced techniques to predict, analyze, and interpret the data acquisition. Therefore, in Nidra, the analytic methods are limited; however, the modules enable developers to construct any method they desire. 

\subsection{Storage}\label{soc:storage}

\begin{figure}
    \centering
    \includegraphics[scale=0.8]{images/Storage_Design.pdf}
    \caption{A design proposal for the structure of the storage concern.}
    \label{fig:hta_storage}
\end{figure}

Storage is the objective of achieving persistent data; data remain available after application termination. The data is characterized into four data entities (i.e., record, sample, module, and user) that contain individual properties.  The components in the storage structure are constructed to be extensible and scalable in terms of future data, restructure of data, and removal of data.  In Figure \ref{fig:hta_storage}, a proposition of the structure for storage with components and their dependencies is illustrated, and the components are described as:

\begin{itemize}
    \item[5.1] \verb|Data Access Layer|: It exposes specific operations (such as insertion of a record) without revealing the database logic. This component is also known as a data access object (DAO), that provides an abstract interface to a database. The advantage of this interface is to have a single entry point for each database operation and to extend and modify the operation for future data easily. 
    \item[5.2] \verb|Database|: Is the storage of all the data (see Section \ref{des:dc}).
    \item[5.2.1] \verb|Record|: Is a table in the database, which contains fields appropriately to the record structure. A record contains meta-data about a recording (e.g., name, recording time, user). The design decision and an example of a recording record are illustrated in \Cref{ssec:record}.
    \item[5.2.2] \verb|Sample|: Is a table in the database, which contains fields appropriately to a single sample from a sensor. A sample contains data received from a sensor during a recording. The design decision and an example of a sample record are illustrated in \Cref{ssec:sample}.
    \item[5.2.3] \verb|Module|: Is a table in the database, which contains fields appropriately to the module structure. A module contains the name of the module-application and a reference to the application package. The design decision and an example of a module record are illustrated in \Cref{ssec:module}.
    \item[5.2.4] \verb|User|: Is an object stored on the device, which contains fields appropriately to the user structure. A user contains the patient's biometrical information (e.g., name, weight, height). The design decision and an example of a user are illustrated in \Cref{ssec:user}.
\end{itemize}

\subsubsection{Database Component}\label{des:dc}

\noindent Android provides several options to store data on the device; depending on space requirement, type of data that needs to be stored, and whether the data should be private or accessible to other applications. Two suitable options for storage are (1) internal file storage: storing files to the internal storage private to the application; and (2) database: Android provides full support for SQLite databases, and the database access is private to the application \cite{android_storage}. Based on the options, the design question for this component is should the data be stored in a flat file database on the internal file storage, or should it be stored in an SQLite database?

Flat files database encode a database model (e.g., table) as a collection of records with no structured relationship, into a plain text or binary file. For instance, each line of text holds on a record of data, and the fields are separable by delimiters (e.g., comma or tabs). Another possibility is to encode the data in a preferable data format, such as JSON or CSV files (see \Cref{sec:dataformat}). Flat file databases are easy to use and suited for small scale use; however, they provide no type of security, there is redundancy in the data, and integrity problems \cite{flatfilerdbms}. Locating a record is made possible by loading the file, and systematically iterating until the desired record is found. Similarly, updating a record and deleting a record. Consequently, the design of flat file databases is for simple and limited storage.

SQLite is a relational database management system, which is embedded and supported in Android. A relational database management system (RDBMS) provides data storage in fields and record, represented as columns (fields) and rows (records), in a table. The advantage is the ability to index records, relations between data stored in tables, and support querying of complex data with a query language (e.g., SQL). Also, RDBMS provides data integrity between transactions, improved security, and backup and recovery controls \cite{flatfilerdbms}. 

While a flat-file database is applicable to store small and unchangeable data, it is not suitable for scalable and invasive data change. The samples acquisition during recording makes it unreasonable to use a flat-file database. In the design of Nidra, SQLite is a preferable solution for storing sample entities. As a result, the record entity is be stored in an SQLite database, in order to associate it with the sample entity (with a foreign key). As for the module entity, it is more convenient to store and update a list of records inside a database; therefore, we use the SQLite database for the module entity. Finally, the user entity is stored in the flat file, that is because there only is one user in the application a time; hence, it is more convenient to use a flat-file (e.g., JSON) to store this type of information.

\subsection{Presentation}\label{soc:presentation}
\begin{figure}
    \centering
    \includegraphics[scale=0.7]{images/Presentation_Design.pdf}
    \caption{A design proposal for the structure of the presentation concern.}
    \label{fig:hta_presentation}
\end{figure}

Presentation facilitates the user interface of the application, in terms of visualizing the functionality of the application to the user. The user interface design is guided by the functionality (concerns discussed) in the application and iterative work on the topic. In Figure \ref{fig:hta_presentation}, a proposition of the structure for presentation with components and their dependencies is illustrated, and the components are described as:

\begin{itemize}
    \item[6.1] \verb|UI|: The user interface where interaction between users and the application occurs. 
    \item[6.1.1] \verb|Navigation|: The navigation is a menu with options to change the screen (e.g., feed, recording, and module). 
    \item[6.1.2] \verb|Layout|: Is based on the functionality that is exposed to the user. The layout consists of incorporating the color palette and the screens in the application (the subsequent components). 
    \item[6.1.3] \verb|Color Palette|: A color palette is a set of colors that persist throughout the application (see Section \ref{des:cpc}).
    \item[6.1.1.1] \verb|Settings|: Is s screen with user details, permissions, and credits, with options to modify permission and user details.
    \item[6.1.1.2] \verb|Feed|: Is a list of all records for the user, displayed with details specific to the record to make it distinguishable and easily recognizable. 
    \item[6.1.1.3] \verb|Analytics|: A interactive time-series graph for a single record. 
    \item[6.1.1.4] \verb|Recording|: The process of establishing a recording session, in addition to showing the results after a recording session has ended. 
    \item[6.1.1.5] \verb|Onboarding|: The initial screen displayed to the user, where the user can supply the application with their biometrical data. 
    \item[6.1.1.6] \verb|Modules|: A list of all installed modules, also an option to add more modules. 
    \item[6.1.4] \verb|Feedback & Interactions|: Each screen has different feedback and interaction, which should be handled appropriately. 
\end{itemize}

\subsubsection{Color Palette Component}\label{des:cpc}
The color palette is a component that decides the color scheme in the application. In the proposal of a color system in the design guidelines by Google \cite{colorsystem}, it is essential to pick colors that reflect the style of the application accordingly to: (1) \textit{primary colors}: the most frequently displayed color in the application; (2) \textit{secondary colors}: provides an accent and distinguish color in the application; and (3) \textit{surface, background, error, typography and iconography colors}: colors that reflect the primary and secondary color choices. 

Moreover, choosing colors that meet the purpose of the application is critical. Nidra is most likely to be used during the evening and the morning. According to Google \cite{darktheme}, a dark color theme reduces luminance emitted by the device screen, which reduces eye strain, while still meeting the minimum color contrast ratios, and conserving battery power. Therefore, in our design, we chose a dark color theme (more specifically, a nuance of dark-blue). 

\section{Data Structure}

\subsection{Data Formats} \label{sec:dataformat}

The data format is a part of the process of serialization, which enables data storage in a file, transmission over the Internet, and reconstruction in a different environment. Serialization is the process of converting the state of an object into a stream of bytes, which later can be deserialized by rebuilding the stream of bytes to the original object \cite{sumaray_comparison_2012}. There are several data serialization formats; however, JavaScript Object Notation (JSON) and eXtensible Markup Language (XML) are the two most common data serialization formats. This section discuss these formats, and in the end, we compare them and choose the format that meets the criteria of being compact, human-readable, and universal. 

\subsubsection{JSON}
JSON or JavaScript Object Notation is a light-weight and human-readable format that is commonly used for interchanging data on the web. The format is a text-based solution where the data structure is built on two structures: a collection of name-value pairs (known as objects) and ordered list of values (known as arrays). The JSON format is language-independent and the data structure universally recognized \cite{jsonorg, jsonvxml}. However, it is limited to a few predefined data types (i.e., string, number, boolean, object, array, and null), and extending the data type has to be done with the preliminary types. 

\begin{lstlisting}[language=json, caption={}, captionpos=b]
{
    "user": {
        "firstname": "Ola"
        "lastname": "Nordmann"
    }
}
\end{lstlisting}

\subsubsection{XML}
XML or eXtensible Markup Language is a simple and flexible format derived from Standard Generalized Markup Language (SGML), developed by the XML Working Group under the World Wide Web Consortium (W3C). An XML document consists of markups called tags, which are containers that describe and organize the enclosed data. The tag starts with \verb|<| and ends with \verb|>|; the content is placed between an opening tag and a closing tag. \cite{w3xml, jsonvxml} XML provides mechanisms to define custom data types, using existing data types as a starting point, making it extensible for future data. 

\begin{lstlisting}[language=json, caption={}, captionpos=b]
<user>
    <firstname>Ola</firstname>
    <lastname>Nordmann</lastname>
</user>
\end{lstlisting}

\subsubsection{Comparing}
With the study conducted by Saurabh and D’Souza \cite{jsonvxml}, we compare JSON and XML features and performance. There are apparent differences in the two data formats which affect the overall readability, extensibility, bandwidth performance, and ease of mapping. XML documents are easy to read, while JSON is obscure due to the parenthesis delimiters. XML allows for extended data types, while JSON is limited to a few data types. XML takes more bandwidth due to the metadata overhead, while JSON data is compact and use less amount of bandwidth.

Moreover, a few benchmarks were conducted to measure memory footprint and parsing runtime when serializing and deserializing JSON and XML data. From the conclusion,  in terms of memory footprint and parsing runtime, JSON performances better than XML, but at the cost of readability and flexibility. While these format structures are applicable for transmitting data, choosing a format that is compact, human-readable, and a standard format that is extensible and scalable for future data is essential. In our design, we use the JSON format for transmission of the data.

\begin{figure}
    \centering
    \includegraphics[scale=0.8]{images/DataEntries.pdf}
    \caption{The data model and relationship for the entities in the application: a record has zero to many samples, while a sample only can have one record. A record can have one user, while a user can have many records. A module has no relationship with the other entities.}
    \label{fig:dataentries}
\end{figure}

\subsection{Data Entities}\label{des:dataentity}
Data entities are objects (e.g., things, persons, or places) that the system models and stores information about. In Section \ref{soc:storage}, we introduced four data entities in the application (i.e., record, sample, module, and user). In Figure \ref{fig:dataentries}, the relations between the data entities are shown. The record entity and sample entity store information about the recording session\footnote{the ongoing process of collecting data from the sensor sources and storing the samples (data packets) in the database.} and are separated into two individual entities in order to reduce data redundancy and improve data integrity. The sample entity has a reference to its record entity so they can be associated with each other. The user entity store biometrical information related to the user (i.e., patients). Also, a record entity contains the state of the user's biometrical information at the time of the recording. In other words, the user's biometrical information can change over time (e.g.,  weight changes); therefore, capturing the exact biometrical information at the time of the recording is essential in the context of detecting sleeping disorders with relation to the biometrical information.  A module entity is independent of the other data entities and stores information about the name and the package name of the module-application. The package name is used to locate and launch the module-application. 

In the following sections, we present the properties of each data entity in Nidra, and an example of the data structure for each entity.  


\subsubsection{Record Entity} \label{ssec:record}

The table for the record entity in the database stores metadata (e.g., elapsed time of recording, number of samples, user's biometrical data) related to a recording session. In Table 4.1, we present the information (fields) the record entity contains, which can be described as: 
\begin{itemize}
    \item \verb|ID|: Unique identification of a record, also a primary key for the entry.
    \item \verb|Name|: A name of the record to easily recognize the recording.
    \item \verb|Description|: A summary of the recording session provided by the user. It can be used to briefly describe how the recording session felt (e.g., any abnormalities during the sleep).
    \item \verb|MonitorTime|: The recording session duration in milliseconds.
    \item \verb|Rating|: User defined rating on how the sleeping session felt, in a range between 0--5. 
    \item \verb|User|: User's biometrical information encoded into a JSON string format, in order to capture the state of the user at the time of recording. 
    \item \verb|CreatedAt|: Date of creation of the recording in milliseconds (since January 1, 1970, 00:00:00 GMT).
    \item \verb|UpdatedAt|: Date of update of the recording in milliseconds (since January 1, 1970, 00:00:00 GMT).
\end{itemize}

\begin{table}[!h]
\begin{center}
\scalebox{0.75}{
\begin{tabular}{ |c|c|c|c|c|c|c|c| } 
\hline
\textbf{id} & \textbf{name} & \textbf{description} & \textbf{monitorTime} & \textbf{rating} & \textbf{user} & \textbf{createdAt} & \textbf{updatedAt} \\
\hline
1 & Record \#1 & - & 5963088 & 2.5 & \{...\} & 1554406256000 & 1554406256000  \\ 
\hline
\end{tabular}}
\caption{Example entry in the record table.}
\end{center}
\end{table}





\subsubsection{Sample Entity} \label{ssec:sample}
The table for the sample entity contains a single sensor reading (data packet) sent from the sensor source. Sample entity are stored separated from a record entity; however, they are linked (foreign key) with their corresponding record entity. In Table 4.2, we present the information (fields) the sample entity contains, which can be described as: 

\begin{itemize}
    \item \verb|ID|: Unique identification of a sample, also a primary key for the entry.
    \item \verb|RecordID|: An identification to its corresponding record, also a foreign key. 
    \item \verb|ExplicitTS|: Timestamp of sample arrival based on the time in the sensor. 
    \item \verb|ImplicitTS|: Timestamp of sample arrival based on the time on the device. 
    \item \verb|Sample|: Sensor reading contains metadata and data according to Flow sensor.
\end{itemize}

\begin{table}[!h]
\begin{center}
\scalebox{0.60}{
\begin{tabular}{ |c|c|c|c|c| } 
\hline
\textbf{id} & \textbf{recordId} & \textbf{explicitTS} & \textbf{implicitTS} & \textbf{sample} \\
\hline
1 & 1 & 1554393086000 & 1554400286000 & Time=0ms, deltaT=100, data=1906,1891,1884,1881,1876,1718,1690 \\ 
\hline
\end{tabular}}
\caption{Example entry in the sample table.}
\end{center}
\end{table}



\subsubsection{Module Entity} \label{ssec:module}
The table for the module entity contains metadata of the module added by the user in the application. In Table 4.3, we present the information (fields) the module entity contains, which can be described as:
\begin{itemize}
    \item \verb|ID|: Unique identification of a module, also a primary key for the entry.
    \item \verb|Name|: The name of the module-application.
    \item \verb|PackageName|: The package name of the module-application, such that it can be launched from Nidra. 
\end{itemize}

\begin{table}[!h]
\begin{center}
\scalebox{0.8}{
\begin{tabular}{ |c|c|c| } 
\hline
\textbf{id} & \textbf{name} & \textbf{packageName} \\
\hline
1 & OSA Predicter & com.package.osa\_predicter \\ 
\hline
\end{tabular}}
\caption{Example entry in the module table.}
\end{center}
\end{table}


\subsubsection{User Entity} \label{ssec:user}
The user of the application is the patient, which provides biometrical data (e.g., weight, height, and age). The biometrical data is part of the application to enrich the record. The record captures the biometrical state of the user at the time of the recording. In Nidra, the user entity is not a part of the database, but stored as an object on the mobile device. The design decision of this choice is because the application is limited to one user at the time, and it makes it convenient to capture the state of the object of the given time of recording. It is possible to create a table in the database for the user entity, and for each time the user changes the biometrical data insert it is a separate entry in the database. The record could then have a reference to the latest user entry. However, that increases the complexity of the system, and as of now, the design of the application is to store the user's biometrical data into an object on the device. 

In the listing below, we present the structure of the user entity. In Nidra, the user entity is stored as an object and structured as a JSON format containing biometrical data related to the user: 
\begin{itemize}
    \item \verb|Name|: A string that contains the name of of the patient.
    \item \verb|Age|: An integer with the age of the patient.
    \item \verb|Gender|: A string that contains the gender of the patient.
    \item \verb|Weight|: An integer with the weight of the patient in kilograms.
    \item \verb|Height|: An integer with the height of the patient in centimeters.
\end{itemize}

\begin{figure}[!h]
\begin{lstlisting}[language=json, caption={}, captionpos=b]
{
    "user": {
        "name": "Ola Nordmann",
        "age": 50,
        "gender": "Male",
        "height": 180,
        "weight: 60
    }
}
\end{lstlisting}
\end{figure}



\subsection{Data Packets}\label{design:datapackets}
Data packets are parcels of data that Nidra receives from external applications (e.g., data streams dispatching module) or send to other application (e.g., by sharing the records on the device). This section presents the format of the records and corresponding samples that are sent across applications (e.g., the data is encoded into a file, and the user can select preferred media, such as e-mail). Also, this section presents the data packets forwarded by the data streams dispatching module upon data acquisition from the Flow sensor.

\subsubsection{Sharing}

In \Cref{sec:design_sharing}, a proposal for the structure of exporting and importing data is discussed. Two of the components (\verb|Parse Formatted Content| and \verb|Format File|) use JSON to either encode or decode the data. Listing 4.1 illustrates the content of the encoded data from the application to gain a broader understanding of how the data format looks. To summarize, the JSON string contains an object of record that encompasses the metadata of the recording session, as well as the state of biometrical data of the user. Besides the record object, there is an array of sample objects, where each sample object contains a single sensor reading from the Flow sensor. 

\begin{lstlisting}[language=json, caption={A JSON string object that contains the record which has metadata from the recording session (including the biometrical data of the user), and a list of samples (where each samples contains a single sensor reading).}, captionpos=b]
{
    "record":{  
        "id": 1,
        "name": "Record 1",
        "rating": 2.5,
        "description": "",
        "nrSamples": 6107,
        "monitorTime": 5963088,
        "createdAt": "Apr 4, 2019 9:30:56 PM",
        "updatedAt": "Apr 4, 2019 9:30:56 PM",
        "user": {  
            "age": 50,
            "createdAt": "---",
            "gender": "Male",
            "height": 180,
            "name": "Ola Nordmann",
            "weight": 60
        }
    },
    "samples": [  
        {  
            "explicitTS":"Apr 4, 2019 5:51:26 PM",
            "implicitTS":"Apr 4, 2019 7:51:26 PM",
            "recordId":1,
            "sample":"Time=0ms, deltaT=100, data=1906,1891,1884,1881,1876,1718,1690"
        },
        ...
    ]
}
\end{lstlisting}

\subsubsection{Single Sensor Reading}

Listing 4.2 presents a single sensor reading from the Flow sensor sent from the data streams dispatching module. The \verb|id| is assigned by the data streams dispatching module to identify the subscriber application with the corresponding sensor source. Moreover, the \verb|time| is assigned by the internal timer of the Flow sensor. Finally, the \verb|value| is the data sent from the Flow sensor, which contains time, delta, and data. However, we are most interested in extracting the data values from the reading. 

\begin{lstlisting}[language=json, caption={Contains a single sample received from the Flow sensor.}, captionpos=b]
{
    "id":"1-0",
    "value":"Time=2100ms, deltaT=100, data=1869,1873,1883,1864,1871,1870,1870",
    "time":"2019-07-29T18:20:58.997+0000"
}
\end{lstlisting}


