\section{Background and Motivation}

The medical scenario of focus for this thesis will be in field of sleep-related breathing disorders, which is characterized by abnormal respiration during sleep. Obstructive sleep apnea (OSA) is a group in the disorder, which in layman's terms is when the natural breathing cycle is partialy or complete affected during sleep. As a consequence, OSA decreases the quality of life and untreated OSA can lead to serious illnesses as cardiovascular diseases including diabetes, strokes, and atrial fibrillation [cite]. As of now, diagnosing OSA is performed in sleep laboratories with polysomnography\footnote{Test}. However, this method is both expensive and time-consuming proceedure that takes a toll on the patient and the laboratories. The patients are strapped in a contraption of sensors and ordered to sleep over a night, resulting in a uncomfortable and inconvenient situation for the patient. While, the laboratories has lacking capacity and resources to perform sufficient tests with the patients.

In the recent years, mobile phones has become significally advanced and powerful devices. What would require an entire room of processing power, has been compressed into an handheld and portable unit. As of now, mobile phones comes with powerful processors, sufficient amount of RAM, however with a limitation on the battery capacity. Mobile operates with a operating systems at its core, the operating system is a platform with a ecosystem that enables third-party developers to create and develop application that canbe used by end-users. Also, devices are delivered with sensors (e.g., microphones and accelrometeres), with capability of connecting to external sensors through wired or wireless communication channels. This enables external vendors to create sensors to detect events or changes in an environment and send the information to mobile devices. In the aspect of monitoring sleep-related breathing disorders, there are multiple sensor vendors which translate physiological signals (from the human body) into digital signals that can be processed by the mobile device. An example of such sensor vendor, are SweetZpot [cite]. (Flow Sensor). 

Creative use of mobile health informaton and sensor technologies (mHealth) has the potentioal to improve health research and reduce the cost of healthcare. Todays's health examinations are carried out by specialists with expensive medical refined equipment, which is statonary at hospitals and requires the patients physically presence.  Mobile technologies\footnote{mobile device and sensors that are intended to be worn, carried, or accessed by individuals} can overcome the hurdle of a patients presence, encourage behaviours to prevent or reduce health problems, and provide personalized/on-demand interventions [cite] in the hands of patients. As such, mobile technologies endorse healthcare solutions and opens up for possbilities within diagnosis and treatment of diseases outside the hospital. For example, external examinations can be performed (e.g., measurement of puls or hearth rate) by utilizing the mobil technologies to collect physiological data in order to detect, prevent or analyze health implications locally on the device. Thus, levering mobile devices and external sensor vendors introduces the potentioal to create mHealth application, which uses mobile technologies to evolve in the field of health research.   

A potentional mHealth application is the CESAR project, which aims to use low-cost sensor kits to monitor physiological signals (e.g., related to heart rate, brain activity, oxygen level in the blood) during sleep in order to improve diagnosis of ostructive sleep apnea (OSA). The CESAR project focuses on \textit{"development of new software solutions bridging state-of-the-art consumer electronic devices with appropriate sensors [...] to enable anyone to monitor physiological parameters that are relevant for OSA monitoring at home"} [cite]. This unfolds the potentioal to perform diagnosing of OSA from home with the aid of various sensors sources, opposed to the current method of sleeping over at a laboratory. Also, it minimizes the costs of examinations and stress enduced to the patients, as well as the hospitals. While this idea has its  potentioal, there are various mHealth applications that has an approach to this problem [...]. However, these applications are very specific and not future oriented. To overcome this, the CESAR project facilities for a service to incoorprate current and future sensor source, and managing the data acquistion from subscribed sensor sources. However, the project do not facilitate any kind of applicaton that exposes the functionality to record, share or analyze the data acquisition to the end-user (i.e., patients or reseachers/doctors). Therefore, with intention to expand on the project the motivation for this thesis is to utilizes the service for sensor management to create an application that exposes an user-inteface to collects physicalical data, share the recording across applications, and analyzing the data in a graph. Also, create an application that acts as a platform for other applications to leverage the recordings and perform advanced analytics or extended functionality in order to gather related applications into a centeralized application.

