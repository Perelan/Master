\section{Background and Motivation}

The medical scenario of focus for this thesis is in in the field of sleep-related breathing disorders, which is characterized by abnormal respiration during sleep. Obstructive sleep apnea (OSA) is a category in disorder, which in layman's terms is when the natural breathing cycle is partially or complete affected during sleep. As a consequence, OSA decreases the quality of life, and untreated OSA can lead to severe illnesses like cardiovascular diseases, including diabetes, strokes, and atrial fibrillation [cite]. As of now, the diagnosing of OSA is performed with polysomnography\footnote{test used to diagnose sleep disorders} in sleep laboratories. However, this method is both expensive and time-consuming procedure that takes a toll on the patient and the laboratories. The patient is strapped in a contraption of sensors and ordered to sleep overnight, resulting in an uncomfortable and inconvenient situation for the patient. While the laboratories have limited capacity and resources to perform sufficient tests with the patients.

In recent years, mobile phones have become significantly advanced and powerful devices. What would require an entire room of processing power, has been compressed into a handheld and portable unit. As of now, mobile phones come with powerful processors, a sufficient amount of RAM, and adequate amount of battery capacity. Mobile phones operate with an operating system at its core; the operating system facilitates for a platform that enables developers to create and develop applications that can be used by end-users. Moreover, mobile devices come with sensors (e.g., microphones and accelerometers), with the capability of connecting to external sensors through wired or wireless communication channels (e.g., BlueTooth). As such, external vendors can create sensors that aid in detection of events or changes in an environment, and send the information to mobile devices. In the aspect of monitoring sleep-related breathing disorders, there are various sensor vendors which translate physiological signals (from the human body) into digital signals which can be processed by the mobile device. An example of such a sensor vendor is SweetZpot Inc. [cite]. that provides an affordable respiratory effort belt, which captures the respiratory effort using strain-gauges. The sensor is initially designed to be used during physical activities (e.g., cycling, lifting, and rowing); however, it can be used for collecting breathing data over an extended period, as it is battery-powered and has BlueTooth support. 

Creative use of mobile devices and sensor technologies (mHealth) has the potential to improve health research and reduce the cost of healthcare. Todays's health examinations are carried out by specialists with expensive medical refined equipment, which is stationary at hospitals and requires the patients physically presence.  Mobile technologies\footnote{mobile device and sensors that are intended to be worn, carried or accessed by individuals} can overcome the hurdle of a patients presence, encourage behaviors to prevent or reduce health problems, and provide personalized/on-demand interventions [cite] in the hands of patients. As such, mobile technologies endorse healthcare solutions and opens up for possibilities within diagnosis and treatment of diseases outside the hospital. Levering mobile devices and external sensor vendors allows for innovation in mHealth application, which further allows for progress in the field of health research.   

A potential mHealth application is the CESAR project, which aims to use low-cost sensor kits to monitor physiological signals (e.g., related to heart rate, brain activity, the oxygen level in the blood) during sleep in order to improve diagnosis of obstructive sleep apnea (OSA). The CESAR project focuses on \textit{"development of new software solutions bridging state-of-the-art consumer electronic devices with appropriate sensors [...] to enable anyone to monitor physiological parameters that are relevant for OSA monitoring at home"} [cite]. This unfolds the potential to perform diagnosing of OSA from home with the aid of various sensors sources, as opposed to the current method of sleeping over at a laboratory. Also, it minimizes the costs of examinations and stress induced to the patients, as well as the hospitals. The CESAR project facilities for service to incorporate current and future sensor source, and managing the data acquisition from subscribed sensor sources. However, the project does not facilitate any application that exposes the functionality to record, share, or analyze the data acquisition to the end-user (i.e., patients or researchers/doctors). 

To expand on the project, the motivation for this thesis is to utilize the tool for sensor management in order to create an application that provides an user-interface for the patient. The application should be able collect breathing data, share the recording across applications, and analyzing the data in a graph. Also, create an application that acts as a platform for other applications to leverage the recordings and perform advanced analytics or extended functionality in order to gather related applications into a generalized application.

