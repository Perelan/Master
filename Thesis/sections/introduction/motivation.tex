\section{Background and Motivation}

The medical focus in this thesis is in the field of sleep-related breathing disorders, which is characterized by abnormal respiration during sleep. Obstructive sleep apnea (OSA) is a disorder, which in layman's terms is when the natural breathing cycle is partially or completely obstructed in repetitive episodes during sleep. As a consequence, OSA decreases the quality of life, and untreated OSA can lead to severe illnesses like cardiovascular diseases, including diabetes, strokes, and atrial fibrillation \cite{sleep_disorder}. The diagnosing of OSA is performed with polysomnography in sleep laboratories. However, this method is both an expensive and time-consuming procedure that takes a toll on the patient and the laboratories. The patient is strapped in a contraption of sensors and ordered to sleep overnight, while the laboratories have limited capacity and resources to perform sufficient tests with the patients.

In recent years, mobile phones have become significantly advanced and powerful devices. What would require an entire room of processing power, has been compressed into a handheld and portable device. As of now, mobile phones come with powerful processors, a sufficient amount of RAM, and an adequate amount of battery capacity. Mobile phones operate with an operating system at its core; the operating system facilitates a platform that enables developers to create and develop applications that can be used by end-users. Moreover, mobile devices come with sensors (e.g., microphones and accelerometers), with the capability of connecting to external sensors through wired or wireless communication channels (e.g., BlueTooth). As such, third-party vendors can create sensors that aid in the detection of events or changes in an environment, and send the information to mobile devices. With respect to monitoring sleep-related breathing disorders, there are various sensor vendors which translate physiological signals (from the human body) into digital signals which can be processed by the mobile device. An example of such a sensor vendor is SweetZpot Inc. \cite{flow}, which provides an affordable respiratory effort belt that captures the respiratory effort using strain-gauges, called Flow. The Flow sensor is initially designed to be used during physical activities (e.g., cycling, lifting, and rowing); however, it can be used for collecting breathing data over an extended period, as it is battery-powered and has BlueTooth support. 


Creative use of mobile devices and sensor technologies (mHealth) has the potential to improve health research and reduce the cost of healthcare. Mobile technologies\footnote{mobile device and sensors that are intended to be worn, carried or accessed by individuals} can overcome the hurdle of a patients presence, encourage behaviors to prevent or reduce health problems, and provide personalized/on-demand interventions \cite{kumar2013mobile}. A potential mHealth application is the CESAR project, which aims to use low-cost sensor kits to monitor physiological signals (e.g., related to heart rate, brain activity, the oxygen level in the blood) during sleep in order to provide early detection and analysis of obstructive sleep apnea (OSA). The CESAR project \cite{cesar} focuses on the development of new software solutions using state-of-the-art consumer electronic devices with appropriate sensors sources to enable anyone to monitor physiological parameters that are relevant for OSA monitoring at home. This unfolds the potential to provide early detection of OSA from home with the aid of various sensors sources. 

As of now, the CESAR project facilities for tools that manage the connectivity with sensor source, and forwarding of data packets from these sensor sources to subscribing applications. More specifically, the data streams dispatching tool \cite{daniel} manage applications (subscribers) that listen for data packets from sensor sources (publishers). Moreover, the data acquisition tool \cite{gjoby} facilitates for a sensor wrapper that provides a common interface for sensor sources with different Link Layer technology (e.g., USB, BlueTooth, and ANT+) and sensor-specific protocols (e.g., provider defined SDKs), and acts as a publisher that is connected with the data streams dispatching tool. 

The motivation for this thesis is to expand on the CESAR project by creating an application, called Nidra, for the patients to record, share, and analyze breathing data. Nidra should be able to collect breathing data over an extended period, share the recording data across applications using a media (e.g., e-mail), and to analyze the data. Additionally, Nidra should support the integration of other standalone applications (i.e., modules) which leverages the data from Nidra in order to enrich the data or extend the functionality of Nidra. For example, a module can be using the data from Nidra to feed a machine-learning algorithm that predicts sleep apnea. In the end, the data from the patients should aid the researchers/doctors in the diagnosis of obstructive sleep apnea. However, we can also apply Nidra in other fields of study (e.g., physical activities).