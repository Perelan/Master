\section{Problem Statement}

As indicated in the background and motivation section, we decided to look into opportunity of extending the the CESAR project even further. The market has new and affordable sensors that can aid with the data acquisition, which we seamlessly can integrate with the extensible data acquisition tool. The Flow sensor kit is a interesting sensor source due to its versatility and adaptability of collecting breathing data and connecting to devices with BlueTooth. As of now, the applications that support Flow sensor kit, are not user-friendly or designed to collect data over an extend period. Therefore, we look into designing and implementing an Android application that is used to record, share, and analyze breathing data over an extended period (e.g., during sleep). Also, we want to facilitate for an extensible application such that future developers can extend functionality of the application. 

In the end, we can hopefully strengthen the detection of abnormal sleeping patterns, and decreasing the risk of the symptoms they may endure. Also, we want to create a generalized application that can be used in other fields of studies (e.g., physical activites). As the scope of the this thesis, we will be focusing on the completion of three main goals:

\begin{description}
    \item[Goal 1] Integrate and support for Flow sensor kit with the \textit{extensible data acqusition tool}
    \item[Goal 2] Research and develop a user-friendly application which facilitates collection physilogical data through the extensible data acquisition tool.
    \item[Goal 3] Create a "platform" solution for developers to create modules. 
\end{description}

As part of the goals of this thesis, we also define three system requirements to keep in mind while designing and implementing the application. The three system requirements are the following: 

\begin{description}
    \item[Requirement 1] The application should provide an interface for the patient to 1) record physiological signals (e.g., during sleep); 2) present the results; and 3) share the results.
    \item[Requirement 2] The application should ensure a seamless and continuous data stream, uninterrupted from sensor disconnections and human disruptions.
    \item[Requirement 3] The application should provide an interface for the developers to create modules to enrich the data from records or extend the functionality of the application.
\end{description}

%With the goals in place, we started on designing and implementing an application that suffice the system requirements. The application would then help us on answering the 
