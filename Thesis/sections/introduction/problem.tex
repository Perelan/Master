\section{Problem Statement}

As indicated in the background and motivation section, we decided to look into opportunity of extending the the CESAR project even further. The market has new and affordable sensors that can aid with the data acquisition, which we seamlessly can integrate with the extensible data acquisition tool. In the end, we can hopefully strengthen the detection of abnormal sleeping patterns, and decreasing the risk of the symptoms they may endure. As the scope of the this thesis, we will be focusing on the completion of three main objectives:

\begin{description}
    \item[Objective 1] Integrate and support for Flow sensor kit with the \textit{extensible data acqusition tool}
    \item[Objective 2] Research and develop a user-friendly application which facilitates collection physilogical data through the extensible data acquisition tool.
    \item[Objective 3] Create a "platform" solution for developers to create modules. 
\end{description}

As part of the objectives of this thesis, which constitute the overall goal of the project, we also define three system requirements to keep in mind while designing and implementing the application. The three system requirements are the following: 

\begin{description}
    \item[Requirement 1] The application should provide an interface for the patient to 1) record physiological signals (e.g., during sleep); 2) present the results; and 3) share the results.
    \item[Requirement 2] The application should ensure a seamless and continuous data stream, uninterrupted from sensor disconnections and human disruptions.
    \item[Requirement 3] The application should provide an interface for the developers to create modules to enrich the data from records or extend the functionality of the application.
\end{description}
