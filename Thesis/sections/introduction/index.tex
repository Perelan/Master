\chapter{Introduction}\label{introduction}

\section{Background and Motivation}

The medical scenario of focus for this thesis is in in the field of sleep-related breathing disorders, which is characterized by abnormal respiration during sleep. Obstructive sleep apnea (OSA) is a category in disorder, which in layman's terms is when the natural breathing cycle is partially or complete affected during sleep. As a consequence, OSA decreases the quality of life, and untreated OSA can lead to severe illnesses like cardiovascular diseases, including diabetes, strokes, and atrial fibrillation [cite]. As of now, the diagnosing of OSA is performed with polysomnography\footnote{test used to diagnose sleep disorders} in sleep laboratories. However, this method is both expensive and time-consuming procedure that takes a toll on the patient and the laboratories. The patient is strapped in a contraption of sensors and ordered to sleep overnight, resulting in an uncomfortable and inconvenient situation for the patient. While the laboratories have limited capacity and resources to perform sufficient tests with the patients.

In recent years, mobile phones have become significantly advanced and powerful devices. What would require an entire room of processing power, has been compressed into a handheld and portable unit. As of now, mobile phones come with powerful processors, a sufficient amount of RAM, and adequate amount of battery capacity. Mobile phones operate with an operating system at its core; the operating system facilitates for a platform that enables developers to create and develop applications that can be used by end-users. Moreover, mobile devices come with sensors (e.g., microphones and accelerometers), with the capability of connecting to external sensors through wired or wireless communication channels (e.g., BlueTooth). As such, external vendors can create sensors that aid in detection of events or changes in an environment, and send the information to mobile devices. In the aspect of monitoring sleep-related breathing disorders, there are various sensor vendors which translate physiological signals (from the human body) into digital signals which can be processed by the mobile device. An example of such a sensor vendor is SweetZpot Inc. [cite]. that provides an affordable respiratory effort belt, which captures the respiratory effort using strain-gauges. The sensor is initially designed to be used during physical activities (e.g., cycling, lifting, and rowing); however, it can be used for collecting breathing data over an extended period, as it is battery-powered and has BlueTooth support.

Creative use of mobile devices and sensor technologies (mHealth) has the potential to improve health research and reduce the cost of healthcare. Todays's health examinations are carried out by specialists with expensive medical refined equipment, which is stationary at hospitals and requires the patients physically presence.  Mobile technologies\footnote{mobile device and sensors that are intended to be worn, carried or accessed by individuals} can overcome the hurdle of a patients presence, encourage behaviors to prevent or reduce health problems, and provide personalized/on-demand interventions [cite] in the hands of patients. As such, mobile technologies endorse healthcare solutions and opens up for possibilities within diagnosis and treatment of diseases outside the hospital. Levering mobile devices and external sensor vendors allows for innovation in mHealth application, which further allows for progress in the field of health research.   

A potential mHealth application is the CESAR project, which aims to use low-cost sensor kits to monitor physiological signals (e.g., related to heart rate, brain activity, the oxygen level in the blood) during sleep in order to improve diagnosis of obstructive sleep apnea (OSA). The CESAR project focuses on \textit{"development of new software solutions bridging state-of-the-art consumer electronic devices with appropriate sensors [...] to enable anyone to monitor physiological parameters that are relevant for OSA monitoring at home"} [cite]. This unfolds the potential to perform diagnosing of OSA from home with the aid of various sensors sources, as opposed to the current method of sleeping over at a laboratory. Also, it minimizes the costs of examinations and stress induced to the patients, as well as the hospitals. The CESAR project facilities for service to incorporate current and future sensor source, and managing the data acquisition from subscribed sensor sources. However, the project does not facilitate any application that exposes the functionality to record, share, or analyze the data acquisition to the end-user (i.e., patients or researchers/doctors). 

To expand on the project, the motivation for this thesis is to utilize the tool for sensor management in order to create an application that provides an user-interface for the patient. The application should be able collect breathing data, share the recording across applications, and analyzing the data in a graph. Also, create an application that acts as a platform for other applications to leverage the recordings and perform advanced analytics or extended functionality in order to gather related applications into a generalized application.


\section{Problem Statement}

The market has new and affordable sensors that can aid with the data acquisition, which we seamlessly can integrate with the extensible data streams dispatching tool. The Flow sensor is an interesting sensor source due to its versatility and adaptability of collecting breathing data and connecting to devices with the BlueTooth protocol. 

As for the state of this thesis: (i) applications which supports the Flow sensor has not been designed for end-users, in essence, they provide no user-friendly interface that allows for sharing of the data. In order to extract the data from these applications, the mobile device has to be connected with a PC over USB for data transfer; (ii) there is no sensor wrappers that support the Flow sensor with the data streams dispatching tool in the CESAR project; and (iii) the data streams dispatching tool is not ready to be used by the end-users, because the project facilitates no user-friendly interface for users to record the data from the supported sensor sources.

As such, we look into designing and implementing an Android application (Nidra) that record, share, and analyze breathing data over an extended period (e.g., during sleep) by using the Flow sensor. Additionally, we want to facilitate an extensible application such that future developers can extend functionality or enrich the data in Nidra. In the end, we can hopefully strengthen the analysis of abnormal sleeping patterns to decreasing the risk of the symptoms they may come as a consequence. Also, as a bi-product, the application can be used in other fields of studies (e.g., physical activities). As the scope of this thesis, we focus on the completion of three main goals:

\begin{description}
    \item[Goal 1] Integrate the support for Flow sensor by creating a sensor wrapper that connects with the extensible data streams dispatching tool.
    \item[Goal 2] Research and develop a user-friendly application which facilitates collection of breathing data with the Flow sensor, sharing of the data, and analysis of the data with the use of the extensible data streams dispatching tool.
    \item[Goal 3] Create an extensible solution such that the developers can create standalone applications that integrate with Nidra. 
\end{description}

As part of the goals of this thesis, we also define three system requirements to keep in mind while designing and implementing the application. The three system requirements are the following: 

\begin{description}
    \item[Requirement 1] The application must provide an interface for the patient to (i) record physiological signals (e.g., breathing data); (ii) present the results; and (iii) share the results.
    \item[Requirement 2] The application must ensure that upon sensor disconnections, the connectivity is reinstated to minimize the data loss and its effects on the data analysis.
    \item[Requirement 3] The application must provide an interface for the developers to create modules that integrate with the application.
\end{description}



\section{Limitations}
Based on the objectives and requirements defined in the previous section, the scope of this thesis is to design and implement an application capable of recording physiological obtained by the Flow sensor kit over an extended period. 

We limited the scope of testing for existing sensor source development, and the support for future sensor sources. Further, with the Flow sensor kit provided under development, we restricted the design to collect respiration (breathing) data (opposed to hearthrate).

Additionally, the implementation is Android specific as the previous work performed on the project is designed soley for Android applications. 

\section{Research Methods}

\section{Contributions}
Over the course of this thesis, we design and implemented an application for collecting, sharing and analyzing breathing data and a platform for modules to enrich the applications, called Nidra. The application is focused on creating a generalized application which manges sensor data with focus on sleep apnea / collecting data during sleep over an extended period, which purpose was to enable analyzing and sharing of the data with researchers/doctors.  

Through the work produced in this thesis, 


\section{Thesis Outline}
The thesis is divided into three parts, which the following list presents a general overview of:

\begin{itemize}
    \item Part 1: \textbf{Introduction \& Background}
    \begin{description}[font=\normalfont\itshape]
        \item[Chapther 2: Background] presents the background material necessary for understanding the fundamentals in this thesis. It starts by introducing the CESAR project and the tools provided for data acquisition. Then, an overview of the Android operating system architecture and components are 
        \item[Chapther 3: Related Work] presents the related work focusing on creating a mobile applicaton to collect physiological data in order to diagnose sleep apnea. Finally, we discuss why our solution is an improvement to the related work.
    \end{description}

    \item Part 2: \textbf{Design \& Implementation}
    \begin{description}[font=\normalfont\itshape]
        \item[Chapther 4: Analysis and High-Level Design] encompasses the functional requirements, the design purposal, and the structure of the data in the application.
        \item[Chapther 5: Implementation] realizes the design purposal in Android, with the use of the previous work.
    \end{description}

    \item Part 3: \textbf{Evaluation \& Conclusion}
    
    \begin{description}[font=\normalfont\itshape]
        \item[Chapther 6: Evaluation] conducts various experiements in order to tests if the system requirements is fullfilling.
        \item[Chapther 7: Conclusion] discuss the objectives and the overall goal of the thesis, with suggestions to future work.
    \end{description}

\end{itemize}

