\chapter{Introduction}\label{introduction}

\section{Background and Motivation}

The medical scenario of focus for this thesis is in in the field of sleep-related breathing disorders, which is characterized by abnormal respiration during sleep. Obstructive sleep apnea (OSA) is a category in disorder, which in layman's terms is when the natural breathing cycle is partially or complete affected during sleep. As a consequence, OSA decreases the quality of life, and untreated OSA can lead to severe illnesses like cardiovascular diseases, including diabetes, strokes, and atrial fibrillation [cite]. As of now, the diagnosing of OSA is performed with polysomnography\footnote{test used to diagnose sleep disorders} in sleep laboratories. However, this method is both expensive and time-consuming procedure that takes a toll on the patient and the laboratories. The patient is strapped in a contraption of sensors and ordered to sleep overnight, resulting in an uncomfortable and inconvenient situation for the patient. While the laboratories have limited capacity and resources to perform sufficient tests with the patients.

In recent years, mobile phones have become significantly advanced and powerful devices. What would require an entire room of processing power, has been compressed into a handheld and portable unit. As of now, mobile phones come with powerful processors, a sufficient amount of RAM, and adequate amount of battery capacity. Mobile phones operate with an operating system at its core; the operating system facilitates for a platform that enables developers to create and develop applications that can be used by end-users. Moreover, mobile devices come with sensors (e.g., microphones and accelerometers), with the capability of connecting to external sensors through wired or wireless communication channels (e.g., BlueTooth). As such, external vendors can create sensors that aid in detection of events or changes in an environment, and send the information to mobile devices. In the aspect of monitoring sleep-related breathing disorders, there are various sensor vendors which translate physiological signals (from the human body) into digital signals which can be processed by the mobile device. An example of such a sensor vendor is SweetZpot Inc. [cite]. that provides an affordable respiratory effort belt, which captures the respiratory effort using strain-gauges. The sensor is initially designed to be used during physical activities (e.g., cycling, lifting, and rowing); however, it can be used for collecting breathing data over an extended period, as it is battery-powered and has BlueTooth support.

Creative use of mobile devices and sensor technologies (mHealth) has the potential to improve health research and reduce the cost of healthcare. Todays's health examinations are carried out by specialists with expensive medical refined equipment, which is stationary at hospitals and requires the patients physically presence.  Mobile technologies\footnote{mobile device and sensors that are intended to be worn, carried or accessed by individuals} can overcome the hurdle of a patients presence, encourage behaviors to prevent or reduce health problems, and provide personalized/on-demand interventions [cite] in the hands of patients. As such, mobile technologies endorse healthcare solutions and opens up for possibilities within diagnosis and treatment of diseases outside the hospital. Levering mobile devices and external sensor vendors allows for innovation in mHealth application, which further allows for progress in the field of health research.   

A potential mHealth application is the CESAR project, which aims to use low-cost sensor kits to monitor physiological signals (e.g., related to heart rate, brain activity, the oxygen level in the blood) during sleep in order to improve diagnosis of obstructive sleep apnea (OSA). The CESAR project focuses on \textit{"development of new software solutions bridging state-of-the-art consumer electronic devices with appropriate sensors [...] to enable anyone to monitor physiological parameters that are relevant for OSA monitoring at home"} [cite]. This unfolds the potential to perform diagnosing of OSA from home with the aid of various sensors sources, as opposed to the current method of sleeping over at a laboratory. Also, it minimizes the costs of examinations and stress induced to the patients, as well as the hospitals. The CESAR project facilities for service to incorporate current and future sensor source, and managing the data acquisition from subscribed sensor sources. However, the project does not facilitate any application that exposes the functionality to record, share, or analyze the data acquisition to the end-user (i.e., patients or researchers/doctors). 

To expand on the project, the motivation for this thesis is to utilize the tool for sensor management in order to create an application that provides an user-interface for the patient. The application should be able collect breathing data, share the recording across applications, and analyzing the data in a graph. Also, create an application that acts as a platform for other applications to leverage the recordings and perform advanced analytics or extended functionality in order to gather related applications into a generalized application.


\section{Problem Statement}

The market has new and affordable sensors that can aid with the data acquisition, which we seamlessly can integrate with the extensible data streams dispatching tool. The Flow sensor is an interesting sensor source due to its versatility and adaptability of collecting breathing data and connecting to devices with the BlueTooth protocol. 

As for the state of this thesis: (i) applications which supports the Flow sensor has not been designed for end-users, in essence, they provide no user-friendly interface that allows for sharing of the data. In order to extract the data from these applications, the mobile device has to be connected with a PC over USB for data transfer; (ii) there is no sensor wrappers that support the Flow sensor with the data streams dispatching tool in the CESAR project; and (iii) the data streams dispatching tool is not ready to be used by the end-users, because the project facilitates no user-friendly interface for users to record the data from the supported sensor sources.

As such, we look into designing and implementing an Android application (Nidra) that record, share, and analyze breathing data over an extended period (e.g., during sleep) by using the Flow sensor. Additionally, we want to facilitate an extensible application such that future developers can extend functionality or enrich the data in Nidra. In the end, we can hopefully strengthen the analysis of abnormal sleeping patterns to decreasing the risk of the symptoms they may come as a consequence. Also, as a bi-product, the application can be used in other fields of studies (e.g., physical activities). As the scope of this thesis, we focus on the completion of three main goals:

\begin{description}
    \item[Goal 1] Integrate the support for Flow sensor by creating a sensor wrapper that connects with the extensible data streams dispatching tool.
    \item[Goal 2] Research and develop a user-friendly application which facilitates collection of breathing data with the Flow sensor, sharing of the data, and analysis of the data with the use of the extensible data streams dispatching tool.
    \item[Goal 3] Create an extensible solution such that the developers can create standalone applications that integrate with Nidra. 
\end{description}

As part of the goals of this thesis, we also define three system requirements to keep in mind while designing and implementing the application. The three system requirements are the following: 

\begin{description}
    \item[Requirement 1] The application must provide an interface for the patient to (i) record physiological signals (e.g., breathing data); (ii) present the results; and (iii) share the results.
    \item[Requirement 2] The application must ensure that upon sensor disconnections, the connectivity is reinstated to minimize the data loss and its effects on the data analysis.
    \item[Requirement 3] The application must provide an interface for the developers to create modules that integrate with the application.
\end{description}



\section{Limitations}
Based on the goals and requirements stated in previous section, the scope of this thesis is to design and implement an application capable of recording breathing data obtained by the Flow sensor over an extended period. 

We limite the scope to integrating the support for the Flow sensor in Nidra, and exluded to test for already integrated sensor sources (e.g., Bitalino). Further, with the Flow sensor provided under development, we restricted the design to collect respiration (breathing) data (opposed to hearth-rate or other physiological data).

The application is designed to collect breathing data; we do not perform any analysis to predict or detect sleeping disorders based on the data. However, we facilitate for future developers to utilize the data provided by Nidra to perform such task.

Finally, the implementation is Android specific as the previous work performed on the project is designed soley for Android applications. 

\section{Research Method}
The work in this thesis is classified as \textit{computing research} with a principle approach of an \textit{engineering method} as described in \cite{Glass_1995}. The engineering method (evolutionary paradigm) is to: (i) observe existing methods, (ii) propose a better solution, (iii)  build or develop artifacts\footnote{human-manufactured objects produced during the development}, and (iv) measure and analyze until no further improvements are possible. The report identifies patterns amongst various principle approaches and categorizes the the patterns into phases: \textit{(i) informational phase}, \textit{(ii) propositional phase}, \textit{(iii) analytical phase}, and \textit{(iv) evaluation phase}. Below, we give a brief description of each phase and discuss how our work fits into each of them. 

\subsection{Informational Phase}
The informational phase according to the report is to \textit{"gather or aggregate information via reflection, literature survey, people/organization survey, or poll"}

In this thesis, we survey previous related work in the field of detecting, analyzing, and diagnosing sleep related-breathing disorders on a mobile device. Based on this, we derive that the application created in this thesis has resembelance to previous related work; however, the related-work operates different measure or instrument for solutions (e.g., using microphone and accelerometers to provide early-detection of sleep apnea). As such, we focused on creating an application that allow future developers to create modules on top of our solution (as illustrated in Figure \ref{fig:nidra_modules}). By allowing this, future developers can expedite the innovation in the research and study of sleep-related breathing disorders, as well as allowing the patients to operate with one application. 

\begin{figure}
    \centering
    \includegraphics[scale=0.8]{images/Nidramodules.pdf}
    \caption{Nidra: operating with multiple modules that extends the functionality or provide data enrichiment, and integrating support for multiple sensor source (with the use of the data streams dispatching tool).}
    \label{fig:nidra_modules}
\end{figure}

\subsection{Propositional Phase}
The propositional phase according to the report is to \textit{"propose and/or formulate a hypothesis, method or algorithm, model, theory, or solution"}

The solution proposition in this thesis is to create an application used to record, share, and analyze breathing data collected over an extended period. We want to extend the CESAR project by providing an user-interface to the patient to perform these tasks while using the tools that the project facilitates. Mainly, we want to use the data streams dispatching tool in order to manage current and future sensor sources. In which, we proceed to add support for the Flow sensor. In the end, we wish to facilitate an application that is used by the patients to record their breathing data during sleep and to share the data between researchers/doctors. As such, we aid in analyzing the data to detect sleep-related breathing disorders (e.g., obstructive sleep apnea) from home. 

\subsection{Analytical Phase}
The analytical phase according to the report is to \textit{"analyze and explore a proposition, leading to a demonstration and/or formulation of a principle or theory"}

With the proposition phase of this thesis, we analyze the tasks of the application. We seperate the tasks into concerns (i.e., recording, sharing, analyzing, modules, storage, and presentation) where we propose a structure that encompasses components with functionality and design choices. With each concern combined, we drive towards the fullfing the goals of this thesis. As a demonstration, we realize the design choices by implementing them as an Android application, called Nidra. 

\subsection{Evaluation Phase}
The evaluation phase according to the report is to \textit{"evaluate a proposition or analytic finding by means of experimentation (controlled) or observation (uncontrolled, such as a case study or protocol analysis), perhaps leading to a substantiated model, principle, or theory"}

Based on the requirements and goal of this thesis, we evaluate the application by conducting experiments. Some of the experiments include participants that perform various tasks, such that we can observe the outcome of the tasks on participants without prior knowledge or experience of the application. In the end, we evaluate and conclude whether the goals and requirements of this thesis are sufficient.

\section{Thesis Outline}
The thesis is divided into three parts, which the following list presents a general overview of:

\begin{itemize}
    \item Part 1: \textbf{Introduction \& Background}
    \begin{description}[font=\normalfont\itshape]
        \item[Chapther 2: Background] presents the background material necessary for understanding the fundamentals in this thesis. It starts by introducing the CESAR project and the tools provided for data acquisition, as well as a description of the Flow sensor. Finally, an overview of the Android operating with the information required to understand the structure of the application (Nidra).
        \item[Chapther 3: Related Work] presents the related work focusing on creating a mobile applicaton to collect physiological data in order to diagnose sleep apnea, and with a brief
        discussion on how we contribibute with novelity and improvements from the related work.
    \end{description}

    \item Part 2: \textbf{Design \& Implementation}
    \begin{description}[font=\normalfont\itshape]
        \item[Chapther 4: Analysis and High-Level Design] presents the functional requirements of the application, the tasks derived based on the requirements and goals of the thesis, and the tasks separated into concerns, where we propose a structure of implementation which encompasses component with functionality and design choices. In the end, we discuss the data structure, namely the data entities (i.e., record, sample, module, and user), the data format (JSON versus XML), and the structure of the data packets sent from the sensor sources as well as the data packets sent on sharing. 
        \item[Chapther 5: Implementation] presents the application components of the project (i.e., Nidra, data streams dispatching module, and the Flow sensor wrapper) with the interface of IPC connectivity. Morever, it presents the steps and flow taken in order to implement the tasks based on the design choices by separting the actions and showing how the components in the application interacts.
    \end{description}

    \item Part 3: \textbf{Evaluation \& Conclusion}
    
    \begin{description}[font=\normalfont\itshape]
        \item[Chapther 6: Evaluation] presents four experiements conducted in order to evalute the system requirements of the application.  Each experiement have a short description followed up by results, a discussion on improvements and findings, and conclusion of the experiement. 
        \item[Chapther 7: Conclusion] presents the summary of the thesis, followed up with contributions that answer the goals defined in the problem statement.
    \end{description}

\end{itemize}

