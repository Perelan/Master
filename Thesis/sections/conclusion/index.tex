\chapter{Conclusion}

\section{Summary}
In this thesis, we designed an application for recording, sharing, and analyzing breathing data with the Flow sensor kit. The motivation of this thesis was to extend the CESAR project by providing an interface for patients to collect data during sleep. The data would aid researchers/doctors to analyze, diagnose, or examine the patient for sleep-related breathing disorders (e.g., Obstructive Sleep Apnea).  

In the design of this thesis, we identified the tasks to meet the system requirements of this application. Based on the design choices, we implemented an Android application called Nidra. The application leveraged the tools provided by the previous work performed on the CESAR project, in order to manage current sensor sources, as well as future sensor sources. We implemented a sensor wrapper for the Flow sensor kit, which connected with BlueTooth LE for data acquisition. During the recording, the samples from the sensor sources were unpacked and stored in our SQLite database. Also, we ensured for continuous data stream---uninterrupted from sensor disconnections and human disruptions---by reconnecting with the sensor source. At the end of the recording, metadata about the recording---including the state of the user's biometrical data---was stored in the database. Previous records were presented to the users, where each recording had the functionality to view the analytics or share the record. The sharing functionality allowed for transmitting records across applications, such that researchers/doctors can view and analyze the patient's record. The analytic part of Nidra constituted of a time-series, which enables the users of the application to interact with the data from the recording.  Conclusively, while Nidra's design is for recording sleep with various sensor sources, the application can be applied to other fields (e.g., recording respiration during a workout).  

In the field of detecting and diagnosing sleep-related breathing disorders, we could see that related work collects physiological data on various method to predict or examine sleeping disorders (e.g., Obstructive Sleep Apnea). Therefore, by creating an extensible application which can centralize the application of various techniques and methods in this field, they would benefit by creating an extention to our application in order to save time and effort in creating a base application of their project. As such, Nidra provides an interface for future developers and researchers to create modules which enrich the functionality of Nidra or provide data enrichment to the patient's recordings. The modules are independent Android applications, which is installed alongside Nidra and utilizes the provided records by Nidra.  For example, a module could be to use the data on the patient's device to feed a machine learning algorithm that predicts or diagnose the sleeping disorder.  

Finally, we performed experiments on the application to determine whether the system requirements were fulfilling. The experiments ranged from collecting respiration data for analysis of musical absorption, testing the application over an extended period, creating a simple module-application, including testing the application on real users. The results and observation made during the testing, allowed us to gain a broader understanding of the application, as well as improvements that can be made--discussed in future work. 

\section{Contribution}
In the problem statement of this thesis, we defined three research goals. In this Section, we restate the goals in together with how our work solves the goals. 

\begin{description}
    \item[Goal 1] \textit{Integrate and support for Flow sensor kit with the extensible data acqusition tool}

    In the implementation chapther of this thesis, we created a sensor wrapper which were integrated into the Data Stream Dispatching Module. The Flow sensor kit uses BlueTooth LE as an communication channel, thus, we used the APIs provided by Android to connect with the sensor. We extracted the data from the sensor, noteworthy the respiration data, as well as the battery level, mac address and firmware number. 

    \item[Goal 2] \textit{Research and develop a user-friendly application which facilitates collection physilogical data through the extensible data acquisition tool.}


    Through the course of this thesis, we have discussed the design of an application which facilitates the collection of respiration data, with the focus on the environment of use. The application is of most likely to be used during the night and early morning. Thus, selecting a color pallette that is not eye straining were picked. Also, we focused on limiting the number of actions that a user can perform on each screen. As previously mentioned, the application allows for collecting of data over an extended period of time with sensor sources that are integrated with the tools provided. 


    \item[Goal 3] \textit{Create a "platform" solution for developers to create modules.}

    As the field of collecting data 
\end{description}

\section{Future Work}
With the exploration performed in design and experiments, we can see that the application can be enhanced or improved. While the application fulfilles the goals defined in the problem statement, there are still some improvement that can be made. Below, we present the a short description of future work alongside with our proposition of improvement. 

\begin{description}
    \item[Improve the user-friendliess of Nidra:] In retrospect to the experiment, we could see that the participents had trouble with finding the record button, as well as the indication of whether a recording had started. A proposition is to enlargen til record button, to make it more visible to the users. For the recording, perhaps have more informative description of the various states (e.g., connecting, recording, or disconnected). 
    \item[Add support for other physilogical data] The main focus in this thesis, was to gather breathing data during sleep. However, the possibility to extending Nidra to support other physilogical data is possible (e.g., hearth rate). The extensible approach to application, allows future developers to integrate the support for these data types, by simply extending the database and recording logic. 
    \item[Create an interface for sharing] A proposition in the design for sharing, was to create an interface soley for sharing data between users, without accessing other applications (e.g., mail). The idea was to create a server that maintains a user-base of patients and researchers/doctors, and a respository for shared records. Thus, by providing an interface for the patient to select a desired recipient, would allow for a simpler and convenient sharing for both the patient and the research/doctor. However, the implications are that the user's data are stored in a server, and without any security or authentication is prone to data leaks. That would be a risk in regards to the GDPR compliance.
    \item[Bi-directional channel between Nidra and modules] As of now, the data is packed into a JSON string and bundled into an Intent on launch. As discussed in the design, this would mean that for the module to obtain the data, the user has to launch the module through the application. For the simplicity, this is sufficent, however, for future modules that depends on analyzing the data in real-time, that is not an applicable solution. Therefore, by establishing an bi-directional channel with Nidra and the modules is a solution to this problem. By utilizing binder's (with AIDL) for IPC, the data flow can occur both ways. Nidra can obtain reports or results from the modules, and the modules can obtain samples in real-time and/or selective desired records respectivly. 
    \item[Filter modules based on package-name] Currently, all of the installed applications are listed when selecting a new module to add in Nidra. To improve the user experience, we can have that modules prerequisite is that the package name should start with \textit{com.cesar.X}---or something similar. With this, we can filter out unwanted applications, and display the module-applications that are eliable to the user.
\end{description}