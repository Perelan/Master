\chapter{Conclusion}

\section{Summary}
In this thesis, we designed an application for recording, sharing, and analyzing breathing data with the Flow sensor kit. The motivation of this thesis was to extend the CESAR project by providing an interface for patients to collect data during sleep, share the data across applications, and analyze the data within the application. The data is supposed to aid researchers/doctors to analyze, diagnose, or examine the patient for sleep-related breathing disorders (e.g., Obstructive Sleep Apnea).  

In the design of this thesis, we identified the tasks to meet the system requirements of this application. Based on the design choices, we implemented an Android application called Nidra. The application leveraged the tools provided by the previous work performed on the CESAR project, in order to manage current sensor sources, as well as future sensor sources. We implemented a sensor wrapper for the Flow sensor kit, which connected with BlueTooth LE for data acquisition. During the recording, the samples from the sensor sources were unpacked and stored in our SQLite database. Also, we ensured for continuous data stream---uninterrupted from sensor disconnections and human disruptions---by reconnecting with the sensor source. At the end of the recording, metadata ---including the state of the user's biometrical data---is stored in the database. Previous records are presented to the users, where each recording had the functionality to view the analytics or share the record. The sharing functionality allowed for transmitting records across applications, such that researchers/doctors can view and analyze the patient's records. The analytic part of Nidra constitutes of a time-series, which enables the users to interact with the data from the recording.  Conclusively, while Nidra's design is for recording sleep with various sensor sources, the application can be applied to other fields of study (e.g., recording respiration during a workout).  

In the field of detecting and diagnosing sleep-related breathing disorders, we could see that related work collects physiological data on various method to predict or examine sleeping disorders (e.g., Obstructive Sleep Apnea). Therefore, by creating an extensible application which can centralize the application of various techniques and methods in this field, they would benefit by creating an extension to our application in order to save time and effort in creating a base application of their project. As such, Nidra provides an interface for future developers and researchers to create modules which enrich the functionality of Nidra or provide data enrichment to the patient's recordings. The modules are independent Android applications, which is installed alongside Nidra and utilizes the provided records by Nidra.  For example, a module could be to use the data on the patient's device to feed a machine learning algorithm that predicts or diagnose the sleeping disorder.  

Finally, we performed experiments on the application to determine whether the system requirements were sufficient. The experiments ranged from collecting respiration data for analysis of musical absorption, testing the application over an extended period, creating a simple module-application, including testing the application on real users. The results and observation made during the testing, allowed us to gain a broader understanding of the application, as well as improvements that can be made--discussed in future work. 

\section{Contribution}
In the problem statement of this thesis, we defined three research goals. In this Section, we restate the goals in together with how our work solves the goals. 

\begin{description}
    \item[Goal 1] \textit{Integrate the support for Flow sensor kit with the extensible data stream dispatching tool.}

    In the implementation chapter of this thesis, we created a sensor wrapper for Flow sensor kit, which integrates with the data streams dispatching tool. The Flow sensor kit uses BlueTooth LE as a communication protocol; as such, we used the APIs provided by Android to connect with the sensor. We focused on extracting the flow (breathing) data and battery level, as well as the mac address and firmware level. The sensor collects data on a frequency of 10Hz; however, transmits a bundle of data points every 1.5Hz. The data points were packed into data packets and sent to the data streams dispatching tool, which is responsible for forwarding the data to subscribing applications, in our case, Nidra. As such, we evaluated the Flow sensor kit by conducting experiments, and the results show an accurate reading of the breath. 

    \item[Goal 2] \textit{Research and develop a user-friendly application which facilitates collection of breathing data with the use of the extensible data streams dispatching tool.}


    Through the work performed on this thesis, we designed and implemented an application which facilitates the collection of breathing data. The desired functionality of the application was separated into concerns. In the design of this thesis, we discuss various methods of implementation for each concern and selected method that fits our purpose.  In the implementation of this thesis, we realize the design choices by creating an Android application, called Nidra. Towards this goal of this thesis, the interface of the application was designed with cautious for the environment of use---it is most likely to be used during the night and early morning---and limiting the amount of actions a user can perform in order to create a simple and efficient application. Moreover, the application faciliate an application that can collect breathing data over an extended period, where the data collected from the sensor source (i.e., Flow sensor kit) were forwarded through the data streams dispatching tool. 


    \item[Goal 3] \textit{Create an extensible solution such that the developers can integrate stand-alone applications in our application.}

    In the design and implementation of this thesis, we facilitate for modules that future developers can create to integrate their functionality into Nidra without having to understand how Nidra operates. The motivation for this functionality is to allow future developers to enrich the data or extend the functionality by using the data provided by Nidra, and for patients to operate with one application for detection for sleep apnea. 
\end{description}

\section{Future Work}
Over the course of this thesis, we were able observe enhancements and improvements that can be made to the application. While the application fulfilles the goals defined in the problem statement, there are possibilities to make the experience of the application even richer. Below, we present the a short description of future work alongside with our proposition. 


\begin{description}
    \item[Improve the user-friendliness of Nidra:] In retrospect, we could see that the participants had trouble with finding the record button, as well as the indication of whether a recording had started. A proposition is to enlargen til record button, to make it more visible to the users. For the recording, perhaps have a more informative description of the various states (e.g., connecting, recording, or disconnected). 
    \item[Add support for other physiological data] The main focus in this thesis, was to gather breathing data during sleep. However, the possibility of extending Nidra to support other physiological data is possible (e.g., heart rate). The extensible approach to application, allows future developers to integrate the support for these data types, by simply extending the database and recording logic. 
    \item[Create an interface for sharing] A proposition in the design for sharing, was to create an interface solely for sharing data between users, without accessing other applications (e.g., mail). The idea was to create a server that maintains a user-base of patients and researchers/doctors, and a repository for shared records. By providing an interface for the patient to select the desired recipient, would allow for a simpler and convenient sharing for both the patient and the research/doctor. 
    \item[Bi-directional channel between Nidra and modules] As of now, the data is packed into a JSON string and bundled into an Intent on launch. As discussed in the design, this would mean that for the module to obtain the data the user has to launch the module through the application. For simplicity, this is sufficient; however, for future modules that depends on analyzing the data in real-time, that is not a suitable solution. Therefore, by establishing a bi-directional channel with Nidra and the modules is a solution to this problem. By utilizing binder's (with AIDL) for IPC, the data flow can occur both ways. Nidra can obtain reports or results from the modules, and the modules can obtain samples in real-time and/or selective desired records, respectively. 
    \item[Filter modules based on package-name] Currently, all of the installed applications are listed when selecting a new module to add in Nidra. To improve the user experience, we can have that modules prerequisite is that the package name should start with \textit{com.cesar.X}---or something similar. With this, we can filter out unwanted applications, and display the module-applications that are eligible to the user.
\end{description}