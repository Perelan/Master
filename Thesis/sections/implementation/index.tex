\chapter{Implementation}
This chapter starts with a brief overview of application components for the three individual applications (i.e., Nidra, data streams dispatching module, and flow sensor wrapper), alongside a description of the applications. Moreover, the implementation of the separate concerns identified in the previous chapter as an Android application called \textit{Nidra}. 

\section{Application Components} \label{impl:appcomp}

\begin{figure}
    \centering
    \includegraphics[scale=0.95]{images/Android_Components.pdf}
    \caption{Applications components for the three individual Android applications in the project and IPC connection between them.}
    \label{fig:app_components}
\end{figure}


In this thesis, we operate with three individual applications: Nidra, data streams dispatching module, and the sensor wrapper for the Flow sensor. Figure \ref{fig:app_components} illustrates the Android components (i.e., activity, service, and broadcast receiver) for each application, which run in a separate process on the user's mobile device. The subsequent sections present a brief overview of the applications structure and components.


\subsection{Flow Sensor Wrapper}\label{imp:flowsensor}
As part of the thesis, the goal is to integrate the support for the Flow sensor. We developed a sensor wrapper that connects with the sensor source by using the BlueTooth LE protocol. To create a sensor wrapper, we followed the instructions provided by Gjøby \cite{gjoby} in order to create a new driver application that connects with the data streams dispatching module (DSDM). Below, a brief overview of the main components of the driver is discussed.

\begin{description}[font=\normalfont\itshape]
    \item[WrapperService:] Is instantiated by the DSDM during the sensor discovery phase (described in Section \ref{imp:dsdm}). This component is responsible for handling starting and stopping of the data acquisition (event sent as broadcasts from DSDM), as well as establish an IPC connection using a binder with the DSDM application. 
    \item[CommunicationHandler:] Upon IPC connection with the DSDM and a request to start data collection, a separate thread of this component is created for interacting with the component that is responsible for integration of the Flow sensor (described in Section \ref{imp:comflow}). This component is mainly responsible for forwarding the data acquired from the sensor to the \verb|DataHandler| component. 
    \item[DataHandler:] Preprocesses the collected data from the Flow source into a data packet before forwarding the packet to the DSDM application. Part of this process is to construct the data packet correctly. The data packet is formatted as JSON string and contains the id (Flow) of the sensor wrapper, the current date and time, and the data points from the sensor source. The data packet is then sent, using the established binder connection, to the DSDM application.
\end{description}

Besides the components which manage the connectivity, collection, and disconnection of the Flow sensor with the DSDM, two activities are responsible for providing an interface to select the Flow sensor and display the state of the Flow sensor on the user's screen:
\begin{description}[font=\normalfont\itshape]
    \item[MainActivity:] Presents the state and information of the selected Flow sensor on the user's screen. Currently, it presents the connectivity state (connected or disconnected), the battery level of the sensor, the MAC address and the firmware level, as well as the option to remove or connect to another sensor source. 
    \item[DeviceListActivity:] Available devices or sensors that are in range for BlueTooth connectivity with the mobile device is presented to the user. The user has to select the correct  Flow sensor that will be used for data acquisition.  As a feature, the Flow sensor has a distinguishable icon to make it easier to select the correct sensor source amongst other devices or sensors that is in range of the user's mobile device.
\end{description}
The Flow sensor wrapper stores the name and the MAC address of the selected Flow sensor in a \verb|SharedPreference|. As such, the user has to configure the sensor wrapper once, and the information remains persistent in the application.

The preceding components are a part of the driver to connect with the data streams dispatching module. However, the communication with the sensor source is not a part of these components. The communication with the Flow sensor occurs over BlueTooth LE (BLE) protocol which is designed to provide lower power consumption on data transmission, and sensors that utilize BLE is designed to last for a more extended period. In order to connect with BLE sensors, we can use the API's provided by Android. The implementation of the API's is introduced as a new component (\textit{BluetoothHandler}), which solely communicates with the Flow sensor over BLE. Below, a brief overview of how to establish a connection and interpret the collected data from the Flow sensor through BLE in Android is described. However, a more detailed description of implementation can be found in Appendix D.


\subsubsection{Communicating with the Flow Sensor}\label{imp:comflow}
In order to communicate with the Flow sensor, we have to use the BlueTooth Low Energy (BLE) protocol. To begin with, the user has to select the desired Flow sensor to use for collecting the data. As such, when the command of starting the collection is passed to the \verb|StartReceiver| broadcast receiver which is handled by \verb|WrapperService|, a separate thread of \verb|CommunicationHandler| is created. This thread, start the service of \verb|BluetoothHandler| and initializes the connection to the selected Flow sensor based on the MAC address of the sensor. 

The \verb|BluetoothHandler| is the component we introduce, which manages the connection to the Flow sensor, discovers services provided by the sensor, and manages the decoding of the data received from the sensor. This component acts as a GATT client which connects with a GATT server. The GATT server in our case is the Flow sensor, which provides a service that encompasses several characterstics that contains values and descriptors. In BlueTooth, the objects (e.g., services and characteristics) are identified by a universally unique identifier (UUID)\footnote{A standardized 128-bit format for string ID to uniquely identify information}, and there is a collection of assigned numbers to standard objects \cite{uuid}. The UUID for GATT attributes for BLE accordingly to BlueTooth is structured as following \textit{PREFIX-0000-1000-8000-00805f9b34fb}, where the prefix is the assigned number that categorizes an individual object. The most interesting characteristics to us, are the manufacturer name (prefix: 0x2A29), firmware revision (prefix: 0x2A26), battery level (prefix:0x2A19) and flow (breathing) measures (prefix: 0xFFB3). The latter characteristic prefix is not a part of the standard; however, the manufacturer defined prefix. Also, it is of most interest to us as it contains the values for the breathing data. 

To receive flow (breathing) data and the battery level from the sensor source, we have to enable it by notifying the GATT server. This is performed by specifying the service and the underlying characteristics we want the values from. For example, to enable flow (breathing) data, we specify the service (prefix: OxFFB0) and characteristic (prefix: 0xFFB3), and send it with the API provided by Android.  As such, we enable the Flow sensor to collect breathing data. 

The Flow sensor gatherers data at a frequency of 10 Hz; however, the data from the sensor is sent to the connected devices on approximately 1.5 Hz. Which means each data received from the sensor contains 5-7 data points with a timestamp of acquisition. We proceed to smooth out the data points by averaging the values, which statistically is to filter out misfits of values and to find an estimate of value on a given time. The data is sent to the \verb|CommunicationHandler| which further sends it to the \verb|DataHandler|. The \verb|DataHandler| creates a data packet with the data as a JSON string, and sends the data packet on the binder (created in the \verb|WrapperService|) between the sensor wrapper and DSDM on the method \verb|PutJson()| (described in Section \ref{implement:aidl}). 

When the command of stopping the collection is passed to \verb|StopReceiving| broadcast receiver, the \verb|CommunicationHandler| thread is interrupted. The interruption closes and unbinds the connection with the \verb|BluetoothHandler|. Within the \verb|BluetoothHandler| the connectivity with the GATT server (sensor) is disconnected and closed. Finally, the screen presented to the user shows that the sensor has disconnected.

To summarize, the \verb|BluetoothHandler| connects with the selected Flow sensor with the BlueTooth LE protocol by using the API's provided by Android. By specifying the appropriate service and characteristic we can obtain breathing data from the sensor. The sensor collects data at a frequency of 10 Hz; however, sends a small burst of data with 5-7 data points for a timestamp at a frequency of 1.5 Hz. Moreover, we procced to smooth out the data points by calculating the average value. The data is then sent to the \verb|CommunicationHandler| which further sends it to the \verb|DataHandler|. The \verb|DataHandler| creates a data packet of value including metadata, and forwards the data packet to the data streams dispatching module.  


\subsection{Data Streams Dispatching Module}\label{imp:dsdm}
The data streams dispatching module developed by Bugajski \cite{daniel} provides an interface for application instances to subscribe to data packets from supported and available sensor sources. The modularity this application provides towards managing and supporting various sensor source allows for faster development time. Also, it provides a common interface for communication with sensor sources that have different Link Layer technology (e.g., BlueTooth) and low-level communcation protocols (e.g., vensors that provides sensor spesific SDK). 

This application discovers available sensor wrappers installed on the user's mobile device, establishes connection with the sensor source on requests, and forwards data packets from the sensors to the subscribed applications (all in the \verb|DataStreamDispatchingService| component). Below, we briefly introduce the components and their mechanism in the application.

\begin{description}[font=\normalfont\itshape]
    \item[Sensor Discovery] The initial design of discovery for installed sensor wrappers was performed by sending a broadcast with an action of \textit{HELLO} as an intent. All sensor wrappers are designed to respond to this action with their package name as \textit{id} and the name of the sensor wrapper as \textit{name}. This application is then aware of which sensor wrappers that are available on the mobile device. 
    
    However, during the development of this thesis, Android had limited the use of implicit broadcasts on newer Android versions \cite{broadcasterror}. Implicit broadcasts are those broadcast that does not target a specific application; however, sends out an action with a message and those application that filters and listens for the actions can handle this message accordingly. To overcome this, a re-design of the sensor discovery was made. Instead of DSDM ever so often sends out a HELLO broadcast, the sensor wrapper sends out an explicit broadcast directed to the DSDM to make it aware of its existence. The broadcast is sent to the \textit{SensorDiscovery} directed explicitly to DSDM broadcast receivers, encapsulated with the name and the package name of the sensor wrapper. The DSDM stores the sensor wrapper information in a \verb|SharedPreference|, which will be used locate the sensor wrapper based on the request from the subscribing applications.

    \item[DataStreamDispatchingService] Encompasses most of the functionality of the application. This component acts as a data distributor between applications that desire to connect with sensor sources, and sensor sources forward their data packets to this service. Also, this component ensures to duplicate and make identical data packets to all subscribed applications. The actions that are exposed to the sensor wrapper and the subscribed applications are managed within this component by providing an interface with a binder for IPC communication (see Section \ref{implement:aidl}). 

\end{description}


To summarize, this application acts as a data distributor for supported sensor sources. Subscribing applications can select a desired sensor source (e.g., Flow) for data acquisition. The data received from the sensor source are obtained in the sensor wrapper and sent to this application as data packets. This application duplicate the data packets to those applications that subscribe to the sensor source and send it to the them accordingly. 


\subsection{Nidra}
In this thesis, we focus on creating an interface that can record, share, and analyze breathing data of a patient over an extended period using the Flow sensor. In order to provide an interface, we developed an Android application called Nidra. This section describes the components that Nidra constitute of, while in Section \ref{impl:ioc} we discuss the implementation of Nidra. Below, we describe the main components of the application:
 
\begin{description}[font=\normalfont\itshape]
    \item[MainActivity:] Encompasses most of the functionality in the application, besides recording, by using fragments as a modular approach to seperate functionality. The fragments lie on top of this host activity, and a transition amongst the fragments is triggered based on user interactions. 
    \item[RecordingActivity:] This activity manages the recording part of the application by invoking the \verb|RecordingFragment|. However, the fragment handles the connectivity with the data streams dispatching module, handles the data packets from the sensor sources, and assuring for reconnecting with the sensor on human disruptions and sensor disconnections. Also, it manages the interactions that can be performed on the recording screen (e.g., real-time graph).
    \item[LandingActivity:] When launching the application for the first time, a screen with an introduction to the application is shown to the user. Further, the user is prompted to type in biometrical data (e.g., name, gender, age, height, weight), which will be used to enrich the recording of the patient. 
\end{description}

Moreover, Nidra leverages the functionality that the data streams dispatching module provides. To use the functionality, the recording activity connects with the data streams dispatching module, and the reference that will be used to receive data packets are sent to the service of \textit{DSDService}. This service implements the interface provided by the data streams dispatching module, and the data obtained within this service are directly sent to the \verb|RecordingActivity| for processing. 


To summarize, the components discussed in this section constitute an application that enables recording, sharing, and analysis of breathing data obtained from the Flow sensor. Further, we elaborate on the actions and functionality of Nidra in Section \ref{impl:ioc}.  


\subsection{Inter-Process Communication}\label{implement:aidl}
In order to communicate with between the applications, such as remote procedure calls (RPC) to application components that run remotely, we can use the IPC mechanisms. In Android there are two viable mechanisms to enable IPC: (1) \verb|Binder| enables a process to invoke functions in another process remotely; and (2) \verb|Intent| a message passing interface allowing applications to send messages to each other. In this section we describe how these mechanisms are used in Nidra.

The Intent mechanism is mainly used for sharing bundles of primitive data types (e.g., strings or floats) across activities and applications, as long as the reference (e.g., package name) is valid. Another possibility of using intent is during broadcasts within the application or to other application. In Nidra, \verb|Intent|'s are used to start activites, share data between components with local broadcasts, as well as when launching a module (discussed in Section \ref{impl:ioc}).

To implement the binder mechanism, we can use design of the data streams dispatching modules which use Android Interface Definition Language (AIDL). In order to communicate with processes, the data objects have to be decomposed into primitives that the operating system can understand. AIDL provides this mechanism by providing a programming interface that both the client and the service agree upon. The AIDL interface is defined in an \verb|.AIDL| file, and located in the \verb|src/| directory of the hosting service application (DSDM), likewise, for other applications that bind to the hosting service (Nidra and sensor wrappers). It is essential to have identical \verb|.AIDL| files across the applications, otherwise the system will not recognize it as the same interface. In Listing 5.1, the interface is based on the functionality of the hosting service application provides (DSDM). 

\begin{lstlisting}[language=json, caption={An interface provided by the host service (i.e., DSDM) that provides functionality other applications can use (e.g., Nidra and sensor wrappers)}, captionpos=b]
// MainServiceConnection.aidl
package com.sensordroid;

interface MainServiceConnection {
    void putJson(in String json);
    int Subscribe(String capabilityId, int frequency, String componentPackageName, String componentClassName);
    int Unsubscribe(String capabilityId, String componentClassName);
    String Publish(String capabilityId, String type, String metric, String description);
    void Unpublish(String capabilityId, String key);
    List<String> getPublishers();
}
\end{lstlisting}

In Nidra, some of the functionality is utilized to enable recording. More specifically, \verb|getPubishers()| method is used to get all of the sensors publishers (e.g., Flow sensor), the \verb|Subscribe()| and \verb|Unsubscribe()| is used in order to subscribe and unsubscribe to a specific sensor publisher, and the data that is forwarded from by the sensor publisher sent to the \verb|putJson()| method.

As for the Flow sensor wrapper, uses the same interface to communicate with the data streams dispatching module. However, the only method call is towards \verb|putJson()| when forwarding its data packets. The data streams dispatching module is aware of which type of connectivity (e.g., publisher or subscriber) that forwards the data, therefore, the call by the sensor wrapper is processed as data packets that will be sent to the subscribing applications. 

\section{Implementation of Concerns} \label{impl:ioc}
In the design chapter of this thesis, we conceptualized the tasks by decomposing them into separate concerns and discussing design choices for implementation. This section realizes the discussion by implementing the concerns in Android and developing the application Nidra. For each concern, we illustrate the components and objects that interact with each other, step-by-step; the legend for the figures are shown in \Cref{fig:legend}. 

\begin{figure}[!h]
    \centering
    \includegraphics[scale=0.7]{images/Legend.pdf}
    \caption{Legend for the figures in implementation of concerns: (A) application components with integration of our logic; (B) objects that contains specifics of our logic; (C) an interface for callbacks or listeners; (D) Android-specific objects and components; (E) other installed applications; (F) steps direction; and (G) reference or data flow direction.}
    \label{fig:legend}
\end{figure}

\subsection{Recording}
The design choices for this concern is described in Section X. To give a brief overview; recording is the process of collecting and storing data received from sensors over an extended period. To enable a recording, we need to establish a connection with the available sensors and store the samples retrieved by the sensors on the device. In this thesis, we focus on collecting breathing data from the Flow sensor. 

We will use the data streams dispatching module (hereafter: DSDM), which manages sensor discovery and sensor establishment to supported sensor sources, in our case the Flow sensor.  Moreover, the DSDM facilitates an interface for data acquisition, and the communication between the DSDM and Nidra occurs over IPC using binder's. During recording, we will check for connectivity with the sensors to ensure the sensors are collecting data at an appropriate rate. At the end of the recording, we will store metadata related to the recording and finalize the recording process. 

The functionality of recording can be separated into three actions: (A) start recording; (B) stop recording; and (C) display recording statistics. In the following sections, we will review the steps that enable these actions. 

\subsubsection{Action A: Start Recording}
\begin{figure}
    \centering
    \includegraphics[scale=0.7]{images/Recording_ImpA.pdf}
    \caption{Implementation of recording functionality: (A) start recording}
    \label{fig:impl_recordingA}
\end{figure}

In Figure \ref{fig:impl_recordingA}, an illustration of the component interactions are shown. Action A is to start a recording by connecting and starting data aqusition with the use of DSDM, and to ensure persistent connectivity with the sensor sources. The steps and interactions for this action are: 

\begin{itemize}
    \item[A.1] The recording process starts by creating a new record entity (see: Section X) that is inserted into the SQLite database. An empty record has to be inserted into the database in order to associate new samples with the record (based on the record's id). 
    \item[A.2] Once the record is inserted into the storage, a unique identification (id) is returned. 
    \item[A.3] \verb|ConnectionHandler| is invoked in order to manage the establishment, connection, and disconnection of the IPC between Nidra and DSDM service. The code for establishing the connection is described in the following listing:
\begin{lstlisting}[language=json, caption={Code snippet for connecting with the DSDM (MainServiceConnection is the AIDL file as discussed in Section X)}, captionpos=b]
    Intent intent = new Intent(MainServiceConnection.class.getName());
    intent.setAction("com.sensordroid.ADD_DRIVER");
    intent.setPackage("com.sensordroid");
    context.bindService(intent, serviceCon, Service.BIND_AUTO_CREATE);
\end{lstlisting}
    
    \item[A.4] If the service is offline when binding, the flag \verb|Service.BIND_AUTO_CREATE| will ensure for starting the service. \verb|BindService| allows components to send requests, receive responses, and perform inter-process communication (IPC) based on the interface provided by the host service (DSDM). 
    \item[A.5] Once the service is bound, we can proceed to communicate with the DSDM service. 
       \item[A.6] The \verb|ConnectionHandler| proceeds to initialize the connection with the sensor through the DSDM.  A request to the DSDM for available publishers with \verb|getPublishers()| is made, to retrieve all available sensor publishers connected to the DSDM. Occasionally, the DSDM uses extended time to discover all of the active sensors connected to the device; therefore, we have an interval that checks whether DSDM has any available sensors connected. 
    \item[A.7] Moving on,  a request to the DSDM to \verb|Subscribe| to a sensor is made. We specify that we want the Flow sensor in the \verb|Subscribe| method, in addition, a reference to the package name (Nidra) and a service object (\verb|DSDService|). The service object is where all of the data packets from the subscribed Flow sensor is received (on the \verb|putJson()| method).   
    \item[A.8] Also, a callback to \verb|RecordingFragment| with information of the sensor source (i.e., Flow) that we subscribe to is made, in order to display the information on the user's screen. 
    \item[A.9] The recording has now started, and a timer to measure the time spent on the recording is started. The \verb|ConnectivityHandler| is also initialized, which actively checks that the samples arrive within a specified time frame (as discussed in the design, a frame of 10 seconds that increases throughout the recording). The \verb|ConnectivityHandler| is implemented with a \verb|Handler| with a \verb|PostDelay| that counts down. Upon a sample arrival, the timer is reset. 
    \item[A.10] Periodically, the DSDM receives samples from the subscribed sensor. DSDM forwards the sample from the sensor to the service object (\verb|DSDService|) on the \verb|putJson()| method. The DSDService uses a \verb|LocalBroadcastManger| to send the data packet to the \verb|RecordingFragment|.  
    \item[A.11] \verb|RecordingFragment| listens for the events on the local broadcast receiver. Upon an event, the data that is received from the sensor is inserted as a new sample entity with the current record's id as an association. 
    \item[A.12] Recalling the functionality from step A.9; if the event for the \verb|PostDelay| is triggered, it is equivalent to a sample not being acquired from the sensor. Therefore, we try to reconnect with the subscribed sensor by disconnecting with the DSDM (which will close the connection with the sensor source), followed up by a reconnection with the DSDM and subscribing to the same sensor source. Most of the times, the process of reconnecting works instantaneously; however, some times the sensor might require to be reconnected with several times. 
\end{itemize}

\subsubsection{Action B: Stop Recording}
\begin{figure}
    \centering
    \includegraphics[scale=0.7]{images/Recording_ImpB.pdf}
    \caption{Implementation of recording functionality: (B) stop recording}
    \label{fig:impl_recordingB}
\end{figure}

In Figure \ref{fig:impl_recordingB}, an illustration of the component interactions are shown. Action B is based on user input to stop the recording process. To the user, the recording has terminated, and the user is presented with a screen to provide extra information regarding the recording (e.g., title, description, and rating). For the application, it has to unsubscribe from the connected sensor sources (i.e., Flow), and disconnect the connectivity with the DSDM. Also, transition the user to the screen where the user can provide extra information that will be stored alongside the record.

\begin{itemize}
    \item[B.1] The user decides when to stop a recording with a press of a button. The event to stop the recording is sent to the \verb|ConnectionHandler|.
    \item[B.2] A call to the \verb|Unsubscribe()| method that contains the service object (i.e., DSDService) and the identification of the sensor source (e.g., Flow) is sent to DSDM. The DSDM has to ensure unsubscribing the sensor from a specific application and disconnect the IPC between the application. If there are no subscribing applications to the specific sensor source, the DSDM will signal the sensor to stop sampling and disconnect with the sensor. 
    \item[B.3] The IPC connection between Nidra and DSDM is discontinued by unbinding the service. 
    \item[B.4] The estimated time of recording is calculated, and a transition from \verb|RecordingFragment| to \verb|StoreFragment| is made to finalize the recording with extra information (e.g., title, description, and rating). 
    \item[B.5] The \verb|StoreFragment| uses the record identification retrieved on recording (A.1) in order to enrich the record with statistics and user-defined metadata. The statistics are the monitoring time, number of samples during recording, and retrieving the current state user biometrical data and storing it in the record. The user-defined metadata are the title of the recording, a description enabling the user to add a note to the recording, and a rating between 1--5 (to give a rating on how the recording felt). 
    \item[B.6] The modified record is updated in the database, and the user is transitioned to the \verb|MainActivity|.
\end{itemize}

\subsubsection{Action C: Display Recording Statistics}
During a recording, the user can view the statistics for the recording. More specifically, the user can see the available connected sensors and a graph of the breathing data in real-time. In Figure \ref{fig:impl_recordingC}, an illustration of the component interactions are shown, and the steps and interaction for this action are: 

\begin{figure}
    \centering
    \includegraphics[scale=0.7]{images/Recording_ImpC.pdf}
    \caption{Implementation of recording functionality (C)}
    \label{fig:impl_recordingC}
\end{figure}

\begin{itemize}
    \item[C.1] The data is graphically represented as an intractable time-series graph. By using the Graph library [CITE], we can in similarities to the implementation of the analytics concern (see Section X), implement a graph to illustrate the respiration data to the user.
    \item[C.2] In addition to the time-series graph, we have a list of publishers (e.g., Flow) that we acquired in the begging of the recording. As such, a list of publishers is sent to \verb|SensorAdapter|. 
    \item[C.3] The \verb|SensorAdapter| populates a view with the connected sensor to the user.
\end{itemize}


\subsection{Sharing}

Sharing enables users to transmitt records across application over a media. The functionality of sharing is separated into two concerns, namely importing and exporting records. Hence, the actions for sharing are separated into: (A) exporting one or all records; and (B) import a record from the device. 

Before a user can take these actions, the records from the database have to be presented. The \verb|Feed Fragment| contains a \verb|RecyclerView| which populates the records into inside the \verb|Feed Adapter| (steps: 1-4). The adapter contains all the interactions and the event handling (i.e., button event listener for exporting) for a single record. 

In this Subsection, we will take a look into the steps that are taken to enable the actions:

\begin{figure}
    \centering
    \includegraphics[scale=0.6]{images/Sharing_ImpA.png}
    \caption{Implementation of sharing functionality (A): Exporting one or all Records}
    \label{fig:impl_sharingA}
\end{figure}

\subsubsection{Action A: Exporting one or all Records}
In Figure \ref{fig:impl_sharingA}, an illustration of the steps to export one single recording is shown. However, the \verb|Feed Fragment| has an option to export all record; therefore, by disregarding the first step (A.1), the same structure applies to export all records. In essence, exporting consists of bundling the records and corresponding samples into a formated file, and prompting the user with options to select a media (e.g., mail) for transmittion. The steps can be narrowed down to: 

\begin{itemize}
    \item[A.1] Upon an event for exporting a selected record in \verb|Feed Adapter|, the record information is sent to the \verb|Feed Fragment| through the callback reference (\verb|onRecordAnalyticsClick|) between these components. The record information will be used to determine the corresponding samples for the record.
    \item[A.2] The \verb|Feed Fragment| delegates record information to the \verb|export| method inside of the \verb|Export| class. The class is responsible for enabling exportation. 
    \item[A.3] An operation to retrieve all samples related to the record with the use of the \verb|SampleViewModel| is done. 
    \item[A.4] The \verb|export| method retrieves all of the samples related to the record. Next, the record and the samples are encoded into an exportable JSON format (Ref: Data Format). To enable the sharing interface provided by Android, the content has to be stored on the device. Thus, the encoded data is written into a file on the device, with a filename of \verb|record_(current_date).json|, and the next component uses the reference to the file location. 
    \item[A.5] The encoded file is retrieved with the use of \verb|FileProvider| (facilitates secure sharing of files [ref]). The code for this step are
\begin{lstlisting}[language=json, caption={My Caption}, captionpos=b]
static void shareFileIntent(Activity a, File file) {

    Uri fileUri = FileProvider.getUriForFile(a.getApplicationContext(), a.getApplicationContext().getPackageName() + ".provider", file);

    Intent iShareFile = new Intent(Intent.ACTION_SEND);
    iShareFile.setType("text/*");
    iShareFile.putExtra(
        Intent.EXTRA_SUBJECT, "Share Records");
    iShareFile.putExtra(Intent.EXTRA_STREAM, fileUri);
    ...

    a.startActivity(
        Intent.createChooser(iShareFile, "Share Via"));
}

\end{lstlisting}

    \item[A.6] The user is displayed with a popup interface with several options to share the file over a media. An illustration of the layout can be found in Section Representation. 


\end{itemize}


\begin{figure}
    \centering
    \includegraphics[scale=0.6]{images/Sharing_ImpB.png}
    \caption{Implementation of sharing functionality (B)}
    \label{fig:impl_sharingB}
\end{figure}

\subsubsection{Action B: Import a Record from the Device}
In Figure \ref{fig:impl_sharingB}, an illustration of importing a record from the device is shown. Importing conists of locating the formated file (the user has to obtain the file and store it on the device on beforehand), parsing the content in the file, and storing the data respective to the users database. The steps can be narrowed down to:

\begin{itemize}
    \item[B.1] The user requests to view the import record interface. The interface is provided by Android, and allows the user to select particular kind of data on the device (ref). The code for this action is:
\begin{lstlisting}[language=json, caption={My Caption}, captionpos=b]
private void importRecords() {
    Intent intent = new Intent(Intent.ACTION_GET_CONTENT);
    intent.setType("*/*");
    startActivityForResult(intent, 1);
}

\end{lstlisting}
    \item[B.2] Once the user has selected the desired file, the method \verb|onActivityResult| inside of \verb|Feed Fragment| is called, and location of the selected file can be located. 
    \item[B.3] The file location is an obscured path to the file on the device; thus, parsing the path with the use of \verb|Cursor| method has to be done. After the absolute path is found, the data is decoded accordingly to the data format, and the records are sent back to \verb|Feed Fragment|.
    \item[B.4] The necessary record information and the samples are extracted from the decoded data, and are inserted into the users database. 
\end{itemize}

\subsection{Modules}
Modules are standalone application, that provides data enrichment and extended functionality to the application. The modules leverages the records and samples to analyze, evalutate or detect sleeping disorders. In order to add and launch a module in Nidra, we need the modules package name. The package name and the name of the module-application can be obtained in Android. Thus, the actions to enable modules in the application are: (A) add a module; and (B) launch a module. 

Before a user can take these actions, the records from the database have to be presented. The \verb|Feed Fragment| contains a \verb|RecyclerView| which populates the records into inside the \verb|Feed Adapter| (steps: 1-4). The adapter contains all the interactions and the event handling (i.e., button event listener for exporting) for a single record. 

In this Subsection, we will take a look into the steps that are taken to enable the actions.

\subsubsection{Action A: Add a Module}
In order to add a new module, the user has to install the module-application on the device on beforehand. By listing through the installed application on the device, the user can select the desired module to be added in Nidra. In Figure \ref{fig:impl_modulesA}, an illustration of adding a module is shown, and the steps can be narrowed down to:

\begin{figure}
    \centering
    \includegraphics[scale=0.7]{images/Module_ImpA.pdf}
    \caption{Implementation of module functionality(A): Add a Module}
    \label{fig:impl_modulesA}
\end{figure}

\begin{itemize}
    \item[A.1] Upon an event for adding a new module in \verb|Modules Adapter|, the \verb|Feed Fragment| is notified through the callback reference (\verb|onNewModuleClick|) between these components.
    \item[A.2] The \verb|Modules Fragment| lauches a custome Android dialog, which will list all of the installed application on the device. 
    \item[A.3] The \verb|Apps Adapter| will fetch all of the application that is not a system package, already installed module, or the current application (Nidra). Next, the the adapter for the dialog will be populated with the eligible applications. 
    \item[A.4] Once the user has selected the desired module-application, an event to the \verb|Modules Fragment| through the callback reference \verb|onAppItemClick| between these components are made. The callback contains an object with the \verb|PackageInfo| for the selected module-application.
    \item[A.5] The dialog is dismissed, and the application name and packagename are extracted from the \verb|PackageInfo| for the selected module-application. 
    \item[A.6] Furthermore, the acquired information is stored in our database for modules through the DAO interface. 
\end{itemize}

\subsubsection{Action B: Launch a Module}
A module is launched in a seperate process, due to Android prohibits launching for other applications inside of an application. All added modules are listed and presented to the user in a separate screen, and on launch of a module, all of the data that Nidra obtains from recordings, are encoded into a JSON format and bundled with the launch of the module. In Figure \ref{fig:impl_modulesB}, an illustration of launching a module, and the steps can be narrowed down to:

\begin{figure}
    \centering
    \includegraphics[scale=0.7]{images/Module_ImpB.pdf}
    \caption{Implementation of module functionality(B): Launch a Module}
    \label{fig:impl_modulesB}
\end{figure}

\begin{itemize}
    \item[B.1] Upon an event for launching a module in \verb|Module Adapter|, the packagename of the module is sent to the \verb|Modules Fragment| through the callback reference (\verb|onLaunchModuleClick|) between these components. The packagename  will be used to launch the module-application.
    \item[B.2] All of the records and samples on the device for the user, is bundled and formated into a JSON, and launched:
\begin{lstlisting}[language=json, caption={My Caption}, captionpos=b]
public void onLaunchModuleClick(String packageName) {
    Intent moduleApplication = context.getPackageManager().getLaunchIntentForPackage(packageName);

    if (moduleApplication == null) return;

    String data = formatAllRecordsToJSON();

    Bundle bundle = new Bundle();
    bundle.putString("data", data);

    moduleApplication.putExtras(bundle);

    startActivity(moduleApplication);
}
\end{lstlisting}

    \item[B.3] The activity uses the data provided in the \verb|Intent| that includes the packagename (the name of the module-application to determine the correct application).
    \item[B.4] The selected module is then launched, and presented to the user. The user can at anytime press the back button, to return to Nidra.  
\end{itemize}

%\begin{figure}
%    \centering
%    \includegraphics[scale=0.6]{images/Module_ImpA.png}
%    \caption{Implementation of module functionality(A)}
%    \label{fig:impl_modulesA}
%\end{figure}



%\subsubsection{Data Exchange Implementation}


\begin{figure}
    \centering
    \includegraphics[scale=0.6]{images/Anal_Imp.png}
    \caption{Implementation of analytics functionality (A): Display a Graph for a Single Record}
    \label{fig:impl_analytics}
\end{figure}

\subsection{Analytics}
Analytics is the part of illustrating and analyzing the records. In Nidra, the analytics part of the implementation is limited to a time-series graph for a single record. However, there are possibilities of extending the \verb|Analytics Fragment| with other graphs based on the current structure. The current action for analytics is A) display a graph for a single record to the user. 

Similar to sharing, the records from the database have to be presented. The \verb|Feed Fragment| contains a \verb|RecyclerView| which populates the records into inside the \verb|Feed Adapter| (steps: 1-4). The adapter contains all the interactions and the event handling (i.e., button event listener for analytics) for a single record. 

In this Subsection, we will take a look into the steps that are taken to enable the action.

\subsubsection{Action A: Display a Graph for a Single Record}
Nidra provides a simple time-series graph of respiration data obtained during the recording. The graph data is plotted into a Library, which enables interactions (e.g., zoom and scrolling) on the data. The X-axis is the respiration value based on the Y-axis time of sampling. In Figure \ref{fig:impl_analytics}, an illustration of displaying a graph shown. The steps can be narrowed down to:

\begin{itemize}
    \item[A.1] Upon an event for analytics on a selected record in \verb|Feed Adapter|, the record information is sent to the \verb|Feed Fragment| through the callback reference (\verb|onRecordAnalyticsClick|) between these components. The record information will be used to determine the corresponding samples for the record.
    \item[A.2] A new instance of the \verb|Analytics Fragment| is created, and an transition from the \verb|Feed Fragment| to the \verb|Analytics Fragment| is made. Alongside, the record information is transmitted.
    \item[A.3] An operation to retrieve all samples related to the record with the use of the \verb|SampleViewModel| is done. 
    \item[A.4] The \verb|Analytics Fragment| retrieves all of the samples related to the record. The samples has to be structured according to the graph library to display an interactive time-series graph. (GraphView (ref)).
    \item[A.5] Each sample has to be extracted from the sample-data, acccordingly to the sensor data structure.
    \item[A.6] The sample value is returned, and inserted into an array over datapoints used in the graph. 
    \item[A.7] The use is presented with a graph, which is interactable. The Y-axis has the sample value on given time (in HH:MM:SS) on the X-axis. The graph library enables interactions (e.g., zooming and scrolling) the user can do, to gain a better understanding of the recording. 
\end{itemize}



\subsection{Storage}
\begin{figure}
    \centering
    \includegraphics[scale=0.60]{images/Storage_Imp.pdf}
    \caption{Entity Relationship Diagram}
    \label{fig:impl_storage}
\end{figure}


Storage facilities persistent data which remain available after application termination. In Nidra, there are four individual data entities (i.e., record, sample, module, and user). In the Subsection about Data Entities, we discussed the design choices of the individual data entity. The standard CRUD operations on each data entities are: insert, update, and delete. Also, the user has an operation to retrieve the biometrical data, module and record have an operation to retrieve all of the entries in the database, and samples have an operation to retrieve all of the entries corresponding to a record. 

Android Room provides an abstract layer over SQLite to enable easy database access [Cite]. In Figure \ref{fig:impl_storage}, the flow for accessing and retrieving the data from the database based on the Android Room architecture is shown and can be described as:

\begin{itemize}
	\item[1] Each data entity has a \verb|ViewModel| where all of the CRUD operation (e.g., insert, update, delete, or retrieve) goes through.  A view model is designed to store and manage UI-related data in a conscious way, that allows data to be persistent through configuration changes (e.g., screen rotations) [CITE]. 
 	\item[2a] The predefined operations point to the repository. Repository modules handle data operations and provide an API  which makes data access easy. A repository is a mediator between different data sources (e.g., database, web services, and cache) [CITE]. In Nidra, the only data source is the database, but repository facilities future data sources. 
	\item[2b] The storage of the user is not in the database; however, in a \verb|SharedPreference| on the device. Shared preference points to a file containing key-value pairs and provides methods to read and write. The location of the user's shared preference is \verb|no.uio.cesar.user_storage|. 
	\item[3] Each data entity (disregarding user) has a data access object (DAO), where the SQL operations to the database are defined. 
	\item[4] Based on the operation, the data is accessed and retrieved from the SQLite database.
\end{itemize}


\subsection{Presentation}
In this Subsection we will present the user-interface (UI) based on the functionality of recording, sharing, module and analytics. The interface is developed based on the design descisions of the application, as well as creating a user-interface that is simple and efficent for a user to interact with. We try to limit the actions the user can take on a screen, to make the application simpler to understand and comprehend. 

\subsubsection{UI: Recording}
\begin{figure}
    \centering
    \includegraphics[scale=0.26]{images/Recording_img.pdf}
    \caption{The recording screen displayed to the user; with the screen: (A) during a recording, (B) real-time analytics, and (C) finalizing the recording}
    \label{fig:screen_recording}
\end{figure}

Figure \ref{fig:screen_recording} shows the screen for: (A) during a recording, (B) the real-time analytics, and (C) finalizing the recording. 

\begin{itemize}
    \item[A] The screen has a ripple effect to indicate the state of the recording to the user. There are two types of ripple colours; a blue ripple effect for samples acquistion, and a grey ripple effect if the sensor has disconnected. The ripple effect is only active if the screen is turned on, in order to perserve battery life. During disconnects between the sensor and the device, the user provide no extra input to resolve the issue. However, if the numbers of attempts is reached, the user is presented with the finalizing screen (C). Also, this screen display the elapsed recording time, and has a visible button to stop the recording. 
    \item[B] The user can extend the interface for viewing statistics regarding the recording session. Currently, the interface lists all of the available sensor sources, also a real-time interactable graph of respiration.
    \item[C] The finalzing screen allows users to specify the title and description of recording, to make it distingushable, in addition to giving a feedback on recording so the doctors/researchers can review it. Also, the user can give a rating between 1--5 (where 1 is bad, and 5 is good) to rate the sleeping.

\end{itemize}

\subsubsection{UI: Sharing}
\begin{figure}
    \centering
    \includegraphics[scale=0.26]{images/Sharing_img.pdf}
    \caption{The sharing screen displayed to the user}
    \label{fig:screen_sharing}
\end{figure}

Figure \ref{fig:screen_sharing} shows the screen for: (A) option to import or export, (B) the media selection for exporting, and (C) the file selection for importing.

\begin{itemize}
    \item[A] The feed screen is where all of the records are presented to the user. The user can choose to export one single record, or export all. 
    \item[B] By pressing export all, an overlay with a Android provided sharing screen is presented, where the user can choose a media to export the file on. 
    \item[C] By pressing import, an overlay with downloaded files on the users device is presented. The user can press on the desired file, and the file will be parsed and added to the users collection of records.  
\end{itemize}

\subsubsection{UI: Modules}
\begin{figure}
    \centering
    \includegraphics[scale=0.26]{images/Modules_img.pdf}
    \caption{The module screen displayed to the user}
    \label{fig:screen_modules}
\end{figure}

Figure \ref{fig:screen_modules} shows the screen for: (A) list without any modules, (B) list of installed application, and (C) list with modules. 

\begin{itemize}
    \item[A] The modules screen is shows all of the added modules to the users.
    \item[B] The user can press the \textit{add new module} button, in order to be presented with a list of all installed applications. 
    \item[C] Once the user has selected a module-application to be added in Nidra, the list of modules are updated and presented to the user.
\end{itemize}

\subsubsection{UI: Analytics}
Figure \ref{fig:screen_analytics} shows the screen for: (A) single record in the list, and (B) the analytics for the record

\begin{itemize}
    \item[A] The user can expend records in order to view more functionality and statistics of the record. One of the functionalities are to view the analytics of the record.
    \item[B] An interactable graph is presented to the users, and the graph is populated with the samples obtained from the recording for selected records. The graph shows the respiration value (Y-axis) on given time (X-axis) of sampling. 
\end{itemize}

\begin{figure}
    \centering
    \includegraphics[scale=0.26]{images/Analytics_img.pdf}
    \caption{The analytics screen displayed to the user}
    \label{fig:screen_analytics}
\end{figure}


\section{Miscellaneous}

\subsection{Collecting Data Over a Longer Period}
In Android, applications which are idle in the background or not visible to the user can be killed in order to reclaim resources for other applications or preserve battery time. However, this mechanism is not viable for collecting data over an extended time, because it can kill our applications during recording. To overcome this, there are several methods to prevent the Android system from killing our applications, which is presented in the subsequent sections.

\subsubsection{Keep the CPU Alive}
The Android system provides a wake lock mechanism to keep the CPU running in order to complete work. As long as we keep the CPU alive, we can collect the data over an extended period. Any applications can utilize wake-locks in their application; albeit, the documentation states that holding onto a wake lock for a longer period, shortens the device's battery time. Therefore, it is crucial to release the lock when the recording has terminated. Also, to use wake-locks the permission has to be added in the application's manifest file (see: Section \ref{impl:manifest}). Nidra utilizes the wake lock when the recording has started (inside of the \verb|RecordingFragment|) and are seen in following listing:

\begin{lstlisting}[language=json, caption={}, captionpos=b]
powerManager = (PowerManager) mContext.getSystemService(Context.POWER_SERVICE);
wakeLock = powerManager.newWakeLock(PowerManager.PARTIAL_WAKE_LOCK,
        "CESAR::collection");

wakeLock.acquire();
\end{lstlisting}
The lock is released when the activity is destroyed by terminating the recording process.


\subsubsection{Priority}
Process's lifecycle is not directly related with the host application; however, determined by the system which detects parts of applications that are running, how important they are to the user, and how much memory is available in the system. A process can be killed by the system to reclaim memory for other processes to take its place. However, there are specific measures to prolong the services run time. That is, to increase the process importance in the "process-hierarchy". By assigning a process to be a \textit{foreground process}, we can, for most cases, prevent the system from killing a process. In our case, the \verb|DSDService| which receives data packets from the data streams dispatching module. In the following listing, the code snippet for creating a foreground process is presented.

\begin{lstlisting}[language=json, caption={}, captionpos=b]
public void toForeground() {
    NotificationManager notificationManager = 
        (NotificationManager) this.getSystemService(Context.NOTIFICATION_SERVICE);
    NotificationCompat.Builder builder = null;
    if (android.os.Build.VERSION.SDK_INT >= android.os.Build.VERSION_CODES.O) {
        int importance = NotificationManager.IMPORTANCE_DEFAULT;
        NotificationChannel notificationChannel = new NotificationChannel("ID", "Name", importance);
        notificationManager.createNotificationChannel(notificationChannel);
        builder = new NotificationCompat.Builder(this, notificationChannel.getId());
    } else {
        builder = new NotificationCompat.Builder(this);
    }

    builder.setSmallIcon(R.drawable.ic_info_black_24dp);
    builder.setContentTitle("Nidra");
    builder.setTicker("Recording");
    builder.setContentText("Recording data");

    Intent i = new Intent(this, DSDService.class);
    i.setFlags(Intent.FLAG_ACTIVITY_CLEAR_TOP | Intent.FLAG_ACTIVITY_SINGLE_TOP);
    PendingIntent pi = PendingIntent.getActivity(this, 0, i, 0);
    builder.setContentIntent(pi);

    final Notification note = builder.build();

    startForeground(android.os.Process.myPid(), note);
}
\end{lstlisting}

\subsection{Android Manifest} \label{impl:manifest}
The Android Manifest describes the essential information about our application, such as the application components, permissions, and the package name. The application is constituted by application components, and each component contains metadata describing the application component. Below, we describe the some of permissions and a few application components of Nidra and the Flow sensor wrapper.

\subsubsection{Nidra}

The Nidra manifest file constitutes of three activities, one service, and one provider. The latter is used to share a record between applications. Providers enable other applications to access a file or data from Nidra. With the provider, a direct URI link obtained by the provider grants a more secure sharing of data between application. In the listing below, the attribute \verb|authorities| is the name that identifies the data offered by the provider (often distinguished by package name and postfix of "provider"). Also, the meta-data with the resource contains information with the path to the file in the respective application directory.  

\begin{lstlisting}[language=json, caption={}, captionpos=b]
<provider
    android:name="androidx.core.content.FileProvider"
    android:authorities="${applicationId}.provider"
    android:grantUriPermissions="true">
    <meta-data
        android:name="android.support.FILE_PROVIDER_PATHS"
        android:resource="@xml/provider_paths" />
</provider>
\end{lstlisting}

The permissions for Nidra are presented in the listing below, which includes the wake lock permissions, and the permissions to store data in external storage and the internal storage. The storage permissions are required in order to use the \verb|SharedPreference| method for storing, as well as the storage of files that will be accessed by other applications (during sharing of records).

\begin{lstlisting}[language=json, caption={}, captionpos=b]
<uses-permission android:name="android.permission.WRITE_EXTERNAL_STORAGE" />
<uses-permission android:name="android.permission.READ_EXTERNAL_STORAGE" />
<uses-permission android:name="android.permission.WAKE_LOCK" />
<uses-permission android:name="android.permission.READ_INTERNAL_STORAGE" />
\end{lstlisting}

\subsubsection{Flow Sensor Wrapper}

As for the Flow sensor wrapper, the application components structure are based on the driver created by Gjøby \cite{gjoby}. With an expectation of the activities and the BlueTooth service. In the listing below, the permissions of the application are shown. In order to leverage the Bluetooth LE protocol, we need the permissions of \textit{BLUETOOTH}, \textit{BLUETOOTH\_ADMIN}, \textit{ACCESS\_FINE\_LOCATION}. The latter permission is obligatory because it is used to list the available sensor source in the area. Without permission, Android does not present any list of available sensor sources. 

\begin{lstlisting}[language=json, caption={}, captionpos=b]
<uses-permission android:name="android.permission.BLUETOOTH"/>
<uses-permission android:name="android.permission.BLUETOOTH_ADMIN"/>
<uses-permission android:name="android.permission.WAKE_LOCK"/>
<uses-permission android:name="android.permission.ACCESS_FINE_LOCATION"/>
[...]
\end{lstlisting}


