\chapter{Implementation}

%\section{Architecture Pattern}
%
\begin{figure}
    \centering
    \includegraphics[scale=0.7]{images/MVVM.pdf}
    \caption{Entity Relationship Diagram}
    \label{fig:mvvm}
\end{figure}

The architectural pattern principle enhances the separation of \textit{graphical user interface} logic from the \textit{oparting system} interactions \cite{architecture}. The Model-View-ViewModel (hereafter: MVVM) is an architectural pattern which is well-integrated and incentived by Android. It has three components that consititute the principle:
\begin{itemize}
    \item \verb|Model|: represents the data and the business logic of the application. 
    \item \verb|ViewModel|: interacts with the model, and manages the state of the view.
    \item \verb|View|: handles and manages the user interface of the application.
\end{itemize}

In Figure \ref{fig:mvvm}, the interactions amongst the components are illustrated. The connection between the \verb|View| and \verb|ViewModel| occurs over a data binding connection, which enables the view to change automatically based on changes to the binding of the subscribed data \cite{mvvm}. In Android, the \verb|LiveData| is an obserable data holder that enables data binding, which allows components to observe for data changes. \verb|LiveData| respects the lifecycle of the application components (e.g., activities, fragments, or services), ensuring the \verb|LiveData| only updates the components that are in an active lifecycle state \cite{livedata}. Moreover, \verb|Android Room| provides set of components to facilitate the structure of the model component \cite{room}. More spesifically, it models a database and the entities (which are the tables in the database). In our application, we use this architectural pattern to interact and to model for our data entities (see Section X). 




\section{Android Components}

\begin{figure}
    \centering
    \includegraphics[scale=0.95]{images/Android_Components.pdf}
    \caption{Applications components}
    \label{fig:app_components}
\end{figure}

The CESAR project is introduced in the Background Chapther; to summarize, the data dispatching module discovers and connects with sensor wrappers (each sensor source has its own senor wrapper), and enables data acquesition to applications that subscribe to the data. As for this thesis, we operate with three different applications: Nidra, Data Stream Dispatching Module, and the sensor wrapper for Flow sensor kit. 

Figure \ref{fig:app_components} illustrates the Android components (i.e., activity, service and broadcast receivers) for each applications. All of the applications run in a separate process on a device. In order to perform remote procedure calls (RPC) to application components that run remotely, we can use the IPC mechanism. In Android there are two mechanisms to mechanisms to enable IPC: (1) \verb|Binder| enables a process to remotely invoke functions in another process; and (2) \verb|Intent| a message passing interface aollowing applications to send messages to each other.




\subsection{Inter-Process Communication: AIDL}
To perform IPC using Android Interface Definition Language (AIDL) [cite] we need to define a programming interface that both the client and the service agree upon. In order to communicate with processes, the data objects has to be decomposed into primitives that the operating system can understand, and marshall the objects across the boundary. The AIDL interface is defined in an \verb|.AIDL| file, and located in the \verb|src/| directory of the hosting service application (DSDM), and other applications that binds to the service (Nidra and sensor wrappers). It is important to have identical \verb|.AIDL| files across the applications, otherwise the system will not recognize it as the same interface. In Listing 5.1, the interface is based on the functionality the hosting service application exposes (DSDM). In Nidra, some of the functionality is utilized to enable recording. More spesifically, \verb|getPubishers()| method is used to get all of the sensors publishers (e.g., Bitalino provides multiple sensor capabilites ...), the \verb|Subscribe(...)| and \verb|Unsubscribe(...)| is used in order to subscribe and unsubscribe to a spesific sensor capability, and listing for events on the \verb|putJson(...)|, which is used by the sensor wrappers to send data collected to DSDM, and further sent to all subscribing applications (i.e., Nidra). 


\begin{lstlisting}[language=json, caption={My Caption}, captionpos=b]
// MainServiceConnection.aidl
package com.sensordroid;

interface MainServiceConnection {
    void putJson(in String json);
    int Subscribe(String capabilityId, int frequency, String componentPackageName, String componentClassName);
    int Unsubscribe(String capabilityId, String componentClassName);
    String Publish(String capabilityId, String type, String metric, String description);
    void Unpublish(String capabilityId, String key);
    List<String> getPublishers();
}
\end{lstlisting}



\section{Flow Sensor Kit Development}

\section{Implementation of Concerns}
In Section (ref) we conceptualized the tasks, by decomposing the tasks into components and discussing various techniques and design decisions for implementation. In this Section, we will realize the discussion by implementing the tasks in Android. 

\subsection{Recording}
The design choices for this concern is described in Section X. To give a brief overview; recording is the process of collecting and storing data received from sensors over an extended period. To enable a recording, we need to establish a connection with the available sensors and store the samples retrieved by the sensors on the device. In this thesis, we focus on collecting breathing data from the Flow sensor. 

We will use the data streams dispatching module (hereafter: DSDM), which manages sensor discovery and sensor establishment to supported sensor sources, in our case the Flow sensor.  Moreover, the DSDM facilitates an interface for data acquisition, and the communication between the DSDM and Nidra occurs over IPC using binder's. During recording, we will check for connectivity with the sensors to ensure the sensors are collecting data at an appropriate rate. At the end of the recording, we will store metadata related to the recording and finalize the recording process. 

The functionality of recording can be separated into three actions: (A) start recording; (B) stop recording; and (C) display recording statistics. In the following sections, we will review the steps that enable these actions. 

\subsubsection{Action A: Start Recording}
\begin{figure}
    \centering
    \includegraphics[scale=0.7]{images/Recording_ImpA.pdf}
    \caption{Implementation of recording functionality: (A) start recording}
    \label{fig:impl_recordingA}
\end{figure}

In Figure \ref{fig:impl_recordingA}, an illustration of the component interactions are shown. Action A is to start a recording by connecting and starting data aqusition with the use of DSDM, and to ensure persistent connectivity with the sensor sources. The steps and interactions for this action are: 

\begin{itemize}
    \item[A.1] The recording process starts by creating a new record entity (see: Section X) that is inserted into the SQLite database. An empty record has to be inserted into the database in order to associate new samples with the record (based on the record's id). 
    \item[A.2] Once the record is inserted into the storage, a unique identification (id) is returned. 
    \item[A.3] \verb|ConnectionHandler| is invoked in order to manage the establishment, connection, and disconnection of the IPC between Nidra and DSDM service. The code for establishing the connection is described in the following listing:
\begin{lstlisting}[language=json, caption={Code snippet for connecting with the DSDM (MainServiceConnection is the AIDL file as discussed in Section X)}, captionpos=b]
    Intent intent = new Intent(MainServiceConnection.class.getName());
    intent.setAction("com.sensordroid.ADD_DRIVER");
    intent.setPackage("com.sensordroid");
    context.bindService(intent, serviceCon, Service.BIND_AUTO_CREATE);
\end{lstlisting}
    
    \item[A.4] If the service is offline when binding, the flag \verb|Service.BIND_AUTO_CREATE| will ensure for starting the service. \verb|BindService| allows components to send requests, receive responses, and perform inter-process communication (IPC) based on the interface provided by the host service (DSDM). 
    \item[A.5] Once the service is bound, we can proceed to communicate with the DSDM service. 
       \item[A.6] The \verb|ConnectionHandler| proceeds to initialize the connection with the sensor through the DSDM.  A request to the DSDM for available publishers with \verb|getPublishers()| is made, to retrieve all available sensor publishers connected to the DSDM. Occasionally, the DSDM uses extended time to discover all of the active sensors connected to the device; therefore, we have an interval that checks whether DSDM has any available sensors connected. 
    \item[A.7] Moving on,  a request to the DSDM to \verb|Subscribe| to a sensor is made. We specify that we want the Flow sensor in the \verb|Subscribe| method, in addition, a reference to the package name (Nidra) and a service object (\verb|DSDService|). The service object is where all of the data packets from the subscribed Flow sensor is received (on the \verb|putJson()| method).   
    \item[A.8] Also, a callback to \verb|RecordingFragment| with information of the sensor source (i.e., Flow) that we subscribe to is made, in order to display the information on the user's screen. 
    \item[A.9] The recording has now started, and a timer to measure the time spent on the recording is started. The \verb|ConnectivityHandler| is also initialized, which actively checks that the samples arrive within a specified time frame (as discussed in the design, a frame of 10 seconds that increases throughout the recording). The \verb|ConnectivityHandler| is implemented with a \verb|Handler| with a \verb|PostDelay| that counts down. Upon a sample arrival, the timer is reset. 
    \item[A.10] Periodically, the DSDM receives samples from the subscribed sensor. DSDM forwards the sample from the sensor to the service object (\verb|DSDService|) on the \verb|putJson()| method. The DSDService uses a \verb|LocalBroadcastManger| to send the data packet to the \verb|RecordingFragment|.  
    \item[A.11] \verb|RecordingFragment| listens for the events on the local broadcast receiver. Upon an event, the data that is received from the sensor is inserted as a new sample entity with the current record's id as an association. 
    \item[A.12] Recalling the functionality from step A.9; if the event for the \verb|PostDelay| is triggered, it is equivalent to a sample not being acquired from the sensor. Therefore, we try to reconnect with the subscribed sensor by disconnecting with the DSDM (which will close the connection with the sensor source), followed up by a reconnection with the DSDM and subscribing to the same sensor source. Most of the times, the process of reconnecting works instantaneously; however, some times the sensor might require to be reconnected with several times. 
\end{itemize}

\subsubsection{Action B: Stop Recording}
\begin{figure}
    \centering
    \includegraphics[scale=0.7]{images/Recording_ImpB.pdf}
    \caption{Implementation of recording functionality: (B) stop recording}
    \label{fig:impl_recordingB}
\end{figure}

In Figure \ref{fig:impl_recordingB}, an illustration of the component interactions are shown. Action B is based on user input to stop the recording process. To the user, the recording has terminated, and the user is presented with a screen to provide extra information regarding the recording (e.g., title, description, and rating). For the application, it has to unsubscribe from the connected sensor sources (i.e., Flow), and disconnect the connectivity with the DSDM. Also, transition the user to the screen where the user can provide extra information that will be stored alongside the record.

\begin{itemize}
    \item[B.1] The user decides when to stop a recording with a press of a button. The event to stop the recording is sent to the \verb|ConnectionHandler|.
    \item[B.2] A call to the \verb|Unsubscribe()| method that contains the service object (i.e., DSDService) and the identification of the sensor source (e.g., Flow) is sent to DSDM. The DSDM has to ensure unsubscribing the sensor from a specific application and disconnect the IPC between the application. If there are no subscribing applications to the specific sensor source, the DSDM will signal the sensor to stop sampling and disconnect with the sensor. 
    \item[B.3] The IPC connection between Nidra and DSDM is discontinued by unbinding the service. 
    \item[B.4] The estimated time of recording is calculated, and a transition from \verb|RecordingFragment| to \verb|StoreFragment| is made to finalize the recording with extra information (e.g., title, description, and rating). 
    \item[B.5] The \verb|StoreFragment| uses the record identification retrieved on recording (A.1) in order to enrich the record with statistics and user-defined metadata. The statistics are the monitoring time, number of samples during recording, and retrieving the current state user biometrical data and storing it in the record. The user-defined metadata are the title of the recording, a description enabling the user to add a note to the recording, and a rating between 1--5 (to give a rating on how the recording felt). 
    \item[B.6] The modified record is updated in the database, and the user is transitioned to the \verb|MainActivity|.
\end{itemize}

\subsubsection{Action C: Display Recording Statistics}
During a recording, the user can view the statistics for the recording. More specifically, the user can see the available connected sensors and a graph of the breathing data in real-time. In Figure \ref{fig:impl_recordingC}, an illustration of the component interactions are shown, and the steps and interaction for this action are: 

\begin{figure}
    \centering
    \includegraphics[scale=0.7]{images/Recording_ImpC.pdf}
    \caption{Implementation of recording functionality (C)}
    \label{fig:impl_recordingC}
\end{figure}

\begin{itemize}
    \item[C.1] The data is graphically represented as an intractable time-series graph. By using the Graph library [CITE], we can in similarities to the implementation of the analytics concern (see Section X), implement a graph to illustrate the respiration data to the user.
    \item[C.2] In addition to the time-series graph, we have a list of publishers (e.g., Flow) that we acquired in the begging of the recording. As such, a list of publishers is sent to \verb|SensorAdapter|. 
    \item[C.3] The \verb|SensorAdapter| populates a view with the connected sensor to the user.
\end{itemize}


\subsection{Sharing}

Sharing enables users to transmitt records across application over a media. The functionality of sharing is separated into two concerns, namely importing and exporting records. Hence, the actions for sharing are separated into: (A) exporting one or all records; and (B) import a record from the device. 

Before a user can take these actions, the records from the database have to be presented. The \verb|Feed Fragment| contains a \verb|RecyclerView| which populates the records into inside the \verb|Feed Adapter| (steps: 1-4). The adapter contains all the interactions and the event handling (i.e., button event listener for exporting) for a single record. 

In this Subsection, we will take a look into the steps that are taken to enable the actions:

\begin{figure}
    \centering
    \includegraphics[scale=0.6]{images/Sharing_ImpA.png}
    \caption{Implementation of sharing functionality (A): Exporting one or all Records}
    \label{fig:impl_sharingA}
\end{figure}

\subsubsection{Action A: Exporting one or all Records}
In Figure \ref{fig:impl_sharingA}, an illustration of the steps to export one single recording is shown. However, the \verb|Feed Fragment| has an option to export all record; therefore, by disregarding the first step (A.1), the same structure applies to export all records. In essence, exporting consists of bundling the records and corresponding samples into a formated file, and prompting the user with options to select a media (e.g., mail) for transmittion. The steps can be narrowed down to: 

\begin{itemize}
    \item[A.1] Upon an event for exporting a selected record in \verb|Feed Adapter|, the record information is sent to the \verb|Feed Fragment| through the callback reference (\verb|onRecordAnalyticsClick|) between these components. The record information will be used to determine the corresponding samples for the record.
    \item[A.2] The \verb|Feed Fragment| delegates record information to the \verb|export| method inside of the \verb|Export| class. The class is responsible for enabling exportation. 
    \item[A.3] An operation to retrieve all samples related to the record with the use of the \verb|SampleViewModel| is done. 
    \item[A.4] The \verb|export| method retrieves all of the samples related to the record. Next, the record and the samples are encoded into an exportable JSON format (Ref: Data Format). To enable the sharing interface provided by Android, the content has to be stored on the device. Thus, the encoded data is written into a file on the device, with a filename of \verb|record_(current_date).json|, and the next component uses the reference to the file location. 
    \item[A.5] The encoded file is retrieved with the use of \verb|FileProvider| (facilitates secure sharing of files [ref]). The code for this step are
\begin{lstlisting}[language=json, caption={My Caption}, captionpos=b]
static void shareFileIntent(Activity a, File file) {

    Uri fileUri = FileProvider.getUriForFile(a.getApplicationContext(), a.getApplicationContext().getPackageName() + ".provider", file);

    Intent iShareFile = new Intent(Intent.ACTION_SEND);
    iShareFile.setType("text/*");
    iShareFile.putExtra(
        Intent.EXTRA_SUBJECT, "Share Records");
    iShareFile.putExtra(Intent.EXTRA_STREAM, fileUri);
    ...

    a.startActivity(
        Intent.createChooser(iShareFile, "Share Via"));
}

\end{lstlisting}

    \item[A.6] The user is displayed with a popup interface with several options to share the file over a media. An illustration of the layout can be found in Section Representation. 


\end{itemize}


\begin{figure}
    \centering
    \includegraphics[scale=0.6]{images/Sharing_ImpB.png}
    \caption{Implementation of sharing functionality (B)}
    \label{fig:impl_sharingB}
\end{figure}

\subsubsection{Action B: Import a Record from the Device}
In Figure \ref{fig:impl_sharingB}, an illustration of importing a record from the device is shown. Importing conists of locating the formated file (the user has to obtain the file and store it on the device on beforehand), parsing the content in the file, and storing the data respective to the users database. The steps can be narrowed down to:

\begin{itemize}
    \item[B.1] The user requests to view the import record interface. The interface is provided by Android, and allows the user to select particular kind of data on the device (ref). The code for this action is:
\begin{lstlisting}[language=json, caption={My Caption}, captionpos=b]
private void importRecords() {
    Intent intent = new Intent(Intent.ACTION_GET_CONTENT);
    intent.setType("*/*");
    startActivityForResult(intent, 1);
}

\end{lstlisting}
    \item[B.2] Once the user has selected the desired file, the method \verb|onActivityResult| inside of \verb|Feed Fragment| is called, and location of the selected file can be located. 
    \item[B.3] The file location is an obscured path to the file on the device; thus, parsing the path with the use of \verb|Cursor| method has to be done. After the absolute path is found, the data is decoded accordingly to the data format, and the records are sent back to \verb|Feed Fragment|.
    \item[B.4] The necessary record information and the samples are extracted from the decoded data, and are inserted into the users database. 
\end{itemize}

\subsection{Modules}
Modules are standalone application, that provides data enrichment and extended functionality to the application. The modules leverages the records and samples to analyze, evalutate or detect sleeping disorders. In order to add and launch a module in Nidra, we need the modules package name. The package name and the name of the module-application can be obtained in Android. Thus, the actions to enable modules in the application are: (A) add a module; and (B) launch a module. 

Before a user can take these actions, the records from the database have to be presented. The \verb|Feed Fragment| contains a \verb|RecyclerView| which populates the records into inside the \verb|Feed Adapter| (steps: 1-4). The adapter contains all the interactions and the event handling (i.e., button event listener for exporting) for a single record. 

In this Subsection, we will take a look into the steps that are taken to enable the actions.

\subsubsection{Action A: Add a Module}
In order to add a new module, the user has to install the module-application on the device on beforehand. By listing through the installed application on the device, the user can select the desired module to be added in Nidra. In Figure \ref{fig:impl_modulesA}, an illustration of adding a module is shown, and the steps can be narrowed down to:

\begin{figure}
    \centering
    \includegraphics[scale=0.7]{images/Module_ImpA.pdf}
    \caption{Implementation of module functionality(A): Add a Module}
    \label{fig:impl_modulesA}
\end{figure}

\begin{itemize}
    \item[A.1] Upon an event for adding a new module in \verb|Modules Adapter|, the \verb|Feed Fragment| is notified through the callback reference (\verb|onNewModuleClick|) between these components.
    \item[A.2] The \verb|Modules Fragment| lauches a custome Android dialog, which will list all of the installed application on the device. 
    \item[A.3] The \verb|Apps Adapter| will fetch all of the application that is not a system package, already installed module, or the current application (Nidra). Next, the the adapter for the dialog will be populated with the eligible applications. 
    \item[A.4] Once the user has selected the desired module-application, an event to the \verb|Modules Fragment| through the callback reference \verb|onAppItemClick| between these components are made. The callback contains an object with the \verb|PackageInfo| for the selected module-application.
    \item[A.5] The dialog is dismissed, and the application name and packagename are extracted from the \verb|PackageInfo| for the selected module-application. 
    \item[A.6] Furthermore, the acquired information is stored in our database for modules through the DAO interface. 
\end{itemize}

\subsubsection{Action B: Launch a Module}
A module is launched in a seperate process, due to Android prohibits launching for other applications inside of an application. All added modules are listed and presented to the user in a separate screen, and on launch of a module, all of the data that Nidra obtains from recordings, are encoded into a JSON format and bundled with the launch of the module. In Figure \ref{fig:impl_modulesB}, an illustration of launching a module, and the steps can be narrowed down to:

\begin{figure}
    \centering
    \includegraphics[scale=0.7]{images/Module_ImpB.pdf}
    \caption{Implementation of module functionality(B): Launch a Module}
    \label{fig:impl_modulesB}
\end{figure}

\begin{itemize}
    \item[B.1] Upon an event for launching a module in \verb|Module Adapter|, the packagename of the module is sent to the \verb|Modules Fragment| through the callback reference (\verb|onLaunchModuleClick|) between these components. The packagename  will be used to launch the module-application.
    \item[B.2] All of the records and samples on the device for the user, is bundled and formated into a JSON, and launched:
\begin{lstlisting}[language=json, caption={My Caption}, captionpos=b]
public void onLaunchModuleClick(String packageName) {
    Intent moduleApplication = context.getPackageManager().getLaunchIntentForPackage(packageName);

    if (moduleApplication == null) return;

    String data = formatAllRecordsToJSON();

    Bundle bundle = new Bundle();
    bundle.putString("data", data);

    moduleApplication.putExtras(bundle);

    startActivity(moduleApplication);
}
\end{lstlisting}

    \item[B.3] The activity uses the data provided in the \verb|Intent| that includes the packagename (the name of the module-application to determine the correct application).
    \item[B.4] The selected module is then launched, and presented to the user. The user can at anytime press the back button, to return to Nidra.  
\end{itemize}

%\begin{figure}
%    \centering
%    \includegraphics[scale=0.6]{images/Module_ImpA.png}
%    \caption{Implementation of module functionality(A)}
%    \label{fig:impl_modulesA}
%\end{figure}



%\subsubsection{Data Exchange Implementation}


\begin{figure}
    \centering
    \includegraphics[scale=0.6]{images/Anal_Imp.png}
    \caption{Implementation of analytics functionality (A): Display a Graph for a Single Record}
    \label{fig:impl_analytics}
\end{figure}

\subsection{Analytics}
Analytics is the part of illustrating and analyzing the records. In Nidra, the analytics part of the implementation is limited to a time-series graph for a single record. However, there are possibilities of extending the \verb|Analytics Fragment| with other graphs based on the current structure. The current action for analytics is A) display a graph for a single record to the user. 

Similar to sharing, the records from the database have to be presented. The \verb|Feed Fragment| contains a \verb|RecyclerView| which populates the records into inside the \verb|Feed Adapter| (steps: 1-4). The adapter contains all the interactions and the event handling (i.e., button event listener for analytics) for a single record. 

In this Subsection, we will take a look into the steps that are taken to enable the action.

\subsubsection{Action A: Display a Graph for a Single Record}
Nidra provides a simple time-series graph of respiration data obtained during the recording. The graph data is plotted into a Library, which enables interactions (e.g., zoom and scrolling) on the data. The X-axis is the respiration value based on the Y-axis time of sampling. In Figure \ref{fig:impl_analytics}, an illustration of displaying a graph shown. The steps can be narrowed down to:

\begin{itemize}
    \item[A.1] Upon an event for analytics on a selected record in \verb|Feed Adapter|, the record information is sent to the \verb|Feed Fragment| through the callback reference (\verb|onRecordAnalyticsClick|) between these components. The record information will be used to determine the corresponding samples for the record.
    \item[A.2] A new instance of the \verb|Analytics Fragment| is created, and an transition from the \verb|Feed Fragment| to the \verb|Analytics Fragment| is made. Alongside, the record information is transmitted.
    \item[A.3] An operation to retrieve all samples related to the record with the use of the \verb|SampleViewModel| is done. 
    \item[A.4] The \verb|Analytics Fragment| retrieves all of the samples related to the record. The samples has to be structured according to the graph library to display an interactive time-series graph. (GraphView (ref)).
    \item[A.5] Each sample has to be extracted from the sample-data, acccordingly to the sensor data structure.
    \item[A.6] The sample value is returned, and inserted into an array over datapoints used in the graph. 
    \item[A.7] The use is presented with a graph, which is interactable. The Y-axis has the sample value on given time (in HH:MM:SS) on the X-axis. The graph library enables interactions (e.g., zooming and scrolling) the user can do, to gain a better understanding of the recording. 
\end{itemize}



\subsection{Storage}
\begin{figure}
    \centering
    \includegraphics[scale=0.60]{images/Storage_Imp.pdf}
    \caption{Entity Relationship Diagram}
    \label{fig:impl_storage}
\end{figure}


Storage facilities persistent data which remain available after application termination. In Nidra, there are four individual data entities (i.e., record, sample, module, and user). In the Subsection about Data Entities, we discussed the design choices of the individual data entity. The standard CRUD operations on each data entities are: insert, update, and delete. Also, the user has an operation to retrieve the biometrical data, module and record have an operation to retrieve all of the entries in the database, and samples have an operation to retrieve all of the entries corresponding to a record. 

Android Room provides an abstract layer over SQLite to enable easy database access [Cite]. In Figure \ref{fig:impl_storage}, the flow for accessing and retrieving the data from the database based on the Android Room architecture is shown and can be described as:

\begin{itemize}
	\item[1] Each data entity has a \verb|ViewModel| where all of the CRUD operation (e.g., insert, update, delete, or retrieve) goes through.  A view model is designed to store and manage UI-related data in a conscious way, that allows data to be persistent through configuration changes (e.g., screen rotations) [CITE]. 
 	\item[2a] The predefined operations point to the repository. Repository modules handle data operations and provide an API  which makes data access easy. A repository is a mediator between different data sources (e.g., database, web services, and cache) [CITE]. In Nidra, the only data source is the database, but repository facilities future data sources. 
	\item[2b] The storage of the user is not in the database; however, in a \verb|SharedPreference| on the device. Shared preference points to a file containing key-value pairs and provides methods to read and write. The location of the user's shared preference is \verb|no.uio.cesar.user_storage|. 
	\item[3] Each data entity (disregarding user) has a data access object (DAO), where the SQL operations to the database are defined. 
	\item[4] Based on the operation, the data is accessed and retrieved from the SQLite database.
\end{itemize}


\subsection{Presentation}
In this Subsection we will present the user-interface (UI) based on the functionality of recording, sharing, module and analytics. The interface is developed based on the design descisions of the application, as well as creating a user-interface that is simple and efficent for a user to interact with. We try to limit the actions the user can take on a screen, to make the application simpler to understand and comprehend. 

\subsubsection{UI: Recording}
\begin{figure}
    \centering
    \includegraphics[scale=0.26]{images/Recording_img.pdf}
    \caption{The recording screen displayed to the user; with the screen: (A) during a recording, (B) real-time analytics, and (C) finalizing the recording}
    \label{fig:screen_recording}
\end{figure}

Figure \ref{fig:screen_recording} shows the screen for: (A) during a recording, (B) the real-time analytics, and (C) finalizing the recording. 

\begin{itemize}
    \item[A] The screen has a ripple effect to indicate the state of the recording to the user. There are two types of ripple colours; a blue ripple effect for samples acquistion, and a grey ripple effect if the sensor has disconnected. The ripple effect is only active if the screen is turned on, in order to perserve battery life. During disconnects between the sensor and the device, the user provide no extra input to resolve the issue. However, if the numbers of attempts is reached, the user is presented with the finalizing screen (C). Also, this screen display the elapsed recording time, and has a visible button to stop the recording. 
    \item[B] The user can extend the interface for viewing statistics regarding the recording session. Currently, the interface lists all of the available sensor sources, also a real-time interactable graph of respiration.
    \item[C] The finalzing screen allows users to specify the title and description of recording, to make it distingushable, in addition to giving a feedback on recording so the doctors/researchers can review it. Also, the user can give a rating between 1--5 (where 1 is bad, and 5 is good) to rate the sleeping.

\end{itemize}

\subsubsection{UI: Sharing}
\begin{figure}
    \centering
    \includegraphics[scale=0.26]{images/Sharing_img.pdf}
    \caption{The sharing screen displayed to the user}
    \label{fig:screen_sharing}
\end{figure}

Figure \ref{fig:screen_sharing} shows the screen for: (A) option to import or export, (B) the media selection for exporting, and (C) the file selection for importing.

\begin{itemize}
    \item[A] The feed screen is where all of the records are presented to the user. The user can choose to export one single record, or export all. 
    \item[B] By pressing export all, an overlay with a Android provided sharing screen is presented, where the user can choose a media to export the file on. 
    \item[C] By pressing import, an overlay with downloaded files on the users device is presented. The user can press on the desired file, and the file will be parsed and added to the users collection of records.  
\end{itemize}

\subsubsection{UI: Modules}
\begin{figure}
    \centering
    \includegraphics[scale=0.26]{images/Modules_img.pdf}
    \caption{The module screen displayed to the user}
    \label{fig:screen_modules}
\end{figure}

Figure \ref{fig:screen_modules} shows the screen for: (A) list without any modules, (B) list of installed application, and (C) list with modules. 

\begin{itemize}
    \item[A] The modules screen is shows all of the added modules to the users.
    \item[B] The user can press the \textit{add new module} button, in order to be presented with a list of all installed applications. 
    \item[C] Once the user has selected a module-application to be added in Nidra, the list of modules are updated and presented to the user.
\end{itemize}

\subsubsection{UI: Analytics}
Figure \ref{fig:screen_analytics} shows the screen for: (A) single record in the list, and (B) the analytics for the record

\begin{itemize}
    \item[A] The user can expend records in order to view more functionality and statistics of the record. One of the functionalities are to view the analytics of the record.
    \item[B] An interactable graph is presented to the users, and the graph is populated with the samples obtained from the recording for selected records. The graph shows the respiration value (Y-axis) on given time (X-axis) of sampling. 
\end{itemize}

\begin{figure}
    \centering
    \includegraphics[scale=0.26]{images/Analytics_img.pdf}
    \caption{The analytics screen displayed to the user}
    \label{fig:screen_analytics}
\end{figure}


