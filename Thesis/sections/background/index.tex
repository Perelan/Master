\chapter{Background}

\section{The CESAR Project}
The goal of the CESAR project is to reduce the threshold to perform a clinical diagnosis of obstructive sleep apnea (OSA) and to reduce the time to diagnose the disorder \cite{cesarinfo}. Obstructive sleep apnea is a common sleeping disorder which affects the natural breathing cycle by reducing respiration or all airflow. As a consequence, OSA can lead to serious health implications, and in some cases, death through suffocation.  Moreover, the project aims to use low-cost sensor kit to prototype applications using physiological signals related to heart rate, brain activity, the oxygen level in the blood to monitor sleep and breathing-related illnesses \cite{cesar}.

As of now, the development in the project facilities for data acquisition from various sensor sources and forwarding of the data packets to subscribed applications. Section \ref{background:svein} and Section \ref{background:daniel} is a summary and results of the development on the project. The previous work has incorporated support for a few sensor sources (e.g., BITalino) in order to collect physiological signals. With the support for more sensor sources allows for more precise detection and analysis of the disorder. In Section \ref{bg:flow}, a new affordable sensor source is introduced.

As an overview, Figure \ref{fig:parts} is an illustration of the structure for the CESAR project. The system is divided into three parts: data acquisition part, data streams dispatching part, application part. The first part enables developers to create sensor wrappers that connect with sensor sources through different Link Layer technology (e.g., Ethernet, USB, Bluetooth, WiFi, ANT+, and ZigBee). The second part manages the discovery and collection of these sensor wrappers and package distribution to subscribing applications. The third part uses the interface that the second part provides in order to collect data from the desired sensor sources.

\begin{figure}
    \centering
    \includegraphics[width=0.5\textwidth]{images/parts.png}
    \caption{Structure of the CESAR project, separating functionality into three independent parts \cite{daniel}.}
    \label{fig:parts}
\end{figure}


\subsection{Extensible Data Acquisition Tool} \label{background:svein}
In the thesis "Extensible data acquisition tool for Android" by Gjøby \cite{gjoby} a proposition of a \textit{data acquisition system} for Android is presented. The thesis proposes a system that hides the low-level sensor-specific details into two components, \textit{providers} and \textit{sensor wrappers}. The provider is responsible for the functionality that is common for all data sources (e.g., starting and stopping the data acquisition), while the sensor wrapper is responsible for the data source specific functionality (e.g., communicating with the data source).

The thesis solves the difficulties around creating an extensible data acquisition tool, connecting new and existing sensors, and finding a common interface. The problem statement of the thesis addresses the following concerns regarding sensors:
\begin{itemize}
    \item \textit{Common abstraction/interface for the interchanged data}\\ 
    Sensor platform manufacturers have their low-level protocol to support the functionality of their product. Typically, the manufacturers provide software development kits (SDKs) to hide the low-level protocols so third-party developers can be easier; however, both the low-level protocol and the SDKs are not standardized. Thus, for each sensor, there might be different commands and methods.
    \item \textit{Various Link Layer technologies} \\ 
    Each sensor might use different Link Layer technology (i.e., Ethernet, USB, BlueTooth, WiFi, ANT+, and ZigBee), which means establishing a connection between a device and a sensor might differ. For instance, BlueTooth devices need to be paired, while devices on the WiFi can address each other without any pairing. 
    \item \textit{Reusability of sensor code}\\
    Applications that implement support for the low-level protocol of a sensor type can not be shared between different applications. Thus, introducing duplicate work and code if multiple application wish to use the same sensor type. A framework that isolates the sensor that applications can use, might make it easier for application to utilize the collected data. In addition, isolating the sensors into modules improves the robustness and quality of the implementation. 
    
\end{itemize}

In the thesis, the goal is to develop an extensible system, which enables applications to collect data from various external and built-in sensors through one common interface. The solution around an extensible system is to have the core of the application unchangeable when adding support for a new data source, regardless of the Link Layer technology and communication protocol used by the data source. Making all the data sources behave as the same, is a naive solution to the problem. However, separating the software into two different components, a \textit{provider} component and a \textit{sensor wrapper} component, enables the reuse of functionality that is common amongst the data sources. 

The sensor wrapper application is tailored to suit the Link Layer technology and data exchange protocol of one particular data source. Additionally, responsible for connectivity and communication with the data source. The provider application is responsible for managing the sensor wrappers---starting and stopping the data acquisition---and processing the data received from the sensor wrapper application. Thus, everything that is independents of the data source should be a part of the provider application. With this type of solution, the possibility to reuse the sensor wrapper application for different provider applications is made possible. However, there are some overheads with this solution. Mostly, the inter-process communication that might be costly and increase the complexity of the code. Nonetheless, the flexibility and extensibility gained by separating the functionalists out weights the cost.

When a connection is established with the provider application, a package of metrics and data type (all the data does not change during the acquisition as metadata) is sent describing the context of the data collected. The metadata is necessary because different sensors might sample data in different environments, and some applications are depending on recognizing the environment of the data acquisition. Therefore, it is critical to know what data values are measured. Consequently, exposing sensors and data channels through one common interface requires a field of metadata which can be used to: (1) \textit{distinguish} sensor wrapper and data channel the data originated from; (2) determine the \textit{capabilities} of the sensors (i.e., EEG, ECG, LUX); (3) determine the \textit{unit} the data is represented in (i.e., for temperature, Celsius or Fahrenheit); (4) describing the data channel (i.e., placement of the sensor); and (5) a timestamp of when the data was sampled. 

To summarize, the task of a \textit{sensor wrapper} is to establish a connection to and collect data from precisely one specified data source, and to send the collected data to the \textit{provider application} that is listening for it. A data source (e.g., BiTalion) can have support for multiple sensor attachments (defined as data channel in the thesis), although, only on sensor wrapper is necessary for each data source and their data channels. Each sensor wrapper is tailored to adapt to the data source's Link Layer technology and the communication protocol of a respective data source. Upon activation by a provider application, the data is collected by the sensor wrapper, and pushed to the provider application in a JSON-format. An illustration of the structure is found in Figure \ref{fig:provider_SW}.

\begin{figure}
    \centering
    \includegraphics[width=0.5\textwidth]{images/provider_SW.png}
    \caption{Sharing the collected data between multiple applications \cite{gjoby}}
    \label{fig:provider_SW}
\end{figure}
\section{Extensible Data Stream Dispatching Tool}
The extensible data acquisition tool developed by Gjøby leaves some space for improvements. Such improvements are discussed in the thesis "Extensible data streams dispatching tool for Android" by Daniel Bugajski \cite{daniel}. Bugajski analyses the potential improvements of the data acquisition tools, which can be extracted into:
\begin{itemize}
    \item \textit{Lack of reusability} - only the components that have started the collection can receive the data, and no other components can access the collected data.
    \item \textit{Lack of sharing} - components that perform specific analysis on the collected data in real-time, have no way of share the results of the analysis such that other components can use them. 
    \item \textit{Lack of tuning} - it is not allowed to change the frequency of collection after the start. Thus, the user has to stop the collection and manually change the frequency of the sensor and then restart the collection. 
    \item \textit{Lack of customization} - the set of channels can not be changed during a collection, and the collector receives data from all channels even if it needs only one of them. Thus, the data packet size and resource usage become larger than necessary.
\end{itemize}

In the thesis, the modularity of the architecture is improved by extracting the functional requirement of the , and determining the responsibility of each element by. First, finding a model of all available data channels should be implemented. Then, developing a mechanism for cloning a data packet to allow reusing of data across modules. Finally, letting the modules have support for choosing channels they want to receive data from and publish their own data to. In the model, these components are distinguished as: 
\begin{enumerate}
    \item \textit{sensor-capability model} - is a representation of all distinct data types and contains all information about the channel. A sensor board usually reads and sends different type of data to a mobile device. Thus, this module is used to control every available data type, such that they can be access from the application part at anytime.
    \item \textit{demultiplexer} (DMUX) - is a data cloner, that receives data packets from one input (i.e., from one channel), and duplicates the data a number of times - based on number of subscribers.
    \item \textit{publish-subscribe} mechanism - is an interface responsible for providing  possibilties of becoming a subscriber or a publisher, in addition to be able to terminate these statuses. Additionally, every module from the application will be able to see all capabilities represented by this component, enabling the option to choose a frequency the data should be collected with.
\end{enumerate}

\begin{figure}
    \centering
    \includegraphics[width=0.65\textwidth]{images/demux.png}
    \caption{Sharing the collected data between multiple applications \cite{daniel}}
    \label{fig:demux}
\end{figure}

The combination of how these modules cooperate and communicate with each other, affects the modularity and performance of the architecture. In the thesis, there are various proposed solutions. The naive solution, were to fit all of the elements into each respective sensor wrapper, thus, prioritizing the performance and low resource usage, but making it impossible to distinguish data type from two sensor wrapper with same data type. This is optimal for the cases where collection only occurs on one sensor board. An improved solution, includes to place the demux between the application part and the wrapper layer and inserting the remaining elements in the respective application part. This solution resolves the overload of sensor wrappers performing other task besides collecting data, thus, the wrappers are untouched and they send the data to the demux. However, there are several issues with this solution, e.g., due to various obstacles such as: (1) every application module has to configure its own sensor-capability model; (2) filter requested data packets from all the channels; (3) and deal with the collection speed on its own.

Addressing these issues leads us to the final architecture, which is presented in Figure \ref{fig:demux}, and meets the demands identified in the requirements. In this solution, all elements are placed between the application part and the wrapper layer, forming the data stream dispatching module. The sensor wrapper connects directly to the data streams dispatching module, the module discovers all installed wrappers and populates the sensor-capability model with the data types from all installed sensors. By this, all applications can access a shared sensor-capability mode. A publish-subscribe mechanism enables application modules to subscribe to any capabilities with a preferred sampling rate. Correspondingly, an application module can publish data to other applications through the same interface. The demultiplexing element creates for each subscriber a copy of the data packet.

% [Finish this Section] -> Why is it extensible
These three elements together establish \textit{the data stream dispatching module}. The final architecture has a couple of advantages, such as it is very extensible due to its maintainability. For instance, all communication with other layers occurs through one interface, this way, new instances can be added at any time, without the need of modifying large parts of the system. The system is also very effective, by the means of packets are immediately sent to the application on request (without any buffers), and packets are only sent to the application requesting them, resulting in less resource and power usage, and more battery.

\subsection{Flow Sensor Kit}
Flow is an activity sensor created by SweetZpot Inc., initially designed for measuring breathing and heart rate during activities. The sensor measures breathing per minute, in correlation to the heart rate, for optimal activity measurement. As stated on their page: \textit{"usually, at rest, your breathing varies between 6 and 8 liters per minute. During sleep, breathing can be as low as 3 liters per minute and can reach 160 liters per minute and above during high-intensity athletic activity"} \cite{flow}. Thus, this sensor could be suiting for measuring sleeping problems in the project.

The sensor is a  strap-on placed beneath the chest. It weighs 27 grams and dimensions of 77x43x17mm. Additionally, it is equipped with a 3V Lithium battery, with an estimated battery life of one year (with 7 hours a weekly usage). The sensor can be connected with BlueTooth technology, making it possible to connect with a mobile device.



\section{Android OS}
Android is an operating system (OS) developed by Google Inc.


\subsection{Android Architecture}
The Android platform is an open-source and Linux-based software stack, containing six major components \cite{androidplatform}: 

\begin{figure}
    \centering
    \includegraphics[scale=0.85]{images/Android.pdf}
    \caption{Recording}
    \label{fig:hta_recording}
\end{figure}


\begin{itemize}
    \item \textbf{Applications}: Android provides a core set of applications (e.g., SMS, Mail, and browser) pre-installed on the device. There are support for installing third-party applications, which allows users to install applications developed by external vendors. A user is not bound to use the pre-installed applications for a service (e.g., SMS), and can choose the desired applications for a service. Also, third-party applications can invoke the functionality of the core applications (e.g., SMS), instead of developing the functionality from scratch. 
    \item \textbf{Android Framework}: Is the building blocks to create Android applications by utilizing the core, all exposed through an API. The API enables reuse of core, modular system components, and services; briefly characterized as \textit{View System}: to build the user interface pre-defined components (e.g., lists, grids, and buttons); \textit{Resource Manager}: provides access to resources (e.g., strings, graphics and layout files); \textit{Notification Managers}: allows applications to show custome notifications in the status bar; \textit{Activity Manager}: manages lifecycle of the application; and \textit{Content Providers}: enables applications to access data from other applications.
    \item \textbf{Android Runtime}: Applications run its own process and has its own instance of the Android Runtime (ART). ART is designed to run on multiple virtual machines by executing DEX (Dalvik Executable format) files, which is a bytecode specifcally for Android to optimize memory footprint. Some of the features that ART provides are ahead-of-time (AOT) and just-in-time (JIT) compilation, garbage collection (GC), and debugging support (e.g., sampling profiler, diagnostic exceptions and crash reporting).
    \item \textbf{Native Libraries}: Most of the core Android components and services native code, that requires native libraries, is written in C or C++. The Android platform exposes Java APIs to some of the functionality of the native libraries (with Android NDK).   
    \item \textbf{Hardware Abstract Layer}: Provides an interface to expose hardware capabilities to the Java API framwork. Hardware Abstract Layer (HAL) consists of multiple library modules that implements an interface for specific hardware components (e.g., camera, or bluetooth module).
    \item \textbf{Linux Kernel}: is the fundation of the Android platform. The ART relies on the functionality from the Linux kernal, such as threading and low-level memory management. The Linux kernel provides drivers to services (e.g., Bluetooth, Wifi, and IPC), and incorporates a component for power management. 
\end{itemize} 


\subsection{Application Components}
Application components consists of four core components that are the building blocks of an Android application \cite{appfundamentals}. This section introduces these components; \verb|Activites|, \verb|Services|, \verb|BroadcastReceivers|, and \verb|Content Providers|. The activity is responsible for interactions with the user, services is a component that perform (long-running) tasks in the background, broadcast receivers handles broadcast messages from application components, and content providers manages shared set of application data. To enable the components, the Android system must be aware of the components existence. The existence of the components are defined in the manifest file (\verb|AndroidManifest.xml|), which describes the component and the interactions between them, as well as describe the permission of the application. 


\subsubsection{Activity}
An application can conists of multiple acitvities, and an activity represents a single screen with a user interface \cite{activities}. Applications with multiple activities has to mark one of the activities as main activity, which will be presented to the user on launch. The user interface of an activtiy is constructed in layout files which define the interaction logic of the user interface, and the layout file is inflated into the activity on launch. 

Activites are placed on a stack, and the activity on top of the stack becomes the running activity. Previous activties remain in the stack (unless disgarded), and are brought back if desired.  An activity can exist in three states cite[activites]:

\begin{itemize}
    \item \textbf{Resumed (Running)}: The activity is in the foreground of the screen and has user focus. 
    \item \textbf{Paused}: Another activity is running, but the paused activity is still visible. For instance, the other acitvity does not cover the whole screen. A paused activity maintain its state, but can be killed by the system if the memory situtation is critical.  
    \item \textbf{Stopped}: The activity is obscured by another activity. A stopped activity maintain its state; however, it is not visisible to the user and can be killed if the memory situtation is critical.
\end{itemize}
Paused and stopped activities can be terminated due to insufficient memory by asking the activity to finish. When the paused or stopped activity is re-opened, it must be created all over. 

Activites are part of a activity lifecycle, in Figure \ref{fig:lifecycle}, the state of the activity can be vaugy categorized into:
\begin{itemize}
    \item \textbf{Entire Lifetime}: of an activity occurs between the calls to \verb|OnCreate()| and the call to \verb|OnDestroy()|. The activity sets the states (e.g., defining the layout) in \verb|OnCreate()|, and release remaning resources in \verb|OnDestroy()|.
    \item \textbf{Visible Lifetime}: of an activity happens between the calls to \verb|onStart()| and the call to \verb|onStop()|. Within this lifecycle, the user can see and interact with the application. Any resources that impact or effect the application occurs between these metods. As activties can alternative between state, the system might call these methods multiple times during the lifecycle of the activity.
    \item \textbf{Foreground Lifetime}: of an activity occurs between the calls to \verb|onResume()| and \verb|onPause()|. The activity is on top of the stack, and has user input focus. An activity can frequently transition in this state; therefore, ensuring that the code in these methods are lightweight in order to prevent the user from waiting. 
\end{itemize} 

\begin{figure}
    \centering
    \includegraphics[scale=0.6]{images/androidlifecycle.png}
    \caption{Recording}
    \label{fig:lifecycle}
\end{figure}

\noindent \textbf{Fragment}

\noindent A fragment represents a behavior or is a part of a user interface that can be placed in an activity \cite{fragments}. Fragment allows for reuse of user interface or behavior across applications, and can be combinted to build a multi-pane user interface inside an activity. The fragment allows for more flexibility around the user interface, by allowing activties to comprise of multiple fragments which will have their own layout, events, and lifecycles. The lifecycle of a fragment is quite similar to the activiy lifecycle; with extended states for: fragment attachment/deattachment, fragment view creation/destruction, and host activity creation/destruction. A fragment is cooherent with its host activity, and the state of the fragment is affected by the state of the host activity. Fragment creation and interactions is done through \verb|FragmentManager|.  

\subsubsection{Service}
Service is a component that runs in the background to perform long-running tasks \cite{services}. The application or other applications can start a service which remain in the background even if the user switches applications. In contrast, activites are not able to continue if user switches to another application. Also, a service can bind with a component to interact or perform inter-process communication (IPC). To summarize, a service has two forms:
\begin{itemize}
    \item \textbf{Started}: A component can call the \verb|startService()| method on a service, such that the service can run in the background. 
    \item \textbf{Bound}: A component can call the \verb|bindService()| method on a service, which in return will offer a client-server interface to perform operations (e.g., sending requests, or retrieving results) across processes with inter-process communication (IPC). Multiple component can bind to a service, and the last component to \verb|unbind| will destroy the service. 
\end{itemize}

\subsubsection{Broadcast Receiver}
A broadcast receiver is a component that receives broadcast announcements mostly originating from the system (e.g., screen turned off, battery is low, or a picture was caputed). Applications can subscribe to messages, and the \verb|BroadcastReceiver| can address and process the messages accodingly. Applications can also initiate broadcasts, and the data is delivered as an \verb|Intent| object. A \verb|BroadcastReceiver| can be registered in the activity of the application (with \verb|IntentFilter|), or inside of the manifest file. 

\subsubsection{Content Providers}
Content provides manage access to a set of structured data, and provide mechanism to encapsulate and secure the data \cite{contentproviders}. Content providers is an interface which enables one process to connect its data with another process. Also, in order to copy and paste complex data or files between applications, a content provider is required. For instance, to share a file across a media (e.g., mail), a \verb|FileProvider| (subclass of \verb|ContentProvider|) is needed to facilitate a secure sharing of files \cite{fileprovider}.


\subsection{Process and Threads}
The Android system creates a Linux process with a single thread of execution for an application on launch \cite{proccessandthread}. All components (i.e., activity, service, broadcast receiver, and providers) run in the same process and thread (called the \textit{main} thread), unless the developer arrange for components to run in separate process. A process can also have additional threads for processing. 

When the memory on the device runs low and demanded by processes which are serving the user, Android might kill low priority processes. Android decides to kill the process based on priority; the process hierarchy consits of five levels (lowest priority number is the most important and is killed last):  

\begin{enumerate}
    \item \textbf{Foreground Process}: is a process that is required by the user to interact and function with the application. A foreground process is a often activity that the user interacts with, service that is bound to an interacting activity, service that is running in the \verb|foreground| (with \verb|startForeground()|), service that is executing on of the lifecycle callbacks, and broadcast receivers executing \verb|onReceive()| method.
    \item \textbf{Visible Process}: is a process without foreground components, but affect the user interactions. A visible process is when a foreground process takes controll (however, the visible process can be seen behind it), and a service that is bound to a visible (or foreground) activity. 
    \item \textbf{Service Process}: is a process that execute work which is not displayed to the user (e.g., playing music or downloading data), and are started with the \verb|startService()| method. 
    \item \textbf{Background Process}: is a process that holds information of paused activites. This process state has no impact on the user experiece, and these process are kept in a LRU (least recently used) in order to refrain killing the activity that the user used last. The state of the process can be saved, if the lifecycle method in activity is implemented correctly, to ensure a seamless user experience. 
    \item \textbf{Empty Process}: is a process that do not hold any active application components; however, are kept alive for caching and faster startup time for components that needs to be executed.  
\end{enumerate}

\noindent \textbf{Threads}

\noindent The main thread is responsible for dispatching events to the user interface widget and drawing events. Also, the thread interacts with the application components from the Android UI toolkit; the main thread is also called UI thread. System calls to other components are dispatched from the main thread, and components that run in the same process are instantiated from the main thread. Intensive work (such as long-running operations as network access or database queries) in response to user interaction, can lead to blocking of the user interface. As a consequence, the user can find the application to hang, and might decide to quit or uninstall the application. 

Additionally, the tool kit to update the user interface in Android is not thread-safe; therefore, enforcing the rules of 1) not to block the UI thread; and 2) not to access the Android UI toolkit outside the UI thread. In order to run long-running or blocking operations, one can spawn a new thread or Android provides serveral options: \verb|runOnUiThread|, \verb|postDelay|, and \verb|AsyncTask| (perform asynchornous task in a worker thread, and publishes the results  on the UI thread). 

\subsection{Inter-Process Communication (IPC)}
Inter-process communication is a mechanism to perform remote procedure calls (RPC) to application components that are executed remotely (in another process), with results returned back to the caller. To perform IPC, the caller has to bind to a remote \verb|service| (using \verb|bindService()|). Upon binding to a remote service, a proxy used for communication with the remote service is returned. The proxy decomposes the method calls, and the Binder framework takes the methods and transferrs them to the remote process \cite{binder}. Android offers a language to enable IPC; called Android Interface Definition Language (AIDL) \cite{aidl}. 

Besides AIDL, one can use \verb|Intent| to pass messages across processes. Intent is a messaging object used to request action from another application components \cite{intents}. There are two types of intents: 1) \verb|explicit intents|: used to start a component in the application, by supplying the application package name or component class; and 2) \verb|implicit intents|: declare a genereal action to perform, which enables other applications to handle it. The main uses cases of an intent are to starting an activity; starting a service; and delivering a broadcast.  

\subsection{Data and File Storage}
Android provides options to store application data on the device, depending on space requirement, data type, and whether the data should be accessible to other application or private to the application \cite{filestorage}. There are four distinctive data storage options, depending on the requirement of data that is being stored:
\begin{itemize}
    \item \textbf{Interal File Storage}: The system provides a directory on the file system for the application to store and access files. By default, the files saved in this directory is private to the application. Also, files stored in the interal storage are removed on uninstallation of the application; therefore, storing persistent data that are expected to be on the device regardless of the application removal, should not be using interal file storage. In addition, internal file storage allows for caching, which enables temprarily data storage (that do not require to be persistent). 
    \item \textbf{External File Storage}: External file storage enables storing of files in a space where users can mount to a computer as an external storage device (or to a physicall removeable storage, such as SD card). Files stored into a external storage makes it possible to transfer files on a computer over USB. Files stored in a external file storage enables other applications to access the data, and the data remain available after application uninstallation (unless specified that the stoarge is application-specific). 
    \item \textbf{Shared Preferences}: For storing small and unstructured data, \verb|SharedPreference| enables API to read and write persistent key-value pairs of primitive data types (e.g., booleans, floats, ints, longs, and strings). The storage location is specified by uniquly identifying name, and the data is stored into a XML file. Also, the data stored remain persistent (even after application termination). 
    \item \textbf{Databases}: Android provides support for SQlite databases, which is a relation database management system embedded into the system. The access to the database is private to the application, and accessing the database can be done with the \verb|Room persistence library|. The Room library provides an abstract layer over SQLite APIs.  
\end{itemize}


\subsection{Architecture Patterns}

\begin{figure}
    \centering
    \includegraphics[scale=0.7]{images/MVVM.pdf}
    \caption{Entity Relationship Diagram}
    \label{fig:mvvm}
\end{figure}

The architectural pattern principle enhances the separation of \textit{graphical user interface} logic from the \textit{oparting system} interactions \cite{architecture}. The Model-View-ViewModel (hereafter: MVVM) is an architectural pattern which is well-integrated and incentived by Android. It has three components that consititute the principle:
\begin{itemize}
    \item \textbf{Model}: represents the data and the business logic of the application. 
    \item \textbf{ViewModel}: interacts with the model, and manages the state of the view.
    \item \textbf{View}: handles and manages the user interface of the application.
\end{itemize}

In Figure \ref{fig:mvvm}, the interactions amongst the components are illustrated. The connection between the \verb|View| and \verb|ViewModel| occurs over a data binding connection, which enables the view to change automatically based on changes to the binding of the subscribed data \cite{mvvm}. In Android, the \verb|LiveData| is an obserable data holder that enables data binding, which allows components to observe for data changes. \verb|LiveData| respects the lifecycle of the application components (e.g., activities, fragments, or services), ensuring the \verb|LiveData| only updates the components that are in an active lifecycle state \cite{livedata}. Moreover, \verb|Android Room| provides set of components to facilitate the structure of the model component \cite{room}. More spesifically, it models a database and the entities (which are the tables in the database).

\subsection{Power Management}
Battery life is a concern to the user, as the battery capacity is significantly limited on devices \cite{powermanagement}. Android has feastures to extend battery life by optimizing the behavior of the application and the device, and provides several techniques to improve battery life:
\begin{itemize}
    \item \textbf{App Restrictions}: Users can choose to restrict applications (e.g., applications cannot use network in the backgroud, and foreground services). The application that are restricted, functiontions as expected when the user launches the application; however, are restricted to do background tasks. 
    \item \textbf{App Standby}: An application can be put into standby mode if an user is not actively using it, resulting in the application background activity and jobs is postponed. An application is in standby mode if there is no process in the foreground, a notification is being viewed by the user, and not explicitly being used by the user. 
    \item \textbf{Doze}: When a device is unused for a long period, applications are delayed to do background activites and tasks. The doze-mode enters maintancen window to complete pending work, and then resume sleep for a longer period of time. This cycles through until a maximum of sleep time is reached. Some applications wants to keep the device running to perform long-running tasks (e.g., collecting data), and \verb|WakeLocks| enables this. \verb|WakeLocks| allows application to perform activties and tasks, even while the screen is turned off \cite{wakelocks}. 
    \item \textbf{Exemptions}: Another way of keeping an application awake, is to exemtping applications from being forced to \verb|Doze| or to be in \verb|App Standby|. The exmpted applications are listend in the settings of the device, and users can manualy choose application to exempt. Consequently, exempted applications might overconsume the battery of the device. 
\end{itemize}


\subsection{Bluetooth Low Energy}
Android supports for Bluetooth Low Energy (BLE), which is desgned to provide lower power consumption on data transmittion (in contrast to classic Bluetooth) \cite{bluetoothle}. BLE allows Android applications to communicate with sensors or devices (e.g., heart rate sensor, and fitness devices), that has a stricter power requirements. Sensors that utilize BLE, are designed to last for a longer period of time (e.g., weeks or months before needing to charge or replace battery). The protocol of BLE is optimized for small burst of data exhange, and terms and concepts that form a BLE can be characterized as:
\begin{itemize}
    \item \textbf{Generic Attribute Profile (GATT)}: As of now, all Low Energy applications are based on GATT. GATT is a general spesification for sending and receiving burst of data (known as \textit{attributes}) over a BLE link. 
    \item \textbf{Attribute Protocol (ATT)}: GATT utilites the Attribut Protocol, which uses a few bytes as possible to be uniquely identified by a Universall Unique Identifer (UUID). An UUID is a standardized format for identify information.
    \item \textbf{Characteristic}: Contains a single value and descriptors that describe the characterstics value (i.e., can be seen as a type). 
    \item \textbf{Descriptor}: Are defined attributes that describe a characterstic value (e.g., specify a human-readable description, a range of acceptable values, or a unit of measure).
    \item \textbf{Service}: Is a collection of characterstics (e.g., a service is \textit{Heart Rate Monitor} which includs a characterstic of \textit{heart rate measurement}).
\end{itemize}
In order to enable BLE, factiliting for a GATT server and GATT client is required. Either the sensor or the device take the role of being a server or a client. However, the GATT server offers a set of services (i.e., features), where each service has a set of characterstics. And the GATT client can subscribe and read from the services the GATT server provides. 
